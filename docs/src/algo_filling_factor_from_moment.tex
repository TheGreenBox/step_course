\newpage
\part{ Алгоритм получения управляющих воздействий для кусочно линейной траектории момента }

Расчитаем для случая управления с ОС двухфазным ШД. Параметризуем угол коммутации и обозначим
$\lambda_c$.
Согласно формуле \ref{torque_from_rotor_deviation} момент действующий на ротор в начале импульса
управления:

\begin{equation}
    \label{moment_to_rotor_at_the_begin_of_control_pulse}    
    \tau_{begin} = K_{T} I_{M} N_{r} (\theta_{i} - \theta_{0})
\end{equation}

в конце импульса:

\begin{equation}
    \label{moment_to_rotor_at_the_end_of_control_pulse}    
    \tau_{end} = K_{T} I_{M} N_{r} (\theta_{i} - \theta_{0})
\end{equation}

Средний момент для этого участка:
$$
    \tau_{step} = \frac{ \tau_{begin} + \tau_{end} }{ 2 }
$$

Зная требуемый момент получим требуемый среднеинтегральный ток на этом шаге кусочно линейной
траектори момента. 

Далее возникает очевидно необходимость получения коэффициента ШИМ для заданного тока. Вообще говоря
получить требуемый ток на участке множеством разных способов, к примеру получить требуемый ток
максимальным коэффициентом ШИМ в начале одного импульса, далее поддерживать нужный уровень тока
требуемой величины, а к концу импульса обратным напряжением кам можно быстрее снизить ток до нуля.

Для простоты рассмотрим управление когда на протяжении всего импульса управления подаётся постоянный
уровень коэффициента заполнения ШИМ. Расмотрим кривую нарастания тока в течение импульса управления.

Согласно формулам \ref{winding_current_with_pwm_control_1} 
и \ref{winding_current_with_pwm_control_0} эта кривая ни что иное, как совокупность чередующихся
участков нарастания и спада тока. Предположим и докажем что асимптотическое нарастание тока до
некоторого уровня зависящего от коэффициента заполнения $\zeta$ осуществляется за количество периодов ШИМ
$n_{crit}$ независящее от коэффициента заполнения. 

Максимальное значение тока в импульсе ШИМ согласно \ref{winding_current_with_pwm_control_1} при
$\varepsilon=\zeta$


