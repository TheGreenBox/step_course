\newpage
\part{ Основные определения }

Максимальная частота приемистотости - максимальная управляющая частота, при которой не нагруженный
двигатель может останавливаться и запускаться без пропуска шагов.

Максимальная выходная чатота вращения - максимальная шаговая частота вращения, при которой
ненагруженный двигатель может двигаться без пропуска шагов.

Максимальный пусковой момент - максимальный момент сопротивления нагрузки, с которой двигатель может
запускаться и сохранять синхронность при частоте импульсов в 10 Гц.

Удерживающий момент шагового двигателя - максимальный статический момент который может быть приложен
к валу возбужденного двигателя без последующего вращения. 

Фиксирующий момент шагового двигателя - максимальный статический момент который может быть приложен
к валу невозбужденного двигателя без последующего вращения. 

