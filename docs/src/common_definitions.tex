\subsection{ Основные определения }

\begin{itemize}

    \item \textit{Максимальная частота приемистотости} - максимальная управляющая частота,
    при которой не нагруженный двигатель может останавливаться и запускаться без пропуска шагов.

    \item \textit{Максимальная выходная чатота вращения} - максимальная шаговая частота вращения,
    при которой ненагруженный двигатель может двигаться без пропуска шагов.

    \item \textit{Максимальный пусковой момент} - максимальный момент сопротивления нагрузки,
    с которой двигатель может запускаться и сохранять синхронность при частоте импульсов в 10 Гц.

    \item \textit{Удерживающий момент шагового двигателя} - максимальный статический момент
    который может быть приложен к валу возбужденного двигателя без последующего вращения.

    \item \textit{Фиксирующий момент шагового двигателя} - максимальный статический момент
    который может быть приложен к валу невозбужденного двигателя без последующего вращения.
    
    \item \textit{Положение равновесия или конечное положение} - положения, в которых
    останавливается ротор возбужденного, ненагруженного двигателя.
    
    \item \textit{Положения фиксации} - положения в которых останавливается ротор шагового
    двигателя, имеющий постоянный магнит в невозбужденном состоянии и при отсутствии нагрузки.

    \item \textit{Ошибка углового положения} - максимальная положительная или отрицательная ошибка
    углового положения (по сравнению с нормированным углом шага), которая наблюдается при движении
    ротора из одного положения равновесия в другое.
    
    \item \textit{Точность позиционирования} - максимальная ошибка углового положения для конечного
    положения, относящегося ко всему набору нормированных углов шага, которые выполняются за полный
    оборот ротора.
    
\end{itemize}

