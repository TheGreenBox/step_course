\subsection{Расчёт параметров привода}

\subfile{src/retarget_parameters}

% Сдесь было в названии "свободная табл..." мне показлось что лучше использовать "сводная табл..."
% Могу оказаться не прав
\subsubsection{Сводная таблица результатов}

\begin{table}[h!]
    \centering
    \begin{tabular}{|l|c|l|}
        \hline
        Параметр                                    & Обозначение      & Значение                           \\
        \hline
        Напряжение питания                          & $U_1$            & 24В                                \\
        Шаг единичного углового перемещения нагрузки& $q_d$            & $ \le 1' $                         \\
        Ресурс                                      &                  & 30 000 ч.                          \\
        Диапазон углового перенацеливания нагрузки  & $q_{max}$        & $[-135^\circ .. 135^\circ] $       \\
        Максимальная угловая скорость нагрузки      & $\dot{q}_{max}$  & $1.73$ рад / c                     \\
        Максимальное ускорение нагрузки             & $\ddot{q}_{max}$ & $0.872$ рад /$c^2$                 \\
        Момент инерции нагрузки                     & $J_{\text{ОУ}}$  & $3 ~\text{кг} \cdot \text{м}^2 $   \\
        \hline
    \end{tabular}
    \caption{Требуемые параметры привода}
    \label{drive_parameters_tbl}
\end{table}

\endinput
