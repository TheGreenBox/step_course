\newpage
\subsection{Цели работы}

\begin{enumerate}
    \item{Разработка алгоритма управления ШД с обратной связью}
    \begin{enumerate}
        \item Определение коэффициета заполнения ШИМ внутри импульса управления
        \item Определение угла коммутации для текущей скорости в режимах:
            \begin{itemize}
                \item Разгон, ускорение больше нуля
                \item Торможение, ускорение меньше нуля
                \item Поддержание постоянной скорости, ускорение стремится к нулю
            \end{itemize}
        \item Алгоритм переключения фаз при работе с заданным углом коммутации
        \item Алгоритм для реализации обратной связи по току
    \end{enumerate}

    \item{Разработка алгоритма управления ШД без обратной связи}
        \begin{enumerate}
            \item Определение коэффициета заполнения ШИМ внутри импульса управления
            \item Определение предельной скорости, ниже которой двигатель гарантировано не выйдет из
                синхронизма при постоянной динамической нагрузке
        \end{enumerate}

    \item{Изучение вопроса об использовании ОС по току, обоснование выбора}
        \begin{enumerate}
            \item Требования к реализации
            \item Анализ ограничений
            \item Исследование встроеного в микроконтролер АЦП
            \item Анализ путей решения зашумленности канала АЦП
        \end{enumerate}

    \item{Выявление и изучение паразитных резонансных явлений в шаговом двигателе}
        \begin{enumerate}
            \item Анализ возможных причин
            \item Выявление особо критичных параметров для качетсва системы управления
            \item Методы борьбы, применимость для данной модели
        \end{enumerate}

    \item{Разработка и валидация математических моделей выбранного ШД}
        \begin{enumerate}
            \item Изучение готовой математической модели из пакета MATLAB Simulink
            \item Разработка своей математическая модель в MATLAB Simulink на основе
                    уравнений электрических процессов в фазах статора
            \item Проверка работоспособности и эффективности алгоритмов управления с заданным углом
                    коммутации на математических моделях
            \item Моделирование явлениий среднечастотного резонанса
            % не убирать, в этот раз бы не успели просто это, надо не забыть
            %\item Разработка методики валидации моделей по результатам натурного моделирования
        \end{enumerate}
\end{enumerate}

\newpage
