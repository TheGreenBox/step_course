\documentclass{article}

\usepackage[utf8]{inputenc}

% Подключение гиперссылок
\usepackage{xcolor}
\usepackage[unicode]{hyperref}

% Настройка внешнего вида гиперссылок
\definecolor{LINKCOLOUR}{rgb}{0.1,0.0,0.9}
\hypersetup{colorlinks,breaklinks,urlcolor=LINKCOLOUR,linkcolor=LINKCOLOUR}

% Выбор внутренней TEX−кодировки
\usepackage [T2A]{fontenc}
% Включение переносов для русского и английского языков
\usepackage[english,russian]{babel}

% Начинать первый параграф раздела следует с красной строки
\usepackage{indentfirst}


% Дополнительные математические пакеты
%\usepackage{weird,querr}
\usepackage{amssymb}
\usepackage{amsmath}

% Для корректного копирования из документа
\usepackage{cmap}

% Кто нибудь помнит зачем это тут? A: Неа...
\usepackage{multirow}

% Меняем поля страницы
\usepackage{geometry}
\geometry{left=2cm}    % левое поле
\geometry{right=1.5cm} % правое поле
\geometry{top=2cm}     % верхнее поле
\geometry{bottom=2cm}  % нижнее поле

\usepackage{subfiles}

% работа с импортом изображений
\ifx\pdfoutput\undefined
\usepackage{graphicx}
\else
\usepackage[pdftex]{graphicx}
\fi

% секции и их структура
\usepackage[section]{placeins}
\usepackage{subcaption}

\begin{document}

%\maketitle
\begin{titlepage}
\begin{center}
    {\large Московский Государственный Технический Университет им. Н.Э.Баумана}
    \\[50mm]
    {\LargeКурсовой проект}
    \\[7mm]
    {\LARGE ``Система управления \\ приводами двухстепенного манипулятора \\ на основе шаговых двигателей''}
    \\[37mm]

    \begin{flushright}
        \begin{minipage}{0.5\textwidth}
            \begin{flushleft}
                \textit{Авторы:} \\
                ~Киндяков Александр \\
                ~Никитин Сергей \\[10mm]
                \textit{Руководитель:} \\
                ~Бошляков Андрей Анатольевич
            \end{flushleft}
        \end{minipage}
    \end{flushright}

    \vfill % Заполнить все доступное пространство
    Москва, 2013 г. \\
    \LaTeX
\end{center}
\end{titlepage}

\tableofcontents
\newpage

\subfile{src/common_using_symbol}
\subfile{src/task_description}
\subfile{src/intermediate_task_list}
\subfile{src/theoretical_base}
\subfile{src/modeling}
\subfile{src/current_feedback_description}
\newpage

\section{Заключение}

В результате проведенной работы была построена математическая модель шагового двигателя, проведен
анализ ее ограничений, выполнено сравнение характеристик с моделью шагового двигателя из пакета MatLab
\textit{SimPower Systems -> Second Generation -> Motors and Generators -> Hybrid Stepper Motor}.
Характеристики построенной нами модели полностью совпали с характеристиками модели из пакета MatLab.

Проведен анализ существующих способов управления шаговым приводом, выбраны два оптимальных метода для
режима работы - с обратной связью по углу и без обратной связи (аварийный режим работы).

Составлена блок-схема алгоритма работы системы управления с обратной связью, проведен анализ ее
ограничений.

\subfile{src/bibliography}

\end{document}
