\documentclass{article}

\usepackage[utf8]{inputenc}

% Подключение гиперссылок
\usepackage{xcolor}
\usepackage[unicode]{hyperref}

% Настройка внешнего вида гиперссылок
\definecolor{LINKCOLOUR}{rgb}{0.1,0.0,0.9}
\hypersetup{colorlinks,breaklinks,urlcolor=LINKCOLOUR,linkcolor=LINKCOLOUR}

% Включение переносов для русского и английского языков
\usepackage[english,russian]{babel}

% Начинать первый параграф раздела следует с красной строки
\usepackage{indentfirst}

% Выбор внутренней TEX−кодировки
\usepackage [T2A]{fontenc}

% Дополнительные математические пакеты
%\usepackage{weird,querr}
\usepackage{amssymb}
\usepackage{amsmath}

% Для корректного копирования из документа
\usepackage{cmap}

% Кто нибудь помнит зачем это тут?
\usepackage{multirow}

% Меняем поля страницы
\usepackage{geometry}
\geometry{left=2cm}    % левое поле
\geometry{right=1.5cm} % правое поле
\geometry{top=2cm}     % верхнее поле
\geometry{bottom=2cm}  % нижнее поле

\usepackage{subfiles}

\begin{document}

%\maketitle
\begin{titlepage}
\begin{center}
{\largeМосковский Государственный Технический Университет им. Н.Э.Баумана}
\\[50mm]
{\LARGEКурсовой проект}
\\[7mm]
{\LARGE << <Название> >>}
\\[37mm]

\begin{flushright}
    \begin{minipage}{0.5\textwidth}
        \begin{flushleft}
            \textit{Авторы:} \\ 
            ~Никитин Сергей \\
            ~Киндяков Александр \\[10mm]
            \textit{Руководитель:} \\
            ~Бошляков Андрей Анатольевич
        \end{flushleft}
    \end{minipage}
\end{flushright}

\vfill % Заполнить все доступное пространство
Москва, 2013г. \\
\LaTeX
\end{center}
\end{titlepage}

\tableofcontents
\newpage
\subfile{src/score_statement}
\subfile{src/intermediate_task_list}
\subfile{src/theoretical_base}
\subfile{src/modeling}

\newpage
\section{Заключение}

\subfile{src/bibliography}

\end{document}

