\newpage
\section{Экспериментальная часть}
\subsection{Макет экспериментального стенда}

Макет состоит из спроектированного механического стенда
(см. \ref{sec_stand_construction}) и макетной платы
Texsas~Instruments~C2000~Piccolo~F28035.
В качестве шагового двигателя используется двигатель, запланированный для работы
в реальном приводе~--- Telco~Motion~4T5618M308 (табл. \ref{engine_params}), в
качестве датчика~--- импульсный датчик ЛИР--137А 6250--05--ПИ
(табл. \ref{encoder_params}).

Схема экспериментального стенда изображена на рис.
\ref{pic_experim_stand_scheme}, где

\begin{enumerate}
       \item Плата управления
       \item Источник питания
       \item Шаговой двигатель
       \item Объект регулирования
       \item Датчик углового положения
       \item Муфта соединительная
       \item Муфта соединительная малая
\end{enumerate}

\begin{figure}[hb]
    \centering
    \begin{tikzpicture}[scale=1.5]
        \draw (0,0) rectangle (10,10);
        \node at (5,5) {Шаблон для схемы испытательного стенда};
    \end{tikzpicture}
    \caption{Схема экспериментального стенда}
    \label{pic_experim_stand_scheme}
\end{figure}
\subsubsection{Описание экспериментов}

\subsubsection{Результаты экспериментов}

\subsubsection{Выводы по экспериментальной части}
