\newpage
\section{Исследовательская часть}

\subfile{src/research/goals}
\subfile{src/research/theoretical_base}

\subfile{src/drive_parameters}

\newpage
\subsection{Параметры составных частей системы}
\subfile{src/controlled_object_desc}

\subsubsection{Двигатель}
На этапе энергетического расчета в качестве двигателя привода был выбран
гибридный шаговый двигатель Telco Intercon 4T5618M308.

\begin{table}
    \centering
    \begin{tabular}{l|l}
    \hline
    Шаг, $\degree$                          & 1.8                   \\
    Момент удержания, Н$\cdot$м             & 1.274                 \\
    Напряжение, В                           & 2.3                   \\
    Ток, А/фаза                             & 3.0                   \\
    Сопротивление, Ом/фаза                  & 0.77                  \\
    Индуктивность, мГн/фаза                 & 1.65                  \\
    Момент инерции ротора, кг*м2            & $2.8 \cdot 10^{-5}$   \\
    Масса двигателя, кг                     & 0.7                   \\
    \hline
    \end{tabular}
    \label{engine_params}
\end{table}

\subsubsection{Редуктор}
На этапе энергетического расчёта для сопряжения двигателя и нагрузки
был выбран редуктор с передаточным отношением $i_\text{ред} = 225$. Датчик
устанавливается непосредственно на вал нагрузки, без использования редуктора.

\subsubsection{Датчик}
В качестве датчика положения используется импульсный датчик ЛИР-137А 6250-05-ПИ.

\begin{table}
    \centering
    \begin{tabular}{l|l}
    \hline
        Разрешающая способность преобразователя, дискрет/оборот         & 6250 \\
        Модификация преобразователя                                     & А \\
        Конструктивное исполнение                                       & 5 \\
        Напряжение питания, В                                           & +5 \\
        Тип выходного сигнала                                           & ПИ (TTL) \\
        Интервал рабочих температур, $\degree$С                         & 0..+70 \\
        Класс точности                                                  & 7 класс $\pm75''$ \\
        Масса (без кабеля), кг                                          & 0.03 \\
        Степень защиты от внешних воздействий                           & IP50 \\
        Максимальная скорость вращения вала, об/мин                     & 10000 \\
        Вибрационное ускорение в диапазоне частот 55..2000 Гц, м/с$^2$  & $\leq$ 100 \\
        Момент трогания ротора при температуре 20$\degree$С, Н$\cdot$м  & $\leq 1 \cdot 10^{-3}$ \\
        Ударное ускорение, м/с$^2$                                      & $\leq$ 300 \\
        Момент инерции ротора, кг$\cdot$м$^2$                           & 3 $\cdot 10^{-7}$ \\
        Допустимое осевое смещение вала, мм                             & $\pm$ 0.02 \\
        Допустимое радиальное смещение вала, мм                         & $\pm$ 0.02 \\
        Диаметр корпуса, мм                                             & 37 \\
    \hline
    \end{tabular}
    \label{encoder_params}
\end{table}

\subsection{Функциональная схема лабораторного мехатронного модуля}
\begin{figure}[h!]
    \centering
    % \includegraphics[width=\textwidth, keepaspectratio, clip=true, trim=3cm 3cm 3cm 3cm]
    %                 {./src/pictures/stand_functional_scheme}
    \caption{Функциональная схема лабораторного мехатронного модуля}
    \label{stand_functional_scheme}
\end{figure}

\newpage
\subfile{src/research/math_modeling}

\newpage
\subfile{src/research/current_feedback_description}

\newpage
\subsection{Выводы по исследовательской части}
