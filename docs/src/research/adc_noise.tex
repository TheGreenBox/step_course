\begin{figure}[ht]
    \centering
    \begin{tikzpicture}[font=\small]
        \begin{axis}[
            ybar,
            bar width=20pt,
            xlabel={Показания АЦП},
            ylabel={Число попадани},
            ymin=0,
            ytick=\empty,
            xtick=data,
            axis x line=bottom,
            axis y line=left,
            enlarge x limits=0.2,
            symbolic x coords={2054, 2055, 2056, 2057, 2058, 2059, 2062},
            xticklabel style={anchor=base,yshift=-\baselineskip, rotate=90},
            nodes near coords={\pgfmathprintnumber\pgfplotspointmeta}
        ]
        \addplot[fill=white] coordinates {
            (2054, 29)
            (2055, 103)
            (2056, 473)
            (2057, 278)
            (2058, 240)
            (2059, 25)
            (2062, 1)
        };
        \node[align=left] at (450, 400) {   $m_\text{ацп} = 2056.59$    \\
                                            $p_\text{ацп} = 2062$
                                        };
        \end{axis}
    \end{tikzpicture}
    \caption{Гистограмма показаний АЦП}
\end{figure}

Для определения числа незашумлённых бит АЦП, был проведён соответствующий
эксперимент в условиях, рекомендуемых в \cite{TheGoodTheBadAdcAspects}.
Макетная плата была запитана лабораторным источником питания.
На наблюдаемый токовый канал АЦП был подан ШИМ амплитудой 24 В с коэффициентом
заполнения $\zeta = 1$.

Показания АЦП были собраны с помощью профилировщика, прилагающегося к контролеру
\foreignlanguage{english}{Texas Instruments}.

Получим число незашумлённых бит АЦП, подставив результаты эксперимента в
формулу (\ref{eq_noise_free_code_resolution}).

$$
    N_{\text{незашум}} = \log_{2} \frac{2^{12}}{\abs{2056.59 - 2062}} \simeq 9
$$
