\newpage
\section{ Описание обратной связи по току }
Согласно требованиям задачи необходимо обеспечить скорости переброски системы не менее 
$ \dot{q}_{max} = 1.73$ рад/c, при выбранном на этапе энергетического расчета передаточном
отношении редуктора $ i{gb} = 225 $. Отсюда очевидно, что минимальное время импульса управления:

$$
    T_{ctrl} = \frac{ 2 \cdot \pi }{ \dot{q}_{max} \cdot i_{gb} \cdot N_{r} \cdot p_{sm} \cdot 2 }
$$
$$
    T_{ctrl} = \frac{ 2 \cdot \pi }{ 1.73 \cdot 225 \cdot 50 \cdot 2 \cdot 2 } = 8.07 \cdot 10^{-05}
$$

Что на 2 порядка отличается от электрической постоянной времени фазы $( \simeq2 \cdot 10^{-3} )$.
Для обеспечения максимально возможного момента на скоростях близких к максимальной необходимо
быстрое нарастание тока, как было доказано ранее (\ref{ current_grow_estimate }), время роста можно
уменьшить, если увеличить напряжение.

\subsection{ Описание аппаратной части реализации токовой обратной связи }

\subsubsection{ Схематика аппаратной части }

\subsubsection{ Аналого-цифровое преобразование }

\subsubsection{ Методика оценки зашумленности сигналов с АЦП }

\endinput

