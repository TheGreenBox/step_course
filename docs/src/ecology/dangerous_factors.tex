\newpage
\subsubsection{Анализ вредных и опасных факторов при разработке}

В соответствии с ГОСТ 12.0.003-74 \cite{ecology_gost_003_74}, все опасные и
вредные производственные факторы подразделяются на физические, химические,
биологические и психофизиологические.

\begin{itemize}
    \item   \textit{физические факторы} - поражение электрическим током, недостаточную
            освещенность, электромагнитные поля, ионизирующие излучения, недопустимые
            уровни шума, вибрации, инфра- и ультразвука, и др.
    \item   \textit{химические факторы} - представляют собой вредные для организма
            человека вещества в их различных состояниях
    \item   \textit{биологические факторы} - воздействия различных микроорганизмов,
            а также животных и растений
    \item   \textit{психофизиологические факторы} - физические и эмоциональные
            перегрузки, монотонность труда, умственное перенапряжение

\end{itemize}

Рассмотрим перечисленные факторы подробнее.

\paragraph{Химические факторы}

Согласно СанПиН 2.2.2/2.4.1340-03 \cite{ecology_sanpin_1340_03} содержание вредных
химических веществ в производственных помещениях, в которых работа с использованием
ПЭВМ является основной, не должно превышать предельно допустимых концентраций
загрязняющих веществ в атмосферном воздухе населенных мест в соответствии с
действующими гигиеническими нормативами 2.1.6.1338-03
\cite{ecology_hygiene_norm_1338_03}.

Наличие опасных и вредных химических факторов в помещениях с ПЭВМ в основном
обусловлено широким применением полимерных и синтетических материалов для отделки
интерьера, при изготовлении мебели, ковровых изделий, радиоэлектронных устройств
и их компонентов, изолирующих элементов систем электропитания.

В помещении ежедневно проводится влажная уборка, а так же осуществляется непрерывная
очистка воздуха посредствам системы вентиляции. Работа в помещении производится
на современных персональных компьютерах. Так как полимерные материалы, используемые
для внутренней отделки интерьера помещений с ВДТ и ПЭВМ, разрешены для применения
органами и учреждениями Государственного Санитарно~--~Эпидемиологического надзора и
нет источников химически опасных веществ, то химический фактор можно отнести к
допустимому классу условий труда.

\paragraph{Фактор аэрозоли преимущественно фиброгенного действия}

Основным показателем оценки степени воздействия аэрозолей преимущественно фиброгенного
действия (АПФД) на органы дыхания работника является пылевая нагрузка. В случае
превышения среднесменной ПДК фиброгенной пыли расчет пылевой нагрузки обязателен.
Согласно СанПиН 2.2.4.548-96 \cite{ecology_sanpin_548_96}, пылевая нагрузка на
органы дыхания работника – это реальная или прогностическая величина суммарной
экспозиционной дозы пыли, которую работник вдыхает за весь период фактического
(или предполагаемого) профессионального контакта с пылью.

На рабочем месте инженера-проектировщика отсутствуют источники большого количества
пыли, а также проводиться ежедневная влажная уборка и непрерывно работает
приточно~--~вытяжная система вентиляции.

Таким образом, фактор аэрозоли ПФД, при работе в помещении с ВДТ и ПЭВМ можно
отнести к допустимому классу условий труда.

\paragraph{Биологические факторы}

К группе вредных биологических факторов, которые могут привести к заболеванию
или ухудшению состояния здоровья пользователя, относится повышенное содержание
в воздухе патогенных микроорганизмов, особенно в помещении с большим количеством
работающих при недостаточной вентиляции, в период эпидемий.

В помещении проводится ежедневная влажная уборка и непрерывно работает
приточно~--~вытяжная система вентиляции, поэтому биологический фактор можно отнести
к допустимому классу условий труда.

\paragraph{Неионизирующее электромагнитное излучение}

Конструкция ПЭВМ, использованного при разрабокте, обеспечивает надежную
электробезопасность для работающего с ним человека:

\begin{itemize}
    \item   по способу защиты от поражения электрическим током она удовлетворяет требованиям
            1 класса ГОСТ 25861-83 \cite{ecology_gost_25861_83}, ГОСТ 12.2.007.01-75
            \cite{ecology_gost_007_01_75} и ГОСТ Р 50377-92 \cite{ecology_gost_50377_92};

    \item   по обеспечению электробезопасности обслуживающего персонала - ГОСТ 25861-83
            \cite{ecology_gost_25861_83} и ГОСТ Р50377 \cite{ecology_gost_50377_92}.
\end{itemize}

Защита от поражения электрическим током обеспечивается различными способами,
в том числе:

\begin{itemize}
    \item   применением надежных изоляционных материалов;
    \item   использованием кабелей электропитания с заземляющими проводниками;
    \item   размещением разъемов электропитания на тыльной стороне системного блока
            и монитора;
    \item   использованием для электропитания клавиатуры, ручных манипуляторов,
            интерфейсных кабелей и элементов регулировки и индикации, лицевой панели
            системного блока и монитора низковольтных напряжений (не более 12 В).
\end{itemize}

Системный блок и монитор подключены к трехфазной пятипроводной сети переменного тока
с глухим заземлением нейтрали и автоматами защиты от перегрузок, имеющей
напряжение 220 В и частоту 50 Гц, нетоковедущие корпуса монитора и системного блока
заземлены. Основным регламентирующим документом в области требований к рабочему
месту пользователя персонального компьютера является СанПиН 2.2.2/2.4.1340-03
\cite{ecology_sanpin_1340_03}.

При работе с компьютерами на рабочих местах должны соблюдаться более жесткие нормы,
чем установленные ГОСТ 12.1.006-84 \cite{ecology_gost_006_84}, ГОСТ 12.1.002-84
\cite{ecology_gost_002_84}, ГОСТ 12.1.045-84 \cite{ecology_gost_045_84} и
СанПиН 2.2.4/2.1.8.055-96 \cite{ecology_sanpin_055_96}, СанПиН 2.2.4.723-98
\cite{ecology_sanpin_723_98}. Это связано с тем, что указанные нормативно-правовые
акты ориентированы на рабочие места, находящиеся в зоне действия радиотехнических
источников электромагнитных полей, а так же на работы с источниками электростатических
и постоянных магнитных полей.

Наиболее вредным фактором является наличие электромагнитного поля промышленной
частоты (50 Гц), образованного при использовании трехфазной сети переменного тока
с напряжением 220 В.

Вся электротехника, используемая в помещении, прошла сертификацию на соответствие
требуемым нормам безопасности. Таким образом, неионизирующее излучение относится
к допустимому классу условий труда.

\paragraph{Микроклимат}

При продолжительной работе вычислительных машин и их периферийного оборудования
на рабочем месте пользователя происходит выделение избыточной тепловой энергии.
Перегрев окружающей среды неблагоприятно сказывается на человеке. Влияние
температуры на человеческий организм сочетается с влиянием относительной влажности
воздуха.

\begin{table}[h]
    \centering
    \begin{tabular}{|lrrr|}
        \hline
            Период года & Темпер. воздуха, C\degree
        &   Относ. влажность, \% & Скорость движения воздуха, м/с   \\
        \hline
        Холодный    & не более $22 - 24$ & $40 - 60$ & 0,1          \\
        Теплый      & не более $23 - 25$ & $40 - 60$ & 0,1          \\
        \hline
    \end{tabular}
    \caption{Параметры микроклимата в рабочем помещении}
    \label{microclimat}
\end{table}

Параметры микроклимата являются важными, но не критичными при проектировании
рабочего места пользователя ПЭВМ. В помещении, предназначенном для работы с
компьютерами, отсутствуют источники большого количества тепла. В результате меры
и требования по обеспечению микроклимата не отличаются от требований к любому
офисному помещению. Эти требования приведены в документе СанПиН 2.2.2/2.4.1340-03
\cite{ecology_sanpin_1340_03}.

В помещении находится кондиционер, который обеспечивает необходимые параметры
микроклимата, не превышающие допустимых пределов, поэтому фактор микроклимата
относится к допустимому классу условий труда.

\paragraph{Электробезопастность рабочего места}

Конструкция использованного в работе с мехатронным модулем ПЭВМ и используемые
измерительные приборы обеспечивает надежную электробезопасность для работающего
с ним человека:

\begin{itemize}
    \item   по способу защиты от поражения электрическим током удовлетворяет
            требованиям ГОСТ Р МЭК 60950-2002 \cite{ecology_gost_60950_2002}
            и ГОСТ 25124-82 \cite{ecology_gost_25124_82}

    \item   по обеспечению электробезопасности обслуживающего персонала
            соответствует ГОСТ 25861-83 \cite{ecology_gost_25861_83}
\end{itemize}

Системный блок и монитор подключены к трехфазной сети переменного тока напряжением
220 В и частотой 50 Гц, нетоковедущие корпуса монитора и системного блока и лабораторного
оборудования заземлены, а также организовано их зануление.

Макет мехатронного модуля питается от лабораторного источники питания постоянным
напряжение 24 В. Согласно ГОСТ 12.1.038-82 \cite{ecology_gost_038_82}
предельно допустимые значения постоянных напряжений прикосновения при нормальном
режиме электроустановки не должны превышать 10 В, а при аварийных режимах 40 В
при времени воздействия свыше 1 с.

\paragraph{Пожаробезопастность помещения}

Меры по противопожарной защите промышленных предприятий определены стандартами
ГОСТ 12.1.004-91 \cite{ecology_gost_004_91}, а также строительными нормами
и правилами СниП 2.09.02-85 \cite{ecology_snip_02_85}, СниП 2.04.02-84
\cite{ecology_snip_02_84} и другими типовыми правилами пожарной безопасности для
промышленных предприятий.

При эксплуатации электрооборудования не исключена опасность различного рода возгораний.
В современных приборах очень высока плотность размещения элементов электронных
систем, в непосредственной близости друг от друга располагаются соединительные
провода, коммуникационные кабели. При этом возможны оплавление изоляции соединительных
проводов, их оголение и, как следствие, короткое замыкание, сопровождаемое искрением,
которое ведет к недопустимым перегрузкам элементов электронных схем и как следствие
к возгоранию.

Для уменьшения вероятности возгораний сеть питания оборудована автоматами защиты от перегрузок,
так же следует для этих целей соблюдать технику пожарной безопасности при эксплуатации
электрооборудования.

Огнетушители располагаются поблизости от наиболее опасных для возгорания мест.
Все работники предприятия в обязательном порядке проходят инструктаж по пожарной
безопасности.

\paragraph{Психофизиологические факторы}

Для безопасной организации рабочего места, оборудованного дисплеем и персональным
компьютером, и, следовательно, нормальной работы пользователя-разработчика следует
выполнять нижеописанные основные требования санитарных норм и правил согласно
СанПиН 2.2.2/2.4.1340-03 \cite{ecology_sanpin_1340_03}.

Визуальные эргономические параметры ВДТ являются параметрами безопасности,
и их неправильный выбор приводит к ухудшению здоровья пользователя. Все ВДТ
должны иметь гигиенический сертификат, включающий в том числе и оценку визуальных
параметров. Конструкция ВДТ, его дизайн и совокупность эргономических параметров
должны обеспечивать комфортное и надежное считывание отображаемой информации в
условиях эксплуатации. Конструкция ВДТ должна обеспечивать возможность фронтального
наблюдения экрана путем поворота корпуса в горизонтальной плоскости вокруг
вертикальной оси в пределах $\pm 30 \degree$ и в вертикальной плоскости вокруг
горизонтальной оси в пределах $\pm 30 \degree$ с фиксацией в заданном положении.
Дизайн ВДТ должен предусматривать окраску корпуса в спокойные мягкие тона,
обеспечивающие диффузионное рассеивание света.

Корпус ВДТ и ПЭВМ, клавиатура, а так же другие блоки и устройства ПЭВМ должны
иметь матовую поверхность одного цвета с коэффициентом отражения $0,4 - 0,6$ и
не меть блестящих деталей, которые способны создавать блики. На лицевой стороне
корпуса ВДТ не рекомендуется располагать маркировку, органы управления, какие-либо
вспомогательные надписи или обозначения. При необходимости расположения органов
управления или кнопок на лицевой панели, они должны закрываться крышкой или же
быть утоплены в корпус. При работе с ВДТ для профессиональных пользователей необходимо
обеспечивать значения визуальных параметров в пределах рекомендуемого диапазона.

Для профессиональных пользователей разрешается кратковременная работа при допустимых
значениях визуальных параметров. Оптимальные и допустимые значения визуальных
эргономических параметров должны быть указаны в технологической документации на
ВДТ для различных режимов работы пользователей. Конструкция ВДТ должна предусматривать
наличие кнопок регулировки яркости и контраста, которые обеспечивали бы возможность
регулировки этих параметров от минимальных до максимальных значений.

Допускается применение экранных фильтров, специальных экранов и других средств
индивидуальной защиты, прошедших испытания в аккредитованных лабораториях, а так
же имеющих соответствующий гигиенический сертификат. Конструкция ВДТ и ПЭВМ должны
обеспечивать мощность экспозиционной дозы рентгеновского излучения в любой точке
на расстоянии 0,05 м от экрана и корпуса ВДТ при любых положениях регулировочных
устройств не более эквивалентной дозы, равной 1 мкЗв/час.

Конструкция клавиатуры должна предусматривать:

\begin{itemize}
    \item исполнение в виде отдельного устройства, имеющего возможность свободного
            перемещения
    \item опорное приспособление, позволяющее изменять угол наклона поверхности
            клавиатуры в пределах $5 \degree - 15 \degree$
    \item высоту клавиш не более 30 мм
    \item выделение цветом, размером или формой функциональных групп клавиш
    \item размер клавиш должен быть не менее 13 мм, оптимальный – 15 мм
    \item расстояние между клавишами должно быть не менее 3 мм
    \item одинаковый ход для всех клавиш с сопротивлением нажатию $0,25 - 1,5$ Н
\end{itemize}


В производственных помещениях, в которых работа на ВДТ и ПЭВМ является основной,
уровни шума на рабочих местах не должны превышать значений, установленных для
данных видов работ санитарными нормами 2.2.4./2.1.8.562-96
\cite{ecology_sanitary_norm_562_96}.

При выполнении основной работы на ВДТ и ПЭВМ в помещениях, где работают
инженерно-технические работники, занимающиеся творческой деятельностью, конструированием,
проектированием и программированием уровень шума не должен превышать 50 дБА,
согласно \cite[табл. 2]{ecology_sanitary_norm_562_96}.

Уровень шума используемого современного компьютера находится в пределах от 25 до
40 дБА. В рабочем помещении расположены два рабочих места, оборудованных ПЭВМ.
Суммарный уровень шума составит 50 – 80 дБА, что превышает ПДУ.

Для снижения уровня шума помещениях с ПЭВМ и ВДТ можно использовать звукопоглощающие
материалов для отделки помещений с максимальными коэффициентами звукопоглощения
в области частот 63…80000 Гц. Дополнительным звукопоглощением служат однотонные
занавеси из плотной ткани, гармонирующие с окраской стен и подвешенные в складку
на расстояние 15…20 см от ограждения. Ширина занавеси должна быть в 2 раза больше
ширины окна.

Для достижения допустимых уровней шума, отделка помещения выполнена из
звукопоглощающих материалов.

Источники инфразвука, ультразвука и вибраций отсутствуют.


Тяжесть и напряженность труда характеризуются степенью функционального напряжения
организма. Оно может быть энергетическим, зависящим от мощности работы — при
физическом труде, и эмоциональным — при умственном труде, когда имеет место
информационная перегрузка.

Физическая тяжесть труда — это нагрузка на организм при труде, требующая
преимущественно мышечных усилий и соответствующего энергетического обеспечения.
В таблице (\ref{labor_classes_by_work_process_difficulty_tbl}) представлены классы
условий труда по показателям тяжести трудового процесса в соответствии с
руководством Р 2.2.2006-05 \cite{ecology_man_2_2_2006_05}.

\subfile{src/ecology/complex_tables/labor_classes_by_work_process_difficulty}
\subfile{src/ecology/complex_tables/labor_classes_by_work_process_intensity}

Согласно таблице (\ref{labor_classes_by_work_process_difficulty_tbl}), тяжесть
труда можно отнести к допустимому классу условий труда. Напряженность труда
характеризуется эмоциональной нагрузкой на организм при труде, требующем
преимущественно интенсивной работы мозга по получению и переработке информации.

Работа инженера-разрабочика связанна с творческой деятельностью, анализом большого
объёма разнообразной информации, продолжительной работой за ВТД и ПЭВМ, а также
личной ответственностью за функциональное качество конечной продукции.

В таблице (\ref{labor_classes_by_work_process_intensity_tbl}) представлены классы
условий труда по показателям напряженности трудового процесса, в соответствии с
которыми напряжённость труда можно отнести к вредному классу условий труда 1-й
степени.

Исходя из рассмотренных критериев, можно сделать вывод, что психофизиологические
факторы на рабочем месте инженера-разработчика относятся к оптимальному классу
условий труда.

\paragraph{Освещение}

Освещение в помещении является недостаточным и является наиболее опасным фактором.
Для создания более оптимальных условий труда в разделе \ref{lighting_calculation} произведён расчёт
системы освещения, удовлетворяющей всем нормам труда.
