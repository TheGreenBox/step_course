\newpage

\subsection{Анализ влияния техпроцесса изготовления печатной платы на окружающую среду}

\subsubsection{Сборка}

Основная операция при сборке печатной платы - пайка радиоэлементов (микросхемы, транзисторы, диоды, конденсаторы, резисторы, разъёмы и т.д.).

При выполнении пайки металлов на работающих могут воздействовать вредные и опасные производственные факторы. К вредным производственным факторам относятся
\begin{itemize}
    \item Большая запылённость и загазованность рабочей зоны
    \item Высокая температура жала паяльника
    \item Наличие раскалённых частиц металла
    \item Статическая нагрузка на руки
\end{itemize}

В состав припоя, который используется при пайке, входят такие вредные вещества
как олово, сурьма, свинец, медь. Эти вещества, попадая в организм человека через
легкие, желудочно-кишечный тракт и кожу, могут стать причиной различных заболеваний.

При концентрации паров расплавленного припоя в воздухе может образоваться
свинцовая пыль, которая попадет в легкие работающего, а затем в кровь, вызывая
различные отравления. Олово и медь поступают в организм человека в виде пыли и
паров. Медь и её соли действуют на желудок, вызывая раздражающее действие.
Вдыхание паров олова может привести к заболеванию «литейной лихорадкой» и
инфекционным катарактам верхних дыхательных путей.

Сурьма и её соединения, попадая в организм человека, могут вызвать острое
отравление, раздражение слизистых оболочек верхних дыхательных путей, глаз и кожи.
В результате могут развиваться такие заболевания как конъюнктивит, дерматит.

Негативное влияние на организм может оказать тепловое воздействие со стороны
предварительно нагретых изделий, нагревательных устройств (нарушение терморегуляции,
тепловые удары). Температура плавления припоя, используемая при пайке составляет
$180 - 350 \degree$.

При ручных и полуавтоматических методах пайки имеет место статическая нагрузка на
руки, в результате которой могут возникнуть заболевания нервно-мышечного аппарата
плечевого пояса. Источником вредного воздействия на человека так же является
загрязнение свинцом рабочей поверхности стола и рук монтажника.

К опасным производственным факторам относятся брызги, выбросы раскалённого
металла и шлаков, которые могут привести к ожогам открытых участков тела и
повреждению роговицы глаза.

Неправильная эксплуатация электрооборудования может привести к поражению
электрическим током, поэтому следует строго соблюдать правила защиты от поражения
электрическим током.

\subsubsection{Монтаж}

Требования к санитарному ограничению содержания вредных веществ в воздухе рабочей
зоны определяются ПБ 03-582-03 \cite{ecology_pb_03_582_03}.

Содержание вредных веществ в воздухе рабочей зоны не должно превышать их
предельно допустимых концентраций. Предельно допустимая концентрация вредных
веществ в воздухе рабочей зоны - обязательные санитарные нормативы для
использования при проектировании производственных зданий, технологических
процессов, оборудования и вентиляции, а также для предупредительного и
текущего санитарного надзора.

Для вредных веществ в воздухе рабочей зоны должна устанавливаться предельно
допустимая концентрация на основании данных медико-биологических исследований.

В соответствии с устанавливаемой предельно допустимой концентрацией вредных
веществ должны разрабатываться методы их контроля в воздухе рабочей зоны.
Контроль за содержанием вредных веществ в воздухе рабочей зоны должен проводиться
в соответствии с требованиями ГОСТ 12.1.005-88 \cite{ecology_gost_005_88}.

Предельно допустимой называется такая концентрация вредных веществ, которая в
течение всего стажа работы не вызовет у монтажников отклонений в состоянии
здоровья, которые могут быть обнаружены современными средствами (включая
отдалённые последствия). Предельно допустимые концентрации этих веществ в воздухе
рабочей зоны являются максимальными, и их превышение недопустимо. Предельно
допустимые концентрации по ГОСТ 12.1.005-88 \cite{ecology_gost_005_88}[табл. 6]
указанны в таблице (\ref{assembly_pdk}).

\begin{table}[ht]
    \centering
    \begin{tabular}{l|r|l|l}
        Вещество    & ПДК, $\text{мг}/\text{м}^3$   & Класс опасности   & Агрегатное состояние  \\ \hline
        Олово       & 0,3                           & Высокая           & Пары                  \\
        Ацетон      & 200                           & Низкая            & Пары                  \\
        Бензол      & 5                             & Высокая           & Пары                  \\
        Метиловый спирт & 5                         & Средняя           & Пары                  \\
        Сурьма      & 0,3                           & Высокая           & Пары и аэрозоль       \\
        Свинец      & 0,01                          & Очень высокая     & Пары                  \\
        Медь        & 1                             & Высокая           & Пары                  \\ \hline
    \end{tabular}
    \caption{Предельно допустимые концетрации используемых веществ}
    \label{assembly_pdk}
\end{table}

При выборе припоев и флюсов необходимо учитывать их класс опасности, руководствоваться
при работе требованиями СП 952-72 \cite{ecology_san_norm_925_72} и ОСТ 4ГО.033.200
\cite{ecology_ost_033_200}. Применение припоев, в которых содержатся свинец и
кадмий, следует резко ограничивать. Содержание кадмия в припоях не должно превышать 20\%.

Для устранения влияния вредных веществ на организм человека предусматриваются
следующие мероприятия:

\begin{enumerate}
    \item Операции пайки осуществляются в одном изолированном месте или отдельном помещении
    \item Рабочие места, где производится пайка методом лужения, волной или
        электропаяльником, должны быть оборудованы местными вытяжными устройствами,
        обеспечивающими скорость движения воздуха непосредственно на месте пайки
        не менее $0,6 \text{ м}/\text{c}$. Применение рециркуляции воздуха в
        помещении не допускается
    \item Помещение в котором размещаются участки пайки, необходимо обеспечить
        приточными воздухом, подаваемым равномерно в рабочую зону в количестве,
        составляющем 90\% объёма вытяжки. Недостающие 10\% приточного воздуха следует
        подавать в смежные, более чистые помещения. Подвижность воздуха в рабочей
        зоне не должна превышать более $0,3 \text{ м}/\text{c}$.
    \item Рабочие поверхности столов, ящиков для хранения инструментов, используемых
        на рабочих местах, в конце смены следует очищать и мыть горячим мыльным раствором
    \item Необходимо вести строгий контроль за состоянием воздуха в помещении -
        производить замеры и анализ воздуха в течение всей работы, чтобы не
        допускать более высокой концентрации в воздухе рабочей зоны вредных веществ
\end{enumerate}

Согласно требованиям к организации воздухообмена в производственных помещения
по СНиП 2.04.05-91 \cite{ecology_snip_04_05_95}, воздухообмен должен составлять
$1 - 2$ объема помещения в час. В данном помещении организован воздухообмен
$$
    Q = 150 \text{ м}^3/\text{ч}
$$
В процессе монтажа печатной платы, исходя из максимального количества 100 паек
в час, вредных веществ выделяется
$$
    U = 0.045 \text{ мг}/\text{ч}
$$

Таким образом, количество вредных выбросов в окружающую среду при монтаже
печатной платы составляет
$$
    \frac{U}{Q} = \frac{0,045}{150} = 3 \cdot 10^{-4} \text{ мг}/\text{м}^3
$$

Эта величина не превышает предельно допустимую концертрацию загрязняющих веществ
в атмосферном воздухе населенных мест, согласено ГН 2.1.6.1338-03 \cite{ecology_hygiene_norm_1338_03}.

Таким образом, данная технологическая операция является безопасной для окружающей среды.
