\begin{table}[ht]
    \centering
    \begin{tabu}{|X[l,m]|X[-1,c,m]|X[-1,c,m]|X[-1,c,m]|X[-1,c,m]|} \hline
        \multirow{4}{*}{\parbox{\linewidth}{\centering Показатели тяжести трудового процесса}}
        & \multicolumn{4}{c|}{Классы условий труда}                                                         \\ \cline{2-5}

        & \multirow{3}{*}{Оптимальный}
        & \multirow{3}{*}{Допустимый}
        & \multicolumn{2}{c|}{Вредный}                                                                      \\
        &                                                   &               & \multicolumn{2}{c|}{}         \\ \cline{4-5}
        &                                                   &               & 1-й степени & 2-й степени     \\ \cline{2-5}

                                                            & 1             & 2             & 3.1   & 3.2   \\ \hline

        Физическая динамическая нагрузка                    & \textbullet   &               &       &       \\ \hline
        Масса поднимаемого и перемещаемого груза вручную    & \textbullet   &               &       &       \\ \hline
        Стереотипные рабочие движения                       &               & \textbullet   &       &       \\ \hline
        Статическая нагрузка                                & \textbullet   &               &       &       \\ \hline
        Рабочая поза                                        &               & \textbullet   &       &       \\ \hline
        Наклоны корпуса                                     & \textbullet   &               &       &       \\ \hline
        Перемещения в пространстве                          & \textbullet   &               &       &       \\ \hline
    \end{tabu}
    \caption{Классы условий труда по показателям тяжести трудового процесса}
    \label{labor_classes_by_work_process_difficulty_tbl}
\end{table}
