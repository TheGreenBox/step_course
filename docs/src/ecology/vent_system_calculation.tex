\newpage

\subsection{Расчёт системы вентиляции и местного вытяжного устройств}

Процесс разработки печатной платы сопровождается необходимостью регулярно
осуществлять пайку различной сложности и продолжительности.
Пары припоя и флюса, образующиеся при пайке, оказывают вредное воздействие на
организм человека и на окружающую среду. В состав используемого припоя ПОС-61
входит свинец, по физиологическому воздействию относящийся к группе соматических
ядов, вызывающих нарушения деятельности организма, его отдельных органов и систем.
По классификации из ГОСТ 12.1.005-88 \cite{ecology_gost_005_88}, свинец относится
к классу чрезвычайно опасных веществ с предельно допустимой концентрацией (ПДК)
в воздухе $q_\text{пдк}$ = 0,003 $\text{мг/м}^3$.
Свинец и его соединения вызывают изменения в сердечно-сосудистой и нервной системах,
снижают иммунобиологическую активность человека, а также нарушения ферментативных
реакций, витаминного обмена. Наиболее частыми формами отравления свинцом являются
малокровие, плеврит, свинцовые колики и гепатит.

Так как концентрация вредных паров, выделяющихся в процессе пайки, значительно
превышает ПДК (примерно в $2 - 4$ раза), необходимо применять местную вытяжную вентиляцию.
У каждого рабочего места, где выполняется связанная с пайкой печатной платы часть
технологического процесса, устанавливается всасывающая панель.

Всасывающая панель – это приспособление, применяемое в качестве местного отсоса
при таких ручных операциях, как электросварка, газовая сварка, резка металла,
пайка и т.п. При этом «зеркало» всасывания расположено наклонно к рабочему месту,
что не позволяет попадать вредным веществам в зону дыхания рабочего.

Объём воздуха, удаляемый панелью, находим по формуле (\ref{air_volume_to_vent}):

\begin{equation}
\label{air_volume_to_vent}
    L = F_\text{п} \cdot V_\text{п}
\end{equation}

где $F_\text{п}$ - площадь рабочего проёма, через который засасывается воздух, $\text{м}^2$

$V_\text{п}$ - скорость воздуха в рабочем проёме панели, $\text{м/с}$

Скорость воздуха во всасывающем факеле панели при удалении вредных испарений с
ПДК $q < 1$ $\text{мг/м}^3$ \cite[табл. 1.1]{local_vent_spot_calc_method} принимается
$2 - 3,5 \text{ м/c}$, примем $V_\text{п} = 3 \text{ м/c}$.

Необходимый воздухообмен найдём по формуле (\ref{required_air_exchange}):

\begin{equation}
\label{required_air_exchange}
    L = \frac{G}{q_\text{выт} - q_\text{пр}}
\end{equation}

где G - интенсивность выделения вредных испарений, $\text{мг/час}$
$q_\text{выт}$ – концентрация вредных испарений в удаляемом воздухе, $\text{мг/м}^3$
$q_\text{пр}$ – концентрация вредных испарений в приточном воздухе, $\text{мг/м}^3$

Согласно \cite[п. 2.15]{ecology_san_norm_245_71}:
\begin{equation}
\label{dangeroud_vapor_concentrations}
    \begin{array}{lcr}
        q_\text{пр}  & \leq & 0,3 \cdot q_\text{пдк} = 9 \cdot 10^{-4} \text{ мг/м}^3 \\
        q_\text{выт} & \leq &           q_\text{пдк} = 3 \cdot 10^{-3} \text{ мг/м}^3
    \end{array}
\end{equation}

Для расчётов примем выделение свинца при пайке $G = 18 \text{ мг/час}$,
тогда для максимальных значений концертраций вредных испарений из
(\ref{dangeroud_vapor_concentrations}), необходимый по формуле (\ref{required_air_exchange})
воздухообмен на рабочем месте:

$$
    L = 8570 \text{ мг/м}^3
$$

Тогда из формулы (\ref{air_volume_to_vent}) получим площадь рабочего проёма:

\begin{equation}
\label{working_window_area}
    F_\text{п} = \frac{L}{V_\text{п}} = 0,8 \text{ м}^2
\end{equation}

Найдем длины сторон А и B рабочего проёма исходя из того, что $F_\text{п}  = A \cdot B$ и приняв

\begin{equation}
\label{working_window_area_sides_ratio}
    \frac{A}{B} = \frac{3}{4}
\end{equation}

Тогда получим
$$
    A = 775 \text{ мм}
$$
$$
    B = 1033 \text{ мм}
$$

% TODO: вставить рисунок панели равномерного всасывания

Площадь проходного (живого) сечения таких панелей рекомендуется брать в 4.4
раза меньше габаритной площади, т.е.

\begin{equation}
\label{alive_section_are}
    f_\text{ж} = \frac{F_\text{п}}{4,4} = 0,18 \text{ м}^2
\end{equation}

Общий объём вытяжки будет равен

\begin{equation}
\label{overall_sucktion_volume}
    L_{\sum} = n \cdot f_\text{ж} \cdot V_\text{в}
\end{equation}

где $n$ - число рабочих мест,
$n$ - число рабочих мест,
$V_\text{в}$ - скорость воздуха, $\text{м/с}$

Приняв скорость воздуха в живом сечении равной 3 $\text{м/с}$, для $n = 2$-х
рабочих мест получим

$$
    L_{\sum} = 1,08 \text{ м}^3 / \text{с}
$$

Расчётная производительность вентилятора с учётом потерь и подсоса воздуха
в воздуховодах

\begin{equation}
\label{fan_productivity}
    L_s = 1,1 \cdot L_{\sum} \approx 1,2 \text{ м}^3 / \text{с} = 4,32 \cdot 10^3 \text{ м}^3 / \text{ч}
\end{equation}

% TODO: вставить рисунок расчётной схемы вентиляции
