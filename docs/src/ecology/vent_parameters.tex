\begin{enumerate}
    \item   Расход воздуха \\
            \begin{tabular}{lllll}
                уч-ки 1 - 6:    & $L_k $    & = $f_\text{ж} \cdot V_\text{п}$   & = 1944    & $\text{м}^3 / \text{ч}$ \\
                уч-ки 7 - 9:    & $L_{7-9}$ & = $2 L_k$                         & = 3888    & $\text{м}^3 / \text{ч}$ \\
                уч-к 10:        & $L_{10}$  & = $L_{7} + L_{8}$                 & = 7776    & $\text{м}^3 / \text{ч}$ \\
                уч-к 11:        & $L_{11}$  & = $L_{9} + L_{10}$                & = 11664   & $\text{м}^3 / \text{ч}$ \\
            \end{tabular}

    \item   Длины участков \\
            \begin{tabular}{llll}
                уч-ки 1 - 6:        & $l$           &           & = 1 м \\
                уч-ки 7 и 9:        & $l_{7,9}$     & = $4 l$   & = 4 м \\
                уч-ки 8, 10 и 11:   & $l_{8,10,11}$ & = $2 l$   & = 2 м \\
            \end{tabular}

    \item   Площади и диаметры участков \\
            Воспользуемся формулой, производной от формулы (\ref{air_volume_to_vent})
            для вычисления площади $F_i$ и следующей формулой для вычисления диаметра:
            \begin{equation}
            \label{vent_section_d}
                d_i = \frac{4}{\pi} \sqrt{F_i}
            \end{equation}
            Скорость движения воздуха $V_i$ будем принимать тем большей, чем больше
            расход воздуха в секции.
            \begin{tabular}{lllllllll}
                уч-ки 1 - 6:    & $F_{1-6}$ & = 0,108   & $\text{м}^2$; & $d_{1-6}$ & = 0,37 м; & $V_{1-6}$ & = 5   & $\text{ м/с}$     \\
                уч-к 7-9:       & $F_{7-9}$ & = 0,166   & $\text{м}^2$; & $d_{7-9}$ & = 0,46 м; & $V_{7-9}$   & = 6,5 & $\text{ м/с}$   \\
                уч-к 10:        & $F_{10}$  & = 0,2     & $\text{м}^2$; & $d_{10}$  & = 0,51 м; & $V_{10}$  & = 8   & $\text{ м/с}$     \\
                уч-к 11:        & $F_{11}$  & = 0,25    & $\text{м}^2$; & $d_{11}$  & = 0,56 м; & $V_{11}$  & = 14  & $\text{ м/с}$     \\
            \end{tabular}

    \item   Потери давления на каждом участке
        \begin{itemize}
            \item   \textbf{Участки 1 - 6} \\
                    Определим потери давления на трение. Потеря давления на
                    единицу длины $R$ по номограмме потери давления в воздуховодах равна
                    $$
                        R_{1-6} = 0,8 \text{ Па/м}
                    $$
                    Потеря давления на трение на всём участке
                    $$
                        R_{1-6} \cdot l_{1-6} = 0,8 \text{ Па}
                    $$

                    Местные сопротивления создают воздухоприёмная панель,
                    дроссель-клапан и колено круглое. Определим их коэффициенты
                    сопротивления:

                    - для неподвижной жалюзийной решетки по
                    \cite{air_ventilation_and_conditioning}[табл. 22.22]
                    $$
                        \xi_\text{ж} = 2
                    $$

                    - для дросселя-клапана при угле открытия $\phi = 15 \degree$ по
                    \cite{air_ventilation_and_conditioning}[табл. 22.33]
                    $$
                        \xi_\text{дк} = 0,9
                    $$

                    - для круглого колена при угле поворота потока в $90 \degree$
                    и приняв соотношение диаметра отсека к радиусу колена $\frac{d}{r} = 2$
                    по \cite{air_ventilation_and_conditioning}[табл. 22.26]
                    $$
                        \xi_\text{к} = 0,131 + 0,16 \cdot \left( \frac{d}{r} \right)^{3,5} = 0,15
                    $$

                    Тогда суммарные потери давления:
                    $$
                        P_{1-6} = R_{1-6} \cdot l_{1-6} + \xi_\text{ж} \frac{V_\text{п}^2 \cdot 1,18}{2}
                                    + (\xi_\text{к} + \xi_\text{дк}) \cdot \frac{V_{1-6}^2 \cdot 1,18}{2}
                                = 26,9 \text{ Па}
                    $$

            \item   \textbf{Участки 7 и 9}
                    $$
                        R_{7,9} = 1 \text{ Па/м}
                    $$
                    $$
                        R_{7,9} \cdot l_{7,9} = 4 \text{ Па}
                    $$

                    Местные сопротивления создают вытяжной тройник и круглое колено.
                    Геометрия тройника определяется площадью трёх сечений его воздуховодов:
                    $F_\text{вх1} = F_{1-6}$ - первое входное сечение,
                    $F_\text{вх2} = F_{1-6}$ - второй входное сечение,
                    $F_\text{вых} = F_{7,9}$ - выходное сечение.

                    Для случая $F_\text{вх1} = F_\text{вх2} = F_\text{вх}$,
                    $F_\text{вх} / F_\text{вых} \simeq 0,6$ и угла между воздуховодами
                    в $90 \degree$ коэффициент сопротивления
                    тройника равен $\xi_\text{тр} = 2$ по
                    \cite{air_ventilation_and_conditioning}[табл. 22.28].

                    Тогда суммарные потери давления:
                    $$
                        P_{1-6} = R_{7,9} \cdot l_{7,9}
                                    + (\xi_\text{тр} + \xi_\text{к}) \cdot \frac{V_{7-9}^2 \cdot 1,18}{2}
                                = 57,9 \text{ Па}
                    $$

            \item   \textbf{Участок 8}
                    $$
                        R_{8} = 1 \text{ Па/м}
                    $$
                    $$
                        R_{8} \cdot l_{8,10,11} = 2 \text{ Па}
                    $$

                    Местные сопротивления создаёт только вытяжной тройник.

                    Тогда суммарные потери давления:
                    $$
                        P_{8} = R_{8} \cdot l_{8,10,11}
                                    + \xi_\text{тр} \cdot \frac{V_{7-9}^2 \cdot 1,18}{2}
                                = 51,86 \text{ Па}
                    $$

            \item   \textbf{Участок 10}
                    $$
                        R_{10} = 1,2 \text{ Па/м}
                    $$
                    $$
                        R_{10} \cdot l_{8,10,11} = 2,4 \text{ Па}
                    $$

                    Местные сопротивления создаёт вытяжной тройник с коэффициентом
                    сопротивления $\xi_\text{тр.больш} = 2,4$.

                    Тогда суммарные потери давления:
                    $$
                        P_{10} = R_{10} \cdot l_{8,10,11}
                                    + \xi_\text{тр.больш} \cdot \frac{V_{10}^2 \cdot 1,18}{2}
                                = 93 \text{ Па}
                    $$
            \item   \textbf{Участок 11}
                    $$
                        R_{11} = 1,4 \text{ Па/м}
                    $$
                    $$
                        R_{11} \cdot l_{8,10,11} = 2,8 \text{ Па}
                    $$

                    Тогда суммарные потери давления:
                    $$
                        P_{11} = R_{11} \cdot l_{8,10,11}
                                    + \xi_\text{тр.больш} \cdot \frac{V_{11}^2 \cdot 1,18}{2}
                                = 280,3 \text{ Па}
                    $$
        \end{itemize}

        \item   Полная потеря давления без учета потерь \\ давления в месте
                подсоединения вентилятора
                $$
                    P_\text{полн} = 6 \cdot P_{1-6} + 2 \cdot P_{7,9} + P_{8} + P_{10} + P_{11}
                                = 702,36 \text{ Па}
                $$

                Так как транспортируемый воздух загрязнен, то необходимый напор
                определяется по следующей формуле:
                $$
                    P_s = 1,1 \cdot P_\text{полн} = 772,56 \text{ Па}
                $$
\end{enumerate}

\begin{table}[ht]
    \centering
    \begin{tabular}{l|r|r|r|r|r|r|r}
        № уч-ка & $L_i, \text{ м}^3/\text{ч}$  & $\textit{l}_i$, мм & $F_i, \text{ м}^2$
        & $d_i$, мм & $V_i, \text{ м/c}$ & $R_i, \text{ Па/м}$ & $P_i$, Па              \\ \hline
        1   & 1944  & 1 & 0,108 & 0,37  & 5     & 0,8   & 26,9                          \\
        2   & 1944  & 1 & 0,108 & 0,37  & 5     & 0,8   & 26,9                          \\
        3   & 1944  & 1 & 0,108 & 0,37  & 5     & 0,8   & 26,9                          \\
        4   & 1944  & 1 & 0,108 & 0,37  & 5     & 0,8   & 26,9                          \\
        5   & 1944  & 1 & 0,108 & 0,37  & 5     & 0,8   & 26,9                          \\
        6   & 1944  & 1 & 0,108 & 0,37  & 5     & 0,8   & 26,9                          \\
        7   & 3888  & 4 & 0,166 & 0,46  & 6,5   & 1     & 57,9                          \\
        8   & 3888  & 2 & 0,166 & 0,46  & 6,5   & 1     & 51,86                         \\
        9   & 3888  & 4 & 0,166 & 0,46  & 6,5   & 1     & 57,9                          \\
        10  & 7776  & 2 & 0,2   & 0,51  & 8     & 1,2   & 93                            \\
        11  & 11664 & 2 & 0,25  & 0,56  & 14    & 1,4   & 280,3                         \\

    \end{tabular}
    \caption{Основные показатели при расчёте потерь давления в вентиляции}
    \label{pressure_drop_calc_parameters}
\end{table}
