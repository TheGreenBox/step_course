\bibitem{ecology_man_2_2_2006_05}   Руководство Р 2.2.2006-05.
                                    Руководство по гигиенической оценке факторов
                                    рабочей среды и трудового процесса. Критерии
                                    и классификация условий труда.

\bibitem{ecology_sanpin_1340_03}    СанПиН 2.2.2/2.4.1340-03.
                                    Гигиенические требования к персональным
                                    электронно-вычислительным машинам и
                                    организации работы.

\bibitem{ecology_sanpin_548_96}     СанПиН 2.2.4.548-96.
                                    Гигиенические требования к микроклимату
                                    производственных помещений.

\bibitem{ecology_sanitary_norm_562_96}  Санитарные нормы 2.2.4./2.1.8.562-96.
                                        Шум на рабочих местах, в помещениях
                                        жилых и общественных зданий и
                                        территории жилой застройки.

\bibitem{ecology_gost_60950_2002}   ГОСТ Р МЭК 60950-2002.
                                    Безопасность оборудования информационных
                                    технологий.

\bibitem{ecology_gost_25124_82} ГОСТ 25124-82.
                                Машины вычислительные и системы обработки данных.

\bibitem{ecology_gost_25861_83} ГОСТ 25861-83.
                                Требования электрической и механической
                                безопасности и методы испытаний.

\bibitem{ecology_gost_12_1_038_82}  ГОСТ 12.1.038-82.
                                    Система стандартов безопасности труда.
                                    Электробезопасность. Предельно допустимые
                                    значения напряжений прикосновения и токов.

\bibitem{ecology_gost_12_1_004_91}  ГОСТ 12.1.004-91.
                                    Пожарная безопасность.

\bibitem{ecology_hygiene_norm_1338_03}    Гигиенические нормативы 2.1.6.1338-03.
                                                Предельно допустимые концентрации загрязняющих
                                                веществ в атмосферном воздухе населенных мест.
