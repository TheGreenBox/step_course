\bibitem{ecology_man_2_2_2006_05}   Руководство Р 2.2.2006-05.
                                    Руководство по гигиенической оценке факторов
                                    рабочей среды и трудового процесса. Критерии
                                    и классификация условий труда.

\bibitem{ecology_sanpin_1340_03}    СанПиН 2.2.2/2.4.1340-03.
                                    Гигиенические требования к персональным
                                    электронно-вычислительным машинам и
                                    организации работы.

\bibitem{ecology_sanpin_548_96}     СанПиН 2.2.4.548-96.
                                    Гигиенические требования к микроклимату
                                    производственных помещений.

\bibitem{ecology_sanpin_055_96}     СанПиН 2.2.4/2.1.8.055-96.
                                    Электромагнитные излучения радиочастотного
                                    диапазона.

\bibitem{ecology_sanpin_723_98}     СанПиН 2.2.4.723-98.
                                    Переменные магнитные поля промышленной
                                    частоты (50 Гц) в производственных условиях.

\bibitem{ecology_snip_23_05_95}     СНиП 23.05-95.
                                    Естественное и искусственное освещение.

\bibitem{ecology_snip_04_05_95}     СНиП 2.04.05-91.
                                    Отопление, вентиляция и кондиционирование.

\bibitem{ecology_snip_02_85}    СНиП 2.09.02-85.
                                Производственные здания.

\bibitem{ecology_snip_02_84}    СНиП 2.04.02-84.
                                Водоснабжение.
                                Наружные сети и сооружения.

\bibitem{ecology_sanitary_norm_562_96}  Санитарные нормы 2.2.4./2.1.8.562-96.
                                        Шум на рабочих местах, в помещениях
                                        жилых и общественных зданий и
                                        территории жилой застройки.

\bibitem{ecology_gost_60950_2002}   ГОСТ Р МЭК 60950-2002.
                                    Безопасность оборудования информационных
                                    технологий.

\bibitem{ecology_gost_50377_92} ГОСТ Р 50377-92.
                                Безопасность оборудования информационной
                                технологии, включая электрическое конторское
                                оборудование.

\bibitem{ecology_gost_51251_99} ГОСТ Р 51251-99.
                                Фильтры очистки воздуха. Классификация.
                                Маркировка.

\bibitem{ecology_gost_25124_82} ГОСТ 25124-82.
                                Машины вычислительные и системы обработки данных.

\bibitem{ecology_gost_25861_83} ГОСТ 25861-83.
                                Требования электрической и механической
                                безопасности и методы испытаний.

\bibitem{ecology_gost_003_74}   ГОСТ 12.0.003-74.
                                Опасные и вредные производственные факторы.
                                Классификация.

\bibitem{ecology_gost_002_84}   ГОСТ 12.1.002-84.
                                Система стандартов безопасности труда.
                                Электрические поля промышленной частоты.
                                Допустимые уровни напряженности и требования
                                к проведению контроля на рабочих местах.

\bibitem{ecology_gost_004_91}   ГОСТ 12.1.004-91.
                                Пожарная безопасность.

\bibitem{ecology_gost_005_88}   ГОСТ 12.1.005-88.
                                Общие санитарно-гигиенические требования
                                к воздуху рабочей зоны.

\bibitem{ecology_gost_006_84}   ГОСТ 12.1.006-84.
                                Система стандартов безопасности труда.
                                Электромагнитные поля радиочастот.
                                Допустимые уровни на рабочих местах и требования
                                к проведению контроля.

\bibitem{ecology_gost_038_82}   ГОСТ 12.1.038-82.
                                Система стандартов безопасности труда.
                                Электробезопасность. Предельно допустимые
                                значения напряжений прикосновения и токов.

\bibitem{ecology_gost_045_84}   ГОСТ 12.1.045-84.
                                Электростатические поля. Допустимые уровни на
                                рабочих местах и требования к проведению контроля.

\bibitem{ecology_gost_007_01_75}    ГОСТ 12.2.007.01-75.
                                    Система стандартов безопасности труда.
                                    Изделия электротехнические.
                                    Общие требования безопасности.

\bibitem{gost_21_602_2003}  ГОСТ 21.602-2003.
                            Правила выполнения рабочей документации
                            отопления, вентиляции и кондиционирования.

\bibitem{ecology_san_norm_245_71}   Санитарные нормы 245-71.
                                    Санитарные нормы проектирования промышленных
                                    предприятий.

\bibitem{ecology_hygiene_norm_1338_03}  Гигиенические нормативы 2.1.6.1338-03.
                                        Предельно допустимые концентрации загрязняющих
                                        веществ в атмосферном воздухе населенных мест.

\bibitem{ecology_san_norm_925_72}   Санитарные правила 952-72.
                                    Санитарные правила организации процессов пайки
                                    мелких изделий сплавами, содержащими свинец.

\bibitem{ecology_ost_033_200}   Отраслевой стандарт 4ГО.033.200.

\bibitem{ecology_pb_03_582_03}  Правила безопасности 03-582-03.
                                Правила устройства и безопасной эксплуатации
                                компрессорных установок с поршневыми компрессорами,
                                работающими на взрывоопасных и вредных газах.
