\newpage
\section{Промышленная экология и безопасность}

\subsection{Анализ условий труда на рабочем месте инженера-разработчика}

\subsubsection{Введение}

В рамках данного дипломного проекта проводится проектирование системы управления
манипулятора, расположенного на борту спутника, на основе шагового привода.
Основным рабочим местом является стол, оборудованный персональной ЭВМ
(далее - ПЭВМ) с визуально-дисплейным терминалом (далее - ВДТ) и макетом
манипулятора, который состоит из платы управления, экспериментального стенда с
имитацией нагрузочных моментов, внешнего источника питания и шагового двигателя.

% TODO: сделать и вставить схему рабочего места

Помимо специфических условий зрительной работы, напряженного
нервно-эмоционального характера труда, вынужденной рабочей позы, недостатка
подвижности и физической активности, работающие за ВДТ подвергаются воздействию
низкоэнергетического УФ и рентгеновского излучений, шума, электромагнитных и
электростатических полей, неудовлетворительного микроклимата, вентиляции и
освещения.

\subsubsection{Требования к помещениям для эксплуатации ПЭВМ и ВДТ}

Согласно СанПиН 2.2.2/2.4.1340-03 \cite{ecology_sanpin_1340_03}, помещения с ВДТ
и ПЭВМ должны иметь естественное и искусственное освещение. Рабочее место по
отношению к световым проёмам должно располагаться так, чтобы естественный свет
падал сбоку, преимущественно слева. Необходимо обеспечивать коэффициент
естественной освещенности (КЕО) не ниже $1,2 \%$ в зонах с устойчивым снежным
покровом и не ниже $1,5 \%$ на остальной территории.

Расположение рабочих мест с ВДТ и ПЭВМ для пользователей в подвальных помещениях не
допускается. В случаях производственной необходимости, эксплуатация ВДТ и ПЭВМ в
помещениях без естественного освещения может проводиться только по согласованию
с органами и учреждениями Государственного санитарно-эпидемиологического надзора.

Производственные помещения, где для работы используются преимущественно ВДТ и
ПЭВМ, не должны граничить с помещениями, в которых уровни шума и вибрации превышают
нормируемые значения. Звукоизоляция ограждающих конструкций помещений с ВДТ и ПЭВМ
должна отвечать гигиеническим требованиям и обеспечивать нормируемые параметры шума
согласно требованиям санитарных правил. Помещения с ВДТ и ПЭВМ должны оборудоваться
системами отопления, кондиционирования воздуха или эффективной приточно-вытяжной
вентиляцией.

Для внутренней отделки интерьера помещений с ВДТ и ПЭВМ должны использоваться
диффузно отражающие материалы с коэффициентом отражения $0,7 - 0,8$ для потолка,
$0,5 - 0,6$ для стен, $0,3 - 0,5$ для пола. Полимерные материалы, используемые для
внутренней отделки интерьера помещений с ВДТ и ПЭВМ, должны быть разрешены для
применения органами и учреждениями Государственного санитарно-эпидемиологического
надзора.

Площадь на одно рабочее место с ВДТ и ПЭВМ должна составлять не менее 6-ти $\text{м}^2$,
а объем – не менее 20-ти $\text{м}^3$. Расстояние между тыльной частью одного
видеомонитора и экраном другого видеомонитора должно быть не менее 2-х метров, а
расстояние между их боковыми поверхностями - не менее 1,2-х м.

Конструкция рабочего стола (кресла) должна обеспечивать поддержание рациональной
рабочей позы, позволяя изменить позу с целью снижения статического напряжения мышц
шейно-плечевой области и спины для предупреждения развития утомления. Рабочий стул
должен быть подъемно-поворотным и регулируемым по высоте. Рабочее место организовано
следующим образом. Высота над уровнем пола рабочей поверхности, за которой работает
инженер-программист, составляет 72-х см. Размеры поверхности стола
2 м $\cdot$ 1 м. Под столом должно быть предстусмотрено пространство
для ног с глубиной в 65 см. Расстояние между глазами оператора и
экраном видеодисплея должно
составлять $40 - 80$ см. Рабочий стул разработчика снабжен подъемно-поворотным
механизмом. Высота сиденья должна регулироваться в пределах $40 - 50$ cм. Глубина
сиденья должна составлять не менее 38-ми cм, а ширина – не менее 40 cм. Высота
опорной поверхности спинки - не менее 30-ти cм, ширина – не менее 38-ми см. Угол наклона
спинки стула к плоскости сиденья должен изменяться в пределах $90 \degree - 110 \degree$.
Экран монитора должен находиться от глаз пользователя на оптимальном
расстоянии $0,6 - 0,7$ м, но не ближе 0,5 метра с учетом размеров алфавитно-цифровых
знаков и символов.

Для снижения отрицательного воздействия излучений от компьютера
на здоровье работника, предусмотрен режим труда и отдыха в зависимости от выполняемой
работы и возраста пользователя. Независимо от вида выполняемой работы, общая
продолжительность работы, примерное время непосредственной работы с компьютером
не должно превышать 6-ти часов. Санитарными нормами предусматриваются регламентированные
перерывы длительностью 15-ти минут с периодичностью в каждые 2 часа при вводе
информации, а при считывании её с экрана дисплея перерыв устанавливается в
каждые $1,5 - 2$ часа или по 10 минут через каждый час.

\subfile{src/ecology/dangerous_factors}


\subsubsection{Итоговая оценка условий труда работника по степени вредности}

Оценка условий труда с учетом комбинированного действия факторов проводится на
основании результатов измерений отдельных факторов в которых учтены эффекты
суммации при комбинированном действии химических веществ, биологических факторов,
различных частотных диапазонов электромагнитных излучений.

Общую оценку устанавливают:

\begin{itemize}
    \item   по наиболее высокому классу и степени вредности
    \item   в случае совместного действия 3-х и более факторов, относящихся к классу
            3.1, общая оценка условий труда соответствует классу 3.2
    \item   при сочетании 2-х и более факторов классов 3.2, 3.3, 3.4 - условия
            труда оцениваются соответственно на одну степень выше
\end{itemize}

\subfile{src/ecology/complex_tables/sum_of_labor_conditions}

\subsubsection{Выводы}

Согласно руководству Р 2.2.2006-05 \cite{ecology_man_2_2_2006_05}, проведен анализ
условий труда на рабочем месте инженера~--~программиста. По таблице
(\ref{sum_of_labor_conditions}) можно видеть, что класс условий труда – 3, «Вредный».
