\newpage
\subsection{Расчёт освещения}

Произведем расчет системы освещения, так как этот физический фактор относится
к вредным и не удовлетворят поставленным требованиям. Возможность нормальной
производственной деятельности в научно~--~исследовательских организациях должна быть
обеспечена оптимально спроектированным и выполненным освещением. Сохранность зрения
работника, состояние его нервной системы, безопасность на производстве зависит от
условий освещения.

При освещении помещений используют \textit{естественное}, \textit{искусственное},
а также \textit{смешанное} освещение.
Естественное освещение подразделяют на \textit{боковое}, \textit{верхнее} и
\textit{комбинированное}.
Искусственное освещение по конструктивному исполнению делится на две системы:
\textit{общее} и \textit{комбинированное}. Применение одного местного освещения
внутри помещения не допускается.

Основная задача освещения – создание наилучших условий для зрения. Для ее
выполнения освещение должно отвечать следующим требованиям:

\begin{enumerate}
    \item   Освещенность на рабочем месте должна соответствовать характеру
            зрительной работы, которая определяется тремя параметрами
        \begin{itemize}
            \item объектом различия
            \item фоном с определенным коэффициентом отражения (0,02 – 0,95)
            \item контрастностью объекта
        \end{itemize}
    Контрастность определяется как:
    \begin{equation}
    \label{lighting_contrast}
        K = \frac{L_0 - L_\text{ф}}{L_\text{ф}}
    \end{equation}

    где $L_0$, $L_\text{ф}$ - яркость объекта и фона соответственно

    Средняя величина контрастности $0,2 - 0,3$. До определенного предела увеличения
    яркости повышает производительность труда.

    \item   Необходимо обеспечивать равномерную яркость на всей рабочей поверхности.
            В поле зрения не должны находиться предметы, резко отличающиеся по яркости.

    \item   По рабочей поверхности не должно быть резких теней и не должно быть
            блесткости, т.е. повышения яркости светящихся поверхностей.

    \item   Величина освещенности должна быть постоянной по времени, что достигается
            стабилизацией напряжения питания и сглаживанием пульсаций тока в
            осветительных приборах.

    \item   Выбирается оптимальный спектральный состав освещения путем комбинации
            естественного и искусственного освящений.

    \item   Все электроосветительные приборы должны быть электро- и травмобезопасны.
\end{enumerate}

Основным видом работ, выполняемых инженером~--~разработчиком, является программирование
и черчение на ПЭВМ, расчетные работы с использованием микрокалькуляторов и вычислительной
техники. Величина минимальной освещенности устанавливается по характеру зрительной работы.

\subsubsection{Искусственное освещение помещения}

Определение необходимой мощности осветительной установки для получения заданной
освещенности требует решения следующих вопросов:

\begin{itemize}
    \item \textit{выбор типа источника света}: применение горизонтальных ламп с
    большей светоотдачей оправдано, т.к. в поле зрения инженера~--~разработчика
    нет быстровращающихся предметов и пульсации светового потока практически не
    заметны

    \item \textit{выбор системы освещения}: выбираем комбинированную систему
    согласно СНиП 23.05-95 \cite{ecology_snip_23_05_85}. Система общего освещения
    создает равномерное распределение света. Для повышения общего освещения
    используют местные светильники, обеспечивающие создание направленного света,
    исключающие отраженную блесткость, а также позволяющие выполнять просвечивание
    материалов и деталей

    \item \textit{выбор типа светильников}: исходя из требований, предъявляемых
    к светильникам, выбираем тип светильника – ЛОУ (люминесцентные светильники
    расположенные в светящую линию), устанавливаемый в помещениях с небольшой
    запыленностью и нормальной влажностью

    \item \textit{распределение светильников в помещении и их количество}:
    равномерность распределения освещенности светильниками ЛОУ достигается в случае,
    если расстояние между центрами светильников больше высоты их расположения над
    рабочей поверхностью $H_p$ в 1,4 раза
\end{itemize}

Прием следующие параметры:

\begin{tabular}{ll}
    - освещенность комбинированная    & 400 лк    \\
    - освещенность общая              & 150 лк    \\
    - коэффициент запаса              & k = 1,3   \\
\end{tabular}

\subsubsection{Общее освещение помещения}

Размещение светильников определяется следующими размерами:

\begin{tabular}{ll}
    - высота помещения                                                      & $H$ = 3,2 м \\
    - расстояние светильников от перекрытия                                 & $h_c$ = 0,25 м \\
    - высота светильников над полом                                         & $h_\text{п} = H - h_c$ = 2,95 м \\
    - высота расчетной поверхности (для помещений, связанных с работой ПЭВМ)& $h_p$ = 0,7 м \\
    - расчетная высота                                                      & $h = h_\text{п} - h_p$ = 2,25 м \\
    - расстояние между соседними светильниками по длине помещения           & $L_\text{д}$ = 3,2 м \\
    - расстояние между соседними светильниками по ширине помещения          & $L_\text{ш}$ = 3 м \\
    - расстояние от крайних светильников до стены                           & $l_\text{д} = 0,3 \cdot L_\text{д}$ = 0,96 м, \\
    & $l_\text{ш} = 0,3 \cdot L_\text{ш}$ = 0,73 м \\
\end{tabular}

Используем метод коэффициента использования светового потока, предназначенный для
расчета общего равномерного освещения горизонтальных поверхностей при отсутствии
крупных затемняющих предметов.

Необходимый поток ламп в каждом светильнике:
\begin{equation}
\label{one_lamp_stream}
    \text{Ф} = \frac{E \cdot S \cdot k_\text{зап} \cdot k_\text{неравн}}{N \cdot k_\text{исп}}
\end{equation}

где $E$ = 300 лк - заданная минимальная освещенность для 3-го разряда зрительных работ,

$k_\text{зап}$ = 1,3 - коэффициент запаса для помещений, связанных с работой на ПЭВМ,

$S = 40 \text{ м}^2$ - освещаемая площадь,

$N$ - намечаемое число светильников,

$k_\text{исп}$ – коэффициент использования,

$k_\text{неравн}$ = 1,1 – коэффициент неравномерности освещения для данных $l_\text{ш}$ и
$l_\text{д}$

Для нахождения коэффициента использования $k_\text{исп}$ по справочным таблицам
выберем индекс помещения $i = 1,5$ для ранее вычисленной расчётной высоты $h$ и
примерно оценим коэффициенты отражения поверхностей помещения:
$$
    r_\text{потолок} = 70\%
$$
$$
    r_\text{стены} = 50\%
$$
$$
    r_\text{пол} = 30\%
$$

Тогда из справочных таблиц для освещаемой площади $S$ получим коэффициент использования
$$
    k_\text{исп} = 0,59
$$

Светильники с люминесцентными лампами в помещениях для работы рекомендуют
устанавливать рядами. Положим $N = 4$ как 2 светильника в два ряда.

Подставив вышеописанные параметры в формулу (\ref{one_lamp_stream}), получим
необходимый световой поток одного светильника

$$
    \text{Ф} \simeq 7200 \text{ лм}
$$

Исходя из этого, используем светильники с люменесцентными лампами 2х40 Вт с
общим потоком 7200 лм.

Параметры светильника типа ЛОУ:

\begin{tabular}{ll}
    - мощность    & 2x40 Вт   \\
    - длина       & 1,24 м    \\
    - ширина      & 0,27 м    \\
    - высота      & 0,10 м    \\
\end{tabular}

% TODO: вставить схему расположения светильников

\subsubsection{Комбинированное освещение помещения}
