\ifdefined\DIPLOMA
    Экспериментальный стенд необходим для валидации построенных математических моделей,
    проверки и отладки созданного управляющего программного обеспечения,
    изучения влияния внешних параметров на качество работы алгоритмов управления,
    изучения резонансных явлений в шаговом двигателе.
\else
    \subsection{Натурное моделирование}
    Разработка стенда для валидации построенных математических моделей, изучения
    влияния внешних параметров на качество работы алгоритмов управления, изучение
    резонансных явлений в шаговом двигателе планируется в ближайшее время.
\fi

Сформулируем требования к разрабатываемому стенду:

\begin{enumerate}
    \item Стенд должен содержать в себе полноценный двигатель из разрабатываемого
    привода для моделирования физических процессов в электрических и магнитных цепях.

    \item Должна присутствовать имитация полной нагрузки, с моментом инерции не менее, чем
    значение, указанное в техническом задании, в том числе для того, что бы оценить
    грузоподъемность разрабатываемого привода, его быстродействие на максимальных
    нагрузках. В том числе, что не менее важно, это позволит с некоторой точностью
    определить положения резонанстных явлений на частотной оси.

    \item В механической связи между двигателем и нагрузкой должна присутствовать имитация
    упругости редуктора привода, так как это может оказывать отрицательное воздействие
    на точность конечного механизма.

    \item В модели должен присутствовать один или более датчиков угла, с точностью не
    грубее чем датчик который планируется установить на привод. В оптимальном случае должно
    стоять два датчика: точный - для экспериментальных целей и грубый - близкий
    к тому, что предполагается поставить на привод манипулятора.

    \item Стенд должен обеспечивать легкий доступ к электрическим цепям двигателя,
    усилителя тока и микропроцессорного блока управления для снятия сигналов
    осцилографом.

    \item Конструкция должна иметь массивное основание~-- такое, чтобы
    исключить вероятность опрокидывания конструкции при реверсивных движениях
    и переброске при максимальном ускорении.

    \item Статическая нагрузка в нормальных условиях работы по техническому
    заданию отстутствует, по этой причине в модели ее учитывать не нужно.
\end{enumerate}
