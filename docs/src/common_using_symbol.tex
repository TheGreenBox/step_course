\newpage
\section{Основные принятые обозначения}

\begin{table}[ht!]
    \begin{tabular}{rll}
    $\theta$                        & \textit{рад}  & текущее абсолютное положение ротора \\

    $\theta_{\textit{шаг}}$         & \textit{рад}  & алгоритмический шаг шагового двигателя \\

    $I_{\textit{ст}}$               & \textit{A}    & установившееся значение тока статора \\

    $L_{p}(\theta)$                 & \textit{Гн}   & индуктивность ротора \\

    $L_{\textit{ст}}(\theta)$       & \textit{Гн}   & индуктивность статора \\

    $L_{\textit{р-ст}}(\theta)$     & \textit{Гн}   & взаимоиндукция ротора и статора \\

    $t_{1}$                         & \textit{с}    & время подачи напряжения внутри импульса \\

    $T_\textit{шим}$                & \textit{с}    & период ШИМ \\

    $\zeta$                         & -             & коэффициент заполнения ШИМ \\

    $T$                             & \textit{с}    & постоянная времения обмотки одной фазы \\

    $t$                             & \textit{с}    & текущее время \\

    $N_{r}$                         & -             & число пар полюсов ротора \\

    $\theta_\textit{ком}$           & -             & угол коммутации \\

    $i_\textit{ред}$                & -             & передаточное отношениие редуктора \\

    $p_\textit{шд}$                 & -             & число фаз двигателя \\

    $J_{oy}$                        & $\textit{кг} \cdot \textit{м}^{2}$        & момент инерции объекта управления \\

    $J$                             & $\textit{кг} \cdot \textit{м}^{2}$        & момент инерции объекта управления относительно оси привода \\

    $t_\textit{п}$                  & \textit{с}                                & время перенацеливания камеры \\

    $\omega_{\beta.p}$              & $\textit{рад} \cdot \textit{с}^{-1}$      & рабочая угловая частота эквивалентного гармонического сигнала \\

    $\omega_{p}$                    & $\textit{рад} \cdot \textit{с}^{-1}$      & рабочая угловая частота вращения ротора \\

    $\dot{\beta}_{max}$             & $\textit{рад} \cdot \textit{с}^{-1}$      & максимальная угловая скорость вращения камеры \\

    $\ddot{\beta}_{max}$            & $\textit{рад} \cdot \textit{с}^{-2}$      & максимальное ускорение вращения камеры \\

    $A_{\textit{экв}}$              & \textit{рад}                              & амплитуда эквивалентного гармонического сигнала \\

    $T_{y.min}$                     & \textit{с}                                & минимальная длина импульса управления \\

    $f_{RC.\textit{проп}}$          & \textit{Гц}                               & ширина полосы пропускания RC-фильтра датчика тока \\

    $\psi_{M}$                      & \textit{Вб}                               & потокосцепление ротора шагового двигателя \\

    $K_{M}$                         & $\textit{А} \cdot \textit{рад} \cdot \textit{Н}^{-1} \cdot \textit{м}^{-1}$ & моментный коэффициент \\

    $D_{\textit{тр.сух}}$           & $\textit{Н} \cdot \textit{м}$             & момент сухого трения, действующего в системе \\

    $D_{\textit{тр.вязк}}$          & $\textit{Н} \cdot \textit{м} \cdot \textit{c} \cdot \textit{рад}_{-1}$ & коэффициент вязкого трения, действующего в системе \\

    $K_{d1}$                        & $\textit{Н} \cdot \textit{м}$             & амплитуда первой гармоники резонанса момента \\

    $K_{d2}$                        & $\textit{Н} \cdot \textit{м}$             & амплитуда второй гармоники резонанса момента \\

    $K_{d4}$                        & $\textit{Н} \cdot \textit{м}$             & амплитуда четвертой гармоники резонанса момента \\

    $\phi_{1}$                      & \textit{рад}                              & фаза первой гармоники резонанса момента \\

    $\phi_{2}$                      & \textit{рад}                              & фаза второй гармоники резонанса момента \\

    $\phi_{4}$                      & \textit{рад}                              & фаза четвертой гармоники резонанса момента \\

    \end{tabular}
    \caption{ Основные принятые обозначения }
\end{table}
