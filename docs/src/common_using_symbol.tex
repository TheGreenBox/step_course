\subsection{ Основные принятые обозначения }

$I_{s}$ - установившееся значение тока статора

$L_{s}(\theta)$ - индуктивность статора

$L_{sr}(\theta)$ - взаимоиндукция ротора и статора

$t_{1}$ - время подачи напряжения внутри импульса, c

$T_\text{ШИМ}$ - период ШИМ, c

$\zeta$ - безразмерный коэффициент заполнения ШИМ

$T$ - постоянная времения обмотки одной фазы

$t$ - текущее время, c

$N_{r}$ - число пар полюсов ротора.

$i_\text{ред}$ - передаточное отношениие редуктора, безразмерная величина

$p_{sm}$ - число фаз двигателя, безразмерная величина

$J_{\text{ОУ}}$ - момент инерции объекта управления

$J$ - момент инерции объекта управления относительно оси привода

$t_\text{п}$ - время перенацеливания камеры

$\omega_{\beta.p}$ - рабочая частота эквивалентного гармонического сигнала,

$\dot{\beta}_{max}$ - максимальная угловая скорость вращения камеры, соответствующая параметрам эквивалентной синусоиды

$\ddot{\beta}_{max}$ - максимальное ускорение вращения камеры, соответствующее параметрам эквивалентной синусоиды

$A_\text{экв}$ - амплитуда эквивалентного гармонического сигнала

$T_{y.min}$ - минимальная длина импульса управления, c

$f_{RC.\text{проп} }$ - полоса пропускания RC-фильтра датчика тока
