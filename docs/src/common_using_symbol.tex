\subsection{ Основные принятые обозначения }

$\theta$ - текущее абсолютное положение ротора, рад

$I_{s}$ - установившееся значение тока статора, А

$L_{r}(\theta)$ - индуктивность статора, Гн

$L_{s}(\theta)$ - индуктивность статора, Гн

$L_{sr}(\theta)$ - взаимоиндукция ротора и статора, Гн

$t_{1}$ - время подачи напряжения внутри импульса, c

$T_\text{ШИМ}$ - период ШИМ, c

$\zeta$ - коэффициент заполнения ШИМ, безразмерная величина

$T$ - постоянная времения обмотки одной фазы, с

$t$ - текущее время, c

$N_{r}$ - число пар полюсов ротора, безразмерная величина

$i_\text{ред}$ - передаточное отношениие редуктора, безразмерная величина

$p_{sm}$ - число фаз двигателя, безразмерная величина

$J_{\text{ОУ}}$ - момент инерции объекта управления, $ \text{кг} \cdot \text{м}^{2} $

$J$ - момент инерции объекта управления относительно оси привода, $ \text{кг} \cdot \text{м}^{2} $

$t_\text{п}$ - время перенацеливания камеры, с

$\omega_{\beta.p}$ - рабочая угловая частота эквивалентного гармонического сигнала,
                     $ \text{рад} \cdot \text{с}^{-1} $

$\dot{\beta}_{max}$ - максимальная угловая скорость вращения камеры, соответствующая
                      параметрам эквивалентной синусоиды, $ \text{рад} \cdot \text{с}^{-1} $

$\ddot{\beta}_{max}$ - максимальное ускорение вращения камеры, соответствующее
                       параметрам эквивалентной синусоиды, $ \text{рад} \cdot \text{с}^{-2} $

$A_\text{экв}$ - амплитуда эквивалентного гармонического сигнала, рад.

$T_{y.min}$ - минимальная длина импульса управления, c

$f_{RC.\text{проп} }$ - ширина полосы пропускания RC-фильтра датчика тока, Гц

$\psi_{M}$ - потокосцепление ротора шагового двигателя, Вб (Вебер)

