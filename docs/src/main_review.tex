\documentclass[a4paper, 11pt]{article}

\usepackage[utf8]{inputenc}

% Подключение и настройка внешнего вида гиперссылок
\usepackage{xcolor}
\usepackage[unicode]{hyperref}
\definecolor{LINKCOLOUR}{rgb}{0.1,0.0,0.9}
\hypersetup{colorlinks,breaklinks,urlcolor=LINKCOLOUR,linkcolor=LINKCOLOUR}

% Выбор внутренней TEX−кодировки
\usepackage[T2A]{fontenc}

% Включение переносов для русского и английского языков
\usepackage[english,russian]{babel}

\usepackage{hyphenat}
\selectlanguage{russian}
\hyphenation{
    уп-рав-ле-ни-е уп-рав-ле-ни-я уп-рав-ле-ни-ем уп-рав-ле-ний
    нес-коль-ко
    об-ос-но-ва-ни-е об-ос-но-ва-ни-я об-ос-но-ва-ни-ий об-ос-но-ва-ни-ем
}

% Начинать первый параграф раздела следует с красной строки
\usepackage{indentfirst}

\usepackage{amssymb}
\usepackage{amsmath}

\usepackage{cmap}

\usepackage{geometry}   % Меняем поля страницы
\geometry{left=2cm}     % левое поле
\geometry{right=1.5cm}  % правое поле
\geometry{top=1.5cm}    % верхнее поле
\geometry{bottom=2.4cm} % нижнее поле

% символ градуса и т.д.
\usepackage{gensymb}

% стилевой пакет, открывающий доступ к большому числу типографских значков
\usepackage{textcomp}

% \newcommand\studentnameheader{Никитина Сергея Владимировича}
% \newcommand\studentname{Никитин С. В. }

\newcommand\studentnameheader{Киндякова Александра Андреевича}
\newcommand\studentname{Киндяков А. А. }

\begin{document}
\pagenumbering{gobble}

\begin{center}
    {\large \textbf{Рецензия на дипломный проект студента \\ группы СМ7-122 \\ \studentnameheader}}
\end{center}

Дипломный проект выполнен на тему <<Мехатронный модуль для высокоточной системы
дистанционного зондирования Земли>> и содержит 258 листов расчетно~-- пояснительной
записки, 17 листов формата А1, 6 листов формата А2, 8 листов формата А3 и 6 листов
формата А4 графической части.

Дипломный проект посвящен проектированию привода малого звена двухстепенного
манипулятора, устанавливаемого на орбитальном спутнике дистанционного зондирования
земной поверхности. В качестве типа используемого двигателя был выбран шаговый
двигатель.

В исследовательской части выбран двигатель, проведен энергетический расчёт,
проанализированы существующие алгоритмы управления шаговым двигателем,
построена математическая модель привода, синтезированы алгоритмы управления
шаговым двигателем по датчику и без него.

В конструкторской части дипломного проекта разработано программное обеспечение,
реализующее алгоритмы управления шаговым приводом, разработаны плата управления
и макетный стенд для натурного моделирования.

В экспериментальной части дипломного проекта проведена проверка параметров
движения алгоритмов управления.

Технологическая часть проекта состоит из разработки технологии сборки платы
управления и технологии сборки макетного стенда.

В организационно~-- экономической части проведено технико~-- экономическое
обоснование выбора основных компонентов проектируемого привода.

В разделе промышленной экологии и безопасности произведён анализ опасных и
вредных факторов при разработке, расчёт освещённости как наиболее опасного фактора,
а так же анализ влияния техпроцесса изготовления печатной платы и расчёт системы
вентиляции производственного помещения.

Достоинствами дипломного проекта является повышение надёжности привода путём
осуществления возможности перехода на бездатчиковый режим управления без
серьёзных перестроений перейти, сразу же в момент отказа цепи датчика положения.
Поддерживается несколько алгоритмов шаговых последовательностей, переключение
между которыми может осуществляться мгновенно. Программное обеспечение так же
обладает модульной архитектурой и легко переносимо на другие аппаратные платформы.

К недостаткам дипломного проекта следует отнести отсутствие эффективной
борьбы с шумами в каналах датчика тока и скорости. Датчик тока построен таким
образом, что на этапах снятия величины тока на сериесном резисторе,
усиления сигнала на операционном усилителе, а также в АЦП неизменно будут
возникать шумы как статического так и переменного характера, которые в работе
не фильтруются -- зашумленные разряды полученной величины сигнала просто отбрасываются.
Это, очевидно, приводит к потере части полезной информации и огрублению показаний.
Стоит отнести вышесказанное в той же мере и к измерению скорости, которая в
спроектированной системе вычисляется как результат неявного дифференцирования
сигнала энкодера. Как результат, используется сильно зашумленный сигнал,
ухудшающий динамические свойства системы.

Дипломный проект выполнен в соответствии с техническим заданием, \studentname
показал, что уверенно владеет достаточными теоретическими знаниями, инструментами
и технологиями программных комплексов, владеет современной элементной базой,
а так же, что способен использовать их на практике.

Считаю, что дипломный проект заслуживает оценки <<отлично>>, а \studentname
заслуживает присвоения квалификации <<инженер>>.

\paragraph{ }
С.н.с отд. СМ4~--~4 НИИСМ

\end{document}
