\documentclass{article}

\usepackage[utf8]{inputenc}

% Подключение гиперссылок
\usepackage{xcolor}
\usepackage[unicode]{hyperref}

% Настройка внешнего вида гиперссылок
\definecolor{LINKCOLOUR}{rgb}{0.1,0.0,0.9}
\hypersetup{colorlinks,breaklinks,urlcolor=LINKCOLOUR,linkcolor=LINKCOLOUR}

% Выбор внутренней TEX−кодировки
\usepackage [T2A]{fontenc}
% Включение переносов для русского и английского языков
\usepackage[english,russian]{babel}

% Начинать первый параграф раздела следует с красной строки
\usepackage{indentfirst}


% Дополнительные математические пакеты
%\usepackage{weird,querr}
\usepackage{amssymb}
\usepackage{amsmath}

% Для корректного копирования из документа
\usepackage{cmap}

% Кто нибудь помнит зачем это тут? A: Неа...
\usepackage{multirow}

% Меняем поля страницы
\usepackage{geometry}
\geometry{left=2cm}    % левое поле
\geometry{right=1.5cm} % правое поле
\geometry{top=2cm}     % верхнее поле
\geometry{bottom=2cm}  % нижнее поле

\usepackage{subfiles}

% работа с импортом изображений
\ifx\pdfoutput\undefined
\usepackage{graphicx}
\else
\usepackage[pdftex]{graphicx}
\fi

% секции и их структура
\usepackage[section]{placeins}
\usepackage{subcaption}

\begin{document}

%\maketitle
\begin{titlepage}
\begin{center}
    {\large Московский Государственный Технический Университет им. Н. Э. Баумана}
    \\[5cm]
    {\Large Экономическая часть}
    \\[5cm]

    \begin{flushright}
        \begin{minipage}{0.5\textwidth}
            \begin{flushleft}
                \textit{Авторы:} \\
                \hbox to 8cm {Киндяков Александр Андреевич \hfil \underline{\hspace{2cm} } }
                \vspace{\baselineskip}
                \hbox to 8cm {Никитин Сергей Владимирович \hfil \underline{\hspace{2cm} } }
                \vspace{\baselineskip}
                \hbox to 8cm {\textit{Группа:} \hfil СМ7-122}
                \vspace{2cm}
                \textit{Руководитель:} \\
                \hbox to 8cm {Бошляков Андрей Анатольевич \hfil \underline{\hspace{2cm} } }
            \end{flushleft}
        \end{minipage}
    \end{flushright}

    \vfill % Заполнить все доступное пространство
    Москва, 2014 г. \\
    \LaTeX
\end{center}
\end{titlepage}

\tableofcontents
\subfile{src/ecology/workplace_analysis}

\newpage
\begin{thebibliography}{30}
    \subfile{src/ecology/normatives}
\end{thebibliography}

\end{document}
