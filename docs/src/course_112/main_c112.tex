\documentclass{article}

\usepackage[utf8]{inputenc}

% Подключение гиперссылок
\usepackage{xcolor}
\usepackage[unicode]{hyperref}

% Настройка внешнего вида гиперссылок
\definecolor{LINKCOLOUR}{rgb}{0.1,0.0,0.9}
\hypersetup{colorlinks,breaklinks,urlcolor=LINKCOLOUR,linkcolor=LINKCOLOUR}

% Выбор внутренней TEX−кодировки
\usepackage [T2A]{fontenc}
% Включение переносов для русского и английского языков
\usepackage[english,russian]{babel}

% Начинать первый параграф раздела следует с красной строки
\usepackage{indentfirst}


% Дополнительные математические пакеты
%\usepackage{weird,querr}
\usepackage{amssymb}
\usepackage{amsmath}

% Для корректного копирования из документа
\usepackage{cmap}

% Кто нибудь помнит зачем это тут? A: Неа...
\usepackage{multirow}

% Меняем поля страницы
\usepackage{geometry}
\geometry{left=2cm}    % левое поле
\geometry{right=1.5cm} % правое поле
\geometry{top=2cm}     % верхнее поле
\geometry{bottom=2cm}  % нижнее поле

\usepackage{subfiles}

% работа с импортом изображений
\ifx\pdfoutput\undefined
\usepackage{graphicx}
\else
\usepackage[pdftex]{graphicx}
\fi

% секции и их структура
\usepackage[section]{placeins}
\usepackage{subcaption}

\begin{document}

%\maketitle
\begin{titlepage}
\begin{center}
    {\large Московский Государственный Технический Университет им. Н. Э. Баумана}
    \\[50mm]
    {\LargeКурсовой проект по теме}
    \\[7mm]
    {\LARGE ``Система управления \\ приводами двухстепенного манипулятора \\ на основе шаговых двигателей''}
    \\[37mm]

    \begin{flushright}
        \begin{minipage}{0.5\textwidth}
            \begin{flushleft}
                \textit{Авторы:} \\
                \hbox to 8cm {Киндяков Александр Андреевич \hfil \underline{\hspace{2cm} } }
                \vspace{\baselineskip}
                \hbox to 8cm {Никитин Сергей Владимирович \hfil \underline{\hspace{2cm} } }
                \vspace{2cm}
                \textit{Руководитель:} \\
                \hbox to 8cm {Бошляков Андрей Анатольевич \hfil \underline{\hspace{2cm} } }
            \end{flushleft}
        \end{minipage}
    \end{flushright}

    \vfill % Заполнить все доступное пространство
    Москва, 2013 г. \\
    \LaTeX
\end{center}
\end{titlepage}

\tableofcontents

\newpage
\subfile{src/common_using_symbol}

\newpage
\section{Постановка задачи}
\subfile{src/product_requirements}

\newpage
\subfile{src/controlled_object_desc}
\subfile{src/drive_parameters}
\subfile{src/research/goals}
\subfile{src/research/theoretical_base}

\newpage
\section{Моделирование}
\subfile{src/research/math_modeling}
\subfile{src/natural_modeling}
\subfile{src/current_feedback_description}

\newpage
\section{Заключение}

В результате проведенной работы, были разработаны алгоритмы переключения фаз при
работе с заданным углом коммутации и алгоритм управления ШД без обратной связи.
Решена задачa определения коэффициета заполнения ШИМ внутри импульса
управления.

Найден способ определения угла коммутации для текущей скорости. Изучена методика
расчета предельной скорости, ниже которой двигатель гарантировано не выйдет из
синхронизма при постоянной динамической нагрузке.

Принято решение использовать обратную связь по току для управления моментом на
выходе двигателя. Выполнен анализ аппаратного решения датчика тока в фазах двигателя
предлагаемого TexasInstruments. Cформированы требования к аппаратной части датчиков
тока, изучена методика исследования свойств аналогово цифрового преобразователя.

Выполнено моделирование на готовой модели из пакета
\textit{SimPower Systems -> Second Generation -> Motors and Generators -> Hybrid Stepper Motor}.
А так же была разработана и отлажена своя матическая моделей в MATLAB Simulink на основе
уравнений электрических процессов в фазах двигателя и математическом аппарате резонансных явлений.

На моделях были опробованы и отлажены разработанные алгоритмы управления.

\subfile{src/bibliography}

\end{document}
