\section{Организационно экономическая часть}
Функционально--стоимостной анализ (ФСА) --- метод системного
анализа функций и свойств различных объектов и затрат на их
реализацию.
Наиболее широко ФСА в настоящее время применяется для технических
объектов--изделий, их частей и деталей, оборудования, технологических процессов
производства.
Основная цель анализа при этом --- выявление резервов снижения затрат на
исследования и разработки, производство и эксплуатацию рассматриваемых объектов,
при условии сохранения соответсвия техническому заданию конечного изделия.
Кроме конструирования и технологии технических объектов в поле деятельности ФСА
в настоящее время включаются организационные и управленческие процессы,
производственные структуры предприятий, объединений и научно--исследовательских
организаций.
Если исходить из общей предпосылки системного анализа, то объектом ФСА может
быть любой элемент сложной производственно--экономической системы, отвечающий
требованиям выделенных выше признаков.


Развитие теории ФСА нашло широкое применение в всех отраслях машиностроительной
промышленности.
Это связано с системностью метода, ставящего своей задачей в каждом конкретном
случае выявить структуру рассматриваемого объекта, разложить его на простейшие
элементы, дать им объективную оценку (со стороны потребительской стоимости
--- интегрального качества и со стороны стоимости затрат на исследования,
производство и эксплуатацию).
В силу своей системности ФСА позволяет выявить в каждом анализируемом объекте
причинно--следственные связи между качеством, эксплуатационно--техническими
характеристиками и стоимостью.
На основе этого создаются основания для исключения механических методов
планирования затрат от достигнутого уровня, установления нормативов на основе
сложившегося уровня трудоёмкости себестоимости и расхода материалов.


Достоинством ФСА является наличие достаточно простых расчётных и графических
методов, позволяющих дать двойственную количественную оценку выявленных
причинно--следственных связей. Это достоинство ставит ФСА в ряд наиболее
эффективных методов анализа не только технических, но и
производственно--экономических систем, структур, методов организации и
планирования, управления производством и научными исследованиями. Однако работы
по ФСА проводятся в отрыве от экономических расчётов на предприятиях и в
объединениях.
Поэтому экономические нормативы действующего производства не охвачены
функциональным подходом, базируются на предметном экономическом анализе,
планировании от достигнутого уровня.
Функционально--стоимостной анализ управленческих систем позволяет выполнить
следующие виды работ:
\begin{itemize}
    \item Определение и проведение общего анализа себестоимости
    бизнес--процессов на предприятии (маркетинг, производство продукции и
    оказание услуг, сбыт, менеджмент качества, техническое и гарантийное
    обслуживание и др.)
    \item Проведение функционального анализа, связанного с установлением и
    обоснованием выполняемых структурными подразделениями предприятий функций с
    целью обеспечения выпуска продукции высокого качества и оказания услуг
    \item Определение и анализ основных, дополнительных и ненужных
    функциональных затрат
    \item Сравнительный анализ альтернативных вариантов снижения затрат в
    производстве, сбыте и управлении за счёт упорядочения функций структурных
    подразделений предприятия
    \item Анализ интегрированного улучшения результатов деятельности
предприятия
\end{itemize}

В настоящее время метод ФСА стал всеобъемлющим инструментом
оценки систем, процессов и концепций.

\subsection{Основные термины и определения}

\paragraph{Функционально--стоимостной анализ (ФСА)} --- метод системного
исследования функций объектов, направленный на обеспечение общественно
необходимых потребительских свойств объектов и минимальных затрат на их
проявление на всех этапах жизненного цикла.

\paragraph{Объект ФСА} --- изделие, технологический процесс, производственная
организационная структура, а также их отдельные элементы, подвергаемые
исследованию в целях выбора оптимального варианта реализации выполняемых ими
основных функций при минимальных затратах.

\paragraph{Функция} --- внешнее проявление свойств объекта в определённых
условиях, сущность объекта.

\paragraph{Конструкция} --- форма проявления функции.

\paragraph{Носитель функции} --- материальный или иной объект либо отдельные
его элементы, а также их совокупность, реализующие функцию.

\paragraph{Классификация функций} --- группировка функций по определённым
признакам.
Различают следующие группы:
\begin{itemize}
    \item По области проявления:
        \begin{itemize}
            \item общеобъектные (внешние)
            \item внутриобъектные (внутренние)
        \end{itemize}
    \item По роли в удовлетворении потребностей:
        \begin{itemize}
            \item среди внешних
                \begin{itemize}
                    \item главные (эксплуатационные)
                    \item второстепенные
                \end{itemize}
            \item среди внутренних
                \begin{itemize}
                    \item основные (рабочие)
                    \item вспомогательные
                \end{itemize}
        \end{itemize}
    \item По степени необходимости:
        \begin{itemize}
            \item необходимые (полезные)
            \item излишние (бесполезные, вредные)
        \end{itemize}
    \item По характеру проявления:
        \begin{itemize}
            \item номинальные (требуемые)
            \item действительные (реализуемые)
            \item потенциальные
        \end{itemize}
\end{itemize}

\paragraph{Функционально--структурная модель (ФСМ)}
--- условное изображение исследуемого объекта в форме матрицы (графа),
получаемое путём совмещения структурно--элементной и функциональной моделей.

\paragraph{Функционально--стоимостная диаграмма (ФСД)}
--- графическое представление соотношений значимости (роли) функций, качества их
исполнения и затрат на реализацию.

\paragraph{Обобщённый (комплексный) показатель качества исполнения функции}
--- безразмерная величина, служащая для выбора способа осуществления функции,
определяемая на основе экспертных оценок значимости данной функции и единичной
оценки качества исполнения функции данным способом.
Характеризует технический уровень конструкторско--технологического решения.

\subsection{Выбор стратегии ФСА}
Различают прямую и обратную стратегию ФСА.

\paragraph{Прямая стратегия}
направлена непосредственно на создание совершенного, экономичного и надёжного
изделия.

\paragraph{Обратная стратегия}
предполагает совершенствование самого ФСА, т.е. его методики, организации
нормативного обеспечения.

Прямая стратегия предпочтительнее, т.к. больше соответствует целям и задачам
дипломного проектирования.
Чаще других используются такие формы прямой стратегии как
конструктивно--структурная и функционально--последовательная.
Для проекта лучше всего подходит последовательная стратегия.
Такая стратегия типична для ФСА информационно--измерительных систем, т.к.
принятие решений по способу исполнения функции одного из элементов изделия
оказывает влияние на выбор количества и типа других элементов, т.е. на структуру
изделия в целом.

\subsection{Функционально--структурный анализ}
Изучили систему с целью определения основных направлений его совершенствования.
Для этого использовали функционально--последовательную стратегию ФСА.
Резделение на подсистемы выполнили в соответствии с функциональной схемой.

По каждой функции выявили возможные способы её осуществления.
После отклонения вариантов, не удовлетворяющих требованию технического
задания, остались наиболее конкурентоспособные альтернативы:

\begin{enumerate}
    \item Микроконтролер
        Основная функция: реализует высокоуровневую логику движения,
        исполняет команды бортовой ЭВМ, осуществляет функции корректирующего
        звена в системе управления, выполняет функции по контролю за общим
        состоянием системы.

        \begin{tabular}{|p{3.5cm}|p{3.5cm}|p{3.5cm}|p{3.5cm}|}
            \hline
            Вариант 1 & Вариант 2 & Вариант 3 & Вариант 4 \\
            \hline
            \textit{AM1802--Sitara ARM} &
            \textit{TMS320F28035 32bit} &
            \textit{LPC2132FBD64 16bit} &
            \textit{AT89C51ED2--SLSUM 8bit} \\
            \hline
        \end{tabular}

    \item Двигатель
        Основная функция: приводит в движение объект управления.

        \begin{tabular}{|p{3.5cm}|p{3.5cm}|p{3.5cm}|p{3.5cm}|}
            \hline
            Вариант 1 & Вариант 2 & Вариант 3 & Вариант 4 \\
            \hline
            Шаговый &
            Бесколлекторный &
            Индукционный &
            Коллекторный \\
            \hline
        \end{tabular}

    \item Цифровой датчик угла
        Основная функция: Преобразует угловое положение ротора в код

        \begin{tabular}{|p{3.5cm}|p{3.5cm}|p{3.5cm}|p{3.5cm}|}
            \hline
            Вариант 1 & Вариант 2 & Вариант 3 & Вариант 4 \\
            \hline
            Оптический, цифровой &
            Индукционный, цифровой &
            СКВТ + преобразователь в цифровой код &
            Потенциоиетр + АЦП \\
            \hline
        \end{tabular}

    \item Драйвер ШИМ:
        Основная функция: преобразует цифровой сигнал ШИМ в аналоговый сигнал,
        усиливает сигнал до значений необходимых двигателю.

        \begin{tabular}{|p{3.5cm}|p{3.5cm}|p{3.5cm}|p{3.5cm}|}
            \hline
            Вариант 1 & Вариант 2 & Вариант 3 & Вариант 4 \\
            \hline
            \textit{TI DRV8412} &
            \textit{FSBB15CH60 SPM27CA} &
            \textit{L6205N} &
            \textit{L293DNE} \\
            \hline
        \end{tabular}

    \item Редуктор
        Основная функция: передает механическое воздействие от двигателя к
        нагрузке, осуществляет механическую развязку, первичное усиление
        стопорящего воздействия, редукцию механического воздействия.

        \begin{tabular}{|p{3.5cm}|p{3.5cm}|p{3.5cm}|}
            \hline
            Вариант 1 & Вариант 2 & Вариант 3 \\
            \hline
            Волновой редуктор &
            Червячный редуктор &
            Редуктор с прямозубыми ЗК \\
            \hline
        \end{tabular}

    \item Операционная система
        Основная функция: Обеспечивает параллельное или псевдопараллельное
        выполнение задач (многозадачность). Распределяет ресурсы вычислительной
        системы между процессами. Разграничивает доступ различных процессов к
        ресурсам. Обеспечение надежности вычислений (невозможности одного
        вычислительного процесса намеренно или по ошибке повлиять на вычисления
        в другом процессе). Управляет ресурсами вычислительной системы.

        \begin{tabular}{|p{3.5cm}|p{3.5cm}|}
            \hline
            Вариант 1 & Вариант 2 \\
            \hline
            FreeRTOS &
            без ОС \\
            \hline
        \end{tabular}
\end{enumerate}

\subsection{Сравнение альтернатив по техническому уровню}
В таблице (\ref{technical_lvl_comparison_table}) каждой паре альтернатив
соответствуют две ячейки $(i, j)$ и $(j, i)$.
Если технический уровень альтернативы $i$ выше технического уровня альтернативы
$j$, то в ячейке $(i, j)$ проставляют $1$, а в $(j, i)$ $0$, и наоборот.
В последней колонке таблице (\ref{technical_lvl_comparison_table}) весовые
коэффициенты $D_{\text{ТУ}}$ (технический уровень соответствующих альтернатив).
\begin{table}[h!]
    \centering
    \begin{tabular}{|p{2.4cm}|p{2cm}|p{2cm}|p{2cm}|p{2cm}|p{2cm}|}
        \hline
        \begin{center} Номер варианта (алтернативы) $(i = \overline{1:n})$ \end{center} &
        \multicolumn{3}{c|}{
            \parbox{2.4cm}{
                \centering
                Номер варианта (алтернативы) $(j = \overline{1:m})$
            }
        } &
        \begin{center} $$1 + \sum_{i=1}^n E$$ \end{center} &
        \begin{center} $$\frac{1 + \sum_{i=1}^n E}{\sum_{i=1}^m E}$$ \end{center} \\

        \hline &
        \begin{center} 1 \end{center} &
        \begin{center} 2 \end{center} &
        \begin{center} 3 \end{center} &
        &
        \begin{center} $D_\text{ТУ}$ \end{center}\\

        \hline
        \centering{1} & --- &  1  &  1  & 3 & 0.5  \\ \hline
        \centering{2} &  0  & --- &  1  & 2 & 0.33 \\ \hline
        \centering{3} &  0  &  0  & --- & 1 & 0.17 \\ \hline

        \hline
        \multicolumn{4}{|c|}{итого:} & 6 & 1 \\
        \hline
    \end{tabular}
    \caption{Cравнение альтернатив по техническому уровню}
    \label{technical_lvl_comparison_table}
\end{table}


\subsection{Сравнение альтернатив по затратам}
Аналогично заполняем таблицу (\ref{costs_lvl_comparison_table}) весовыми
коэффициентами $D_\text{СТ}$ (затраты на осуществление соответствующего варианта
изделия).
\begin{table}[h!]
    \centering
    \begin{tabular}{|p{2.4cm}|p{2cm}|p{2cm}|p{2cm}|p{2cm}|p{2cm}|}
        \hline
        \begin{center} Номер варианта (алтернативы) $(i = \overline{1:n})$ \end{center} &
        \multicolumn{3}{c|}{
            \parbox{2.4cm}{
                \centering
                Номер варианта (алтернативы) $(j = \overline{1:m})$
            }
        } &
        \begin{center} $$1 + \sum_{i=1}^n E$$ \end{center} &
        \begin{center} $$\frac{1 + \sum_{i=1}^n E}{\sum_{i=1}^m E}$$ \end{center} \\

        \hline &
        \begin{center} 1 \end{center} &
        \begin{center} 2 \end{center} &
        \begin{center} 3 \end{center} &
        &
        \begin{center} $D_\text{СТ}$ \end{center}\\

        \hline
        \centering{1} & --- &  1  &  1  & 1 & 0.17 \\ \hline
        \centering{2} &  0  & --- &  1  & 3 & 0.5  \\ \hline
        \centering{3} &  0  &  0  & --- & 2 & 0.33 \\ \hline

        \hline
        \multicolumn{4}{|c|}{итого:} & 6 & 1 \\
        \hline
    \end{tabular}
    \caption{Сравнение альтернатив по затратам}
    \label{costs_lvl_comparison_table}
\end{table}

\subsection{Функционально--стоимостная гистограмма}

  \begin{tikzpicture}[scale=0.7]
    %Draw axis
    \coordinate (py) at (0,4);
    \coordinate (ny) at (0,-4);
    \coordinate (px) at (10,0);
    \draw[->,  very thick]   (0,0) --  (px);
    \draw[<->, very thick]   (ny)  --  (py);

    \coordinate (dtu) at (10,0);
    \coordinate (dst) at (10,0);

    \coordinate (dtu) at (10,0);
    \coordinate (dst) at (10,0);

    \coordinate (dtu) at (10,0);
    \coordinate (dst) at (10,0);

    \draw (0, -0.85) rectangle (2.5,2.5);
    \draw (2.5,-2.5) rectangle (5,  1.65);
    \draw (5, -1.65) rectangle (7.5,0.85);

    \node at (1.25, 0.45)  {\textit{1}};
    \node at (1.25, 3)     {\textit{0.5}};
    \node at (1.25, -1.35) {\textit{0.17}};

    \node at (3.75, 0.45) {\textit{2}};
    \node at (3.75, 2.15) {\textit{0.33}};
    \node at (3.75, -3)   {\textit{0.5}};

    \node at (6.25, 0.45)  {\textit{3}};
    \node at (6.25, 1.35)  {\textit{1.17}};
    \node at (6.25, -2.15) {\textit{0.33}};

    \node at ($(py) + (1,0)$) {$D_\text{ТУ}$};
    \node at ($(ny) + (1,0)$) {$D_\text{СТ}$};
    \node at ($(px) + (0,0.5)$) {Номер элемента};

  \end{tikzpicture}
%   \coordinate (alphaas) at ($0.8*(py)$);
%   \coordinate (alphabs) at ($0.533*(py)$);
%   \coordinate (cfas) at ($.6*(px)$);
%   \coordinate (cfbs) at ($.9*(px)$);
%   \coordinate (rl) at ($(cfas)-.2*(px)$);
%   \coordinate (rla) at ($(rl)-.1*(px)$);
%   \coordinate (rlb) at ($(rl)+.1*(px)$);

%   \draw[important line] let \p1=(alphaas), \p2=(cfas) in
%   (\p1) node[left] {$\alpha_s^\A$} -| (\x2, \y1) -| (\p2)
%   node[below] {$\mathit{NV^\A_s}$};
%   \draw[] let \p1=(alphabs), \p2=(cfbs) in
%   (\p1) node[left] {$\alpha_s^\B$} -| (\x2, \y1) -| (\p2)
%   node[below] {$\mathit{NV^\B_s}$};
%   \draw[help lines] let \p1=(rl), \p2=(y) in
%   (\p1) node[below] {$\hat{r}_L$} -- (\x1, \y2);

%   \draw let \p1=($(alphaas)-(alphabs)$), \p2=(rl), \p3=(alphabs) in
%   ($(.5*\x2, .5*\y1+\y3)$) node {$C$};
%   %%D
%   \draw let \p1=($(alphaas)-(alphabs)$), \p2=($(cfas)-(rl)$),
%   \p3=(alphabs), \p4=(rl) in
%   ($(.5*\x2+\x4, .5*\y1+\y3)$) node {$D$};
%   %%E
%   \draw let \p1=(alphabs), \p2=(rl) in
%   ($(.5*\x2, .5*\y1)$) node {$E$};
%   %%F
%   \draw let \p1=(alphabs), \p2=($(cfas)-(rl)$), \p3=(rl) in
%   ($(.5*\x2+\x3, .5*\y1)$) node {$F$};
%   %%G
%   \draw let \p1=(alphabs), \p2=($(cfbs)-(cfas)$), \p3=(cfas) in
%   ($(.5*\x2+\x3, .5*\y1)$) node {$G$};
