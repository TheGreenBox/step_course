\section{Организационно экономическая часть}
Функционально--стоимостной анализ (ФСА) --- метод системного
анализа функций и свойств различных объектов и затрат на их
реализацию.
Наиболее широко ФСА в настоящее время применяется для технических
объектов--изделий, их частей и деталей, оборудования, технологических процессов
производства.
Основная цель анализа при этом --- выявление резервов снижения затрат на
исследования и разработки, производство и эксплуатацию рассматриваемых объектов,
при условии сохранения соответсвия техническому заданию конечного изделия.
Кроме конструирования и технологии технических объектов в поле деятельности ФСА
в настоящее время включаются организационные и управленческие процессы,
производственные структуры предприятий, объединений и научно--исследовательских
организаций.
Если исходить из общей предпосылки системного анализа, то объектом ФСА может
быть любой элемент сложной производственно--экономической системы, отвечающий
требованиям выделенных выше признаков.


Развитие теории ФСА нашло широкое применение в всех отраслях машиностроительной
промышленности.
Это связано с системностью метода, ставящего своей задачей в каждом конкретном
случае выявить структуру рассматриваемого объекта, разложить его на простейшие
элементы, дать им объективную оценку (со стороны потребительской стоимости
--- интегрального качества и со стороны стоимости затрат на исследования,
производство и эксплуатацию).
В силу своей системности ФСА позволяет выявить в каждом анализируемом объекте
причинно--следственные связи между качеством, эксплуатационно--техническими
характеристиками и стоимостью.
На основе этого создаются основания для исключения механических методов
планирования затрат от достигнутого уровня, установления нормативов на основе
сложившегося уровня трудоёмкости себестоимости и расхода материалов.


Достоинством ФСА является наличие достаточно простых расчётных и графических
методов, позволяющих дать двойственную количественную оценку выявленных
причинно--следственных связей. Это достоинство ставит ФСА в ряд наиболее
эффективных методов анализа не только технических, но и
производственно--экономических систем, структур, методов организации и
планирования, управления производством и научными исследованиями. Однако работы
по ФСА проводятся в отрыве от экономических расчётов на предприятиях и в
объединениях.
Поэтому экономические нормативы действующего производства не охвачены
функциональным подходом, базируются на предметном экономическом анализе,
планировании от достигнутого уровня.
Функционально--стоимостной анализ управленческих систем позволяет выполнить
следующие виды работ:
\begin{itemize}
    \item Определение и проведение общего анализа себестоимости
    бизнес--процессов на предприятии (маркетинг, производство продукции и
    оказание услуг, сбыт, менеджмент качества, техническое и гарантийное
    обслуживание и др.)
    \item Проведение функционального анализа, связанного с установлением и
    обоснованием выполняемых структурными подразделениями предприятий функций с
    целью обеспечения выпуска продукции высокого качества и оказания услуг
    \item Определение и анализ основных, дополнительных и ненужных
    функциональных затрат
    \item Сравнительный анализ альтернативных вариантов снижения затрат в
    производстве, сбыте и управлении за счёт упорядочения функций структурных
    подразделений предприятия
    \item Анализ интегрированного улучшения результатов деятельности
предприятия
\end{itemize}

В настоящее время метод ФСА стал всеобъемлющим инструментом
оценки систем, процессов и концепций.

\subsection{Основные термины и определения}

\paragraph{Функционально--стоимостной анализ (ФСА)} --- метод системного
исследования функций объектов, направленный на обеспечение общественно
необходимых потребительских свойств объектов и минимальных затрат на их
проявление на всех этапах жизненного цикла.

\paragraph{Объект ФСА} --- изделие, технологический процесс, производственная
организационная структура, а также их отдельные элементы, подвергаемые
исследованию в целях выбора оптимального варианта реализации выполняемых ими
основных функций при минимальных затратах.

\paragraph{Функция} --- внешнее проявление свойств объекта в определённых
условиях, сущность объекта.

\paragraph{Конструкция} --- форма проявления функции.

\paragraph{Носитель функции} --- материальный или иной объект либо отдельные
его элементы, а также их совокупность, реализующие функцию.

\paragraph{Классификация функций} --- группировка функций по определённым
признакам.
Различают следующие группы:
\begin{itemize}
    \item По области проявления:
        \begin{itemize}
            \item общеобъектные (внешние)
            \item внутриобъектные (внутренние)
        \end{itemize}
    \item По роли в удовлетворении потребностей:
        \begin{itemize}
            \item среди внешних
                \begin{itemize}
                    \item главные (эксплуатационные)
                    \item второстепенные
                \end{itemize}
            \item среди внутренних
                \begin{itemize}
                    \item основные (рабочие)
                    \item вспомогательные
                \end{itemize}
        \end{itemize}
    \item По степени необходимости:
        \begin{itemize}
            \item необходимые (полезные)
            \item излишние (бесполезные, вредные)
        \end{itemize}
    \item По характеру проявления:
        \begin{itemize}
            \item номинальные (требуемые)
            \item действительные (реализуемые)
            \item потенциальные
        \end{itemize}
\end{itemize}

\paragraph{Функционально--структурная модель (ФСМ)}
--- условное изображение исследуемого объекта в форме матрицы (графа),
получаемое путём совмещения структурно--элементной и функциональной моделей.

\paragraph{Функционально--стоимостная диаграмма (ФСД)}
--- графическое представление соотношений значимости (роли) функций, качества их
исполнения и затрат на реализацию.

\paragraph{Обобщённый (комплексный) показатель качества исполнения функции}
--- безразмерная величина, служащая для выбора способа осуществления функции,
определяемая на основе экспертных оценок значимости данной функции и единичной
оценки качества исполнения функции данным способом.
Характеризует технический уровень конструкторско--технологического решения.

\subsection{Выбор стратегии ФСА}
Различают прямую и обратную стратегию ФСА.

\paragraph{Прямая стратегия}
направлена непосредственно на создание совершенного, экономичного и надёжного
изделия.

\paragraph{Обратная стратегия}
предполагает совершенствование самого ФСА, т.е. его методики, организации
нормативного обеспечения.

Прямая стратегия предпочтительнее, т.к. больше соответствует целям и задачам
дипломного проектирования.
Чаще других используются такие формы прямой стратегии как
конструктивно--структурная и функционально--последовательная.
Для проекта лучше всего подходит последовательная стратегия.
Такая стратегия типична для ФСА информационно--измерительных систем, т.к.
принятие решений по способу исполнения функции одного из элементов изделия
оказывает влияние на выбор количества и типа других элементов, т.е. на структуру
изделия в целом.

\subsection{Функционально--структурный анализ}
Изучили систему с целью определения основных направлений его совершенствования.
Для этого использовали функционально--последовательную стратегию ФСА.
Резделение на подсистемы выполнили в соответствии с функциональной схемой.

По каждой функции выявили возможные способы её осуществления.
После отклонения вариантов, не удовлетворяющих требованию технического
задания, остались наиболее конкурентоспособные альтернативы:

\begin{enumerate}
    \item Микроконтролер

        Основная функция: реализует высокоуровневую логику движения,
        исполняет команды бортовой ЭВМ, осуществляет функции корректирующего
        звена в системе управления, выполняет функции по контролю за общим
        состоянием системы.

        \begin{tabular}{|p{3.5cm}|p{3.5cm}|p{3.5cm}|}
            \hline
            Вариант 1 & Вариант 2 & Вариант 3 \\
            \hline
            \textit{AM1802--Sitara ARM} &
            \textit{TMS320F28035 32bit} &
            \textit{AT89C51ED2--SLSUM 8bit} \\
            \hline
        \end{tabular}

    \item Двигатель

        Основная функция: приводит в движение объект управления.

        \begin{tabular}{|p{3.5cm}|p{3.5cm}|p{3.5cm}|}
            \hline
            Вариант 1 & Вариант 2 & Вариант 3 \\
            \hline
            Шаговый &
            Индукционный &
            Коллекторный \\
            \hline
        \end{tabular}

    \item Цифровой датчик угла

        Основная функция: Преобразует угловое положение ротора в код

        \begin{tabular}{|p{3.5cm}|p{3.5cm}|p{3.5cm}|}
            \hline
            Вариант 1 & Вариант 2 & Вариант 3 \\
            \hline
            Оптический, цифровой &
            Индукционный, цифровой &
            Потенциоиетр + АЦП \\
            \hline
        \end{tabular}

    \item Драйвер ШИМ:

        Основная функция: преобразует цифровой сигнал ШИМ в аналоговый сигнал,
        усиливает сигнал до значений необходимых двигателю.

        \begin{tabular}{|p{3.5cm}|p{3.5cm}|p{3.5cm}|p{3.5cm}|}
            \hline
            Вариант 1 & Вариант 2 & Вариант 3 & Вариант 4 \\
            \hline
            \textit{TI DRV8412} &
            \textit{FSBB15CH60 SPM27CA} &
            \textit{L293DNE} \\
            \hline
        \end{tabular}

    \item Редуктор

        Основная функция: передает механическое воздействие от двигателя к
        нагрузке, осуществляет механическую развязку, первичное усиление
        стопорящего воздействия, редукцию механического воздействия.

        \begin{tabular}{|p{3.5cm}|p{3.5cm}|p{3.5cm}|}
            \hline
            Вариант 1 & Вариант 2 & Вариант 3 \\
            \hline
            Волновой редуктор &
            Червячный редуктор &
            Редуктор с прямозубыми ЗК \\
            \hline
        \end{tabular}

    \item Операционная система

        Основная функция: Обеспечивает параллельное или псевдопараллельное
        выполнение задач (многозадачность). Распределяет ресурсы вычислительной
        системы между процессами. Разграничивает доступ различных процессов к
        ресурсам. Обеспечение надежности вычислений (невозможности одного
        вычислительного процесса намеренно или по ошибке повлиять на вычисления
        в другом процессе). Управляет ресурсами вычислительной системы.

        \begin{tabular}{|p{3.5cm}|p{3.5cm}|p{3.5cm}|}
            \hline
            Вариант 1 & Вариант 2 & Вариант 3 \\
            \hline
            Linux &
            FreeRTOS &
            без ОС \\
            \hline
        \end{tabular}
\end{enumerate}

\subsection{Влияние функций на технический уровень изделия}
В таблице (\ref{tbl_technical_lvl_comparison}) каждой паре функций
соответствуют две ячейки $(i, j)$ и $(j, i)$.
Если значимость функции $i$ выше значимости функции
$j$, то в ячейке $(i, j)$ проставляют $1$, а в $(j, i)$ $0$, и наоборот.
В последней колонке таблице (\ref{tbl_technical_lvl_comparison}) весовые
коэффициенты $D_{\text{ТУ}}$ (значимость соотвествующих функций и, следоавтельно,
их влияние на технический уровень конечного изделия.
\begin{table}[h!]
    \centering
    \begin{tabular}{|c|c|c|c|c|c|c|c|c|}
        \hline
        \multirow{2}{2.4cm}{
            \centering
            Номер варианта (алтернативы) ($i = \overline{1:n}$)
        } &
        \multicolumn{6}{c|}{
            \parbox[t]{2.4cm}{
                \centering
                Номер варианта (алтернативы) ($j = \overline{1:m}$)
            }
        } &
        \multirow{2}{1.7cm}[0pt]{
            \centering
            $$1 + \sum_{i=1}^n E$$
        } &
        \multirow{2}{3.2cm}{
            \centering
            $$ D_\text{ТУ} = \frac{1 + \sum_{i=1}^n E}{\sum_{i=1}^m E}$$
        } \\
        &
        \centering{1} &
        \centering{2} &
        \centering{3} &
        \centering{4} &
        \centering{5} &
        \centering{6} & & \\
        \hline \hline
        \centering{1} &---& 0 & 0 & 1 & 0 & 1 & 3 & 0.142857 \\ \hline
        \centering{2} & 1 &---& 1 & 1 & 1 & 1 & 6 & 0.285714 \\ \hline
        \centering{3} & 1 & 0 &---& 1 & 0 & 1 & 4 & 0.190476 \\ \hline
        \centering{4} & 0 & 0 & 0 &---& 0 & 1 & 2 & 0.095238 \\ \hline
        \centering{5} & 1 & 0 & 1 & 1 &---& 1 & 5 & 0.238095 \\ \hline
        \centering{6} & 0 & 0 & 0 & 0 & 0 &---& 1 & 0.047619 \\ \hline
        \hline
        \multicolumn{7}{|r|}{итого:} & 21 & 1 \\
        \hline
    \end{tabular}
    \caption{Cравнение альтернатив по техническому уровню}
    \label{tbl_technical_lvl_comparison}
\end{table}


\subsection{Сравнение функций по затратам}
Аналогично заполнили таблицу (\ref{tbl_costs_lvl_comparison}) весовыми
коэффициентами $D_\text{СТ}$ (затраты на осуществление соответствующего варианта
изделия).
\begin{table}[h!]
    \centering
    \begin{tabular}{|c|c|c|c|c|c|c|c|c|}
        \hline
        \multirow{2}{2.4cm}{
            \centering
            Номер варианта (алтернативы) ($i = \overline{1:n}$)
        } &
        \multicolumn{6}{c|}{
            \parbox[t]{2.4cm}{
                \centering
                Номер варианта (алтернативы) ($j = \overline{1:m}$)
            }
        } &
        \multirow{2}{1.7cm}[0pt]{
            \centering
            $$1 + \sum_{i=1}^n E$$
        } &
        \multirow{2}{3.2cm}{
            \centering
            $$ D_\text{CТ} = \frac{1 + \sum_{i=1}^n E}{\sum_{i=1}^m E}$$
        } \\
        &
        \centering{1} &
        \centering{2} &
        \centering{3} &
        \centering{4} &
        \centering{5} &
        \centering{6} & & \\
        \hline \hline
        \centering{1} &---& 0 & 0 & 1 & 0 & 1 & 3 & 0.142857 \\ \hline
        \centering{2} & 1 &---& 1 & 1 & 0 & 1 & 5 & 0.238095 \\ \hline
        \centering{3} & 1 & 0 &---& 1 & 0 & 1 & 4 & 0.190476 \\ \hline
        \centering{4} & 0 & 0 & 0 &---& 0 & 1 & 2 & 0.095238 \\ \hline
        \centering{5} & 1 & 1 & 1 & 1 &---& 1 & 6 & 0.285714 \\ \hline
        \centering{6} & 0 & 0 & 0 & 0 & 0 &---& 1 & 0.047619 \\ \hline
        \hline
        \multicolumn{7}{|r|}{итого:} & 21 & 1 \\
        \hline
    \end{tabular}
    \caption{Cравнение альтернатив по техническому уровню}
    \label{tbl_costs_lvl_comparison}
\end{table}

\subsection{Функционально--стоимостная гистограмма}
В результате построения функционально--стоимостной гистограммы
(рис. \ref{pic_function_costs_histogram}) получаются два вектора--столбца
значений $D_\text{ТУ}(i)$ и $D_\text{СТ}(i)$. Где $i$ - число функций в модели,
величины $D$ отражают долю влияния данной функции на технический уровень изделия
и долю затрат на исполнение функции в себестоимости изделия.
Функционально--стоимостная гистограмма строится на основании данных таблиц
(\ref{tbl_technical_lvl_comparison}) и (\ref{tbl_costs_lvl_comparison}).

\begin{figure}[h!]
    \centering
    \begin{tikzpicture}[scale=0.2]
        \coordinate (dtu1) at (0, 14.2857);
        \coordinate (dtu2) at (0, 28.5714);
        \coordinate (dtu3) at (0, 19.0476);
        \coordinate (dtu4) at (0, 09.5238);
        \coordinate (dtu5) at (0, 23.8095);
        \coordinate (dtu6) at (0, 04.7619);

        \coordinate (dst1) at (0, -14.2857);
        \coordinate (dst2) at (0, -23.8095);
        \coordinate (dst3) at (0, -19.0476);
        \coordinate (dst4) at (0, -09.5238);
        \coordinate (dst5) at (0, -28.5714);
        \coordinate (dst6) at (0, -04.7619);

        \coordinate (dx) at (11, 0);
        \coordinate (dy) at (0, 2);

        %Draw axis
        \coordinate (py) at (0,30);
        \coordinate (ny) at (0,-30);
        \coordinate (px) at ($7*(dx)$);
        \draw[->,  very thick]   (0,0) --  (px);
        \draw[<->, very thick]   (ny)  --  (py);

        \draw ($(dtu1) + 0*(dx)$) rectangle ($(dst1) + 1*(dx)$);
        \draw ($(dtu2) + 1*(dx)$) rectangle ($(dst2) + 2*(dx)$);
        \draw ($(dtu3) + 2*(dx)$) rectangle ($(dst3) + 3*(dx)$);
        \draw ($(dtu4) + 3*(dx)$) rectangle ($(dst4) + 4*(dx)$);
        \draw ($(dtu5) + 4*(dx)$) rectangle ($(dst5) + 5*(dx)$);
        \draw ($(dtu6) + 5*(dx)$) rectangle ($(dst6) + 6*(dx)$);

        \node at ($(dtu1) + 0*(dx) + 0.5*(dx) + (dy)$) {0.142857};
        \node at ($(dtu2) + 1*(dx) + 0.5*(dx) + (dy)$) {0.285714};
        \node at ($(dtu3) + 2*(dx) + 0.5*(dx) + (dy)$) {0.190476};
        \node at ($(dtu4) + 3*(dx) + 0.5*(dx) + (dy)$) {0.095238};
        \node at ($(dtu5) + 4*(dx) + 0.5*(dx) + (dy)$) {0.238095};
        \node at ($(dtu6) + 5*(dx) + 0.5*(dx) + (dy)$) {0.047619};

        \node at ($(dst1) + 0*(dx) + 0.5*(dx) - (dy)$) {0.142857};
        \node at ($(dst2) + 1*(dx) + 0.5*(dx) - (dy)$) {0.238095};
        \node at ($(dst3) + 2*(dx) + 0.5*(dx) - (dy)$) {0.190476};
        \node at ($(dst4) + 3*(dx) + 0.5*(dx) - (dy)$) {0.095238};
        \node at ($(dst5) + 4*(dx) + 0.5*(dx) - (dy)$) {0.285714};
        \node at ($(dst6) + 5*(dx) + 0.5*(dx) - (dy)$) {0.047619};

        \node at ($0.5*(dx) + (dy)$) {1};
        \node at ($1.5*(dx) + (dy)$) {2};
        \node at ($2.5*(dx) + (dy)$) {3};
        \node at ($3.5*(dx) + (dy)$) {4};
        \node at ($4.5*(dx) + (dy)$) {5};
        \node at ($5.5*(dx) + (dy)$) {6};

        \node at ($(py) + 0.5*(dx)$) {$D_\text{ТУ}$};
        \node at ($(ny) + 0.5*(dx)$) {$D_\text{СТ}$};
        \node at ($(px) + (dy)$) {Номер элемента};
    \end{tikzpicture}
    \caption{Функционально--стоимостная гистограмма}
    \label{pic_function_costs_histogram}
\end{figure}


\subsection{Оценка способов исполнения функций}
Составим функционально-стоимостные матрицы:
\begin{enumerate}
    \item Матрица $U(i,j)$ --- матрица оценки технического уровня
    \item Матрица $С(i,j)$ --- матрица оценки затрат
\end{enumerate}

Следующие далее таблицы составлены попарно аналогично таблицам
(\ref{tbl_technical_lvl_comparison}) и (\ref{tbl_costs_lvl_comparison}).
соответственно. В них последний столбец будет соответствовать строке, отвечающей
за технический уровень или затраты по каждому элементу (исполняемой функции).
Сравнение альтернатив по техническому уровню

\begin{enumerate}
    \item Микроконтролер
    \begin{table}[h!]
        \centering
        \begin{tabular}{|c|p{2cm}|p{2cm}|p{2cm}|c|}
            \hline
            \multirow{2}{2.4cm}{
                \centering
                Номер варианта (алтернативы) $(i = \overline{1:n})$
            } &
            \multicolumn{3}{p{6cm}|}{
                \begin{center}
                    Номер варианта (алтернативы) $(j = \overline{1:m})$
                \end{center}
            } &
            \multirow{2}{2cm}{
                $$1 + \sum_{i=1}^n E$$
            } \\
            & \centering{ 1 } & \centering{ 2 } & \centering{ 3 } & \\
            \hline
            \centering{1} & \centering{---} & \centering{ 1 } & \centering{ 1 } & 1 \\ \hline
            \centering{2} & \centering{ 0 } & \centering{---} & \centering{ 1 } & 3 \\ \hline
            \centering{3} & \centering{ 0 } & \centering{ 0 } & \centering{---} & 2 \\ \hline
        \end{tabular}
        \caption{Сравнение альтернатив микроконтролеров по техническому уровню}
        \label{tbl_mcu_tech_lvl_comparison}
    \end{table}

    \item Двигатель

    \item Цифровой датчик угла

    \item Драйвер ШИМ:

    \item Редуктор

    \item Операционная система

\end{enumerate}
