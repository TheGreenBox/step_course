\newpage
\section{Организационно~--~экономическая часть}

\subsection{Общие сведения о функционально~--~стоимостном анализе}
Функционально~--~стоимостной анализ (ФСА) --- метод системного
анализа функций и свойств различных объектов и затрат на их
реализацию.
Наиболее широко ФСА в настоящее время применяется для технических
объектов~--~изделий, их частей и деталей, оборудования, технологических процессов
производства.
Основная цель анализа при этом --- выявление резервов снижения затрат на
исследования и разработки, производство и эксплуатацию рассматриваемых объектов,
при условии сохранения соответсвия техническому заданию конечного изделия.
Кроме конструирования и технологии технических объектов в поле деятельности ФСА
в настоящее время включаются организационные и управленческие процессы,
производственные структуры предприятий, объединений и научно~--~исследовательских
организаций.
Если исходить из общей предпосылки системного анализа, то объектом ФСА может
быть любой элемент сложной производственно~--~экономической системы, отвечающий
требованиям выделенных выше признаков.


Развитие теории ФСА нашло широкое применение в всех отраслях машиностроительной
промышленности.
Это связано с системностью метода, ставящего своей задачей в каждом конкретном
случае выявить структуру рассматриваемого объекта, разложить его на простейшие
элементы, дать им объективную оценку (со стороны потребительской стоимости ---
интегрального качества и со стороны стоимости затрат на исследования,
производство и эксплуатацию).
В силу своей системности ФСА позволяет выявить в каждом анализируемом объекте
причинно~--~следственные связи между качеством, эксплуатационно~--~техническими
характеристиками и стоимостью.
На основе этого создаются основания для исключения механических методов
планирования затрат от достигнутого уровня, установления нормативов на основе
сложившегося уровня трудоёмкости себестоимости и расхода материалов.


Достоинством ФСА является наличие достаточно простых расчётных и графических
методов, позволяющих дать двойственную количественную оценку выявленных
причинно~--~следственных связей. Это достоинство ставит ФСА в ряд наиболее
эффективных методов анализа не только технических, но и
производственно~--~экономических систем, структур, методов организации и
планирования, управления производством и научными исследованиями. Однако работы
по ФСА проводятся в отрыве от экономических расчётов на предприятиях и в
объединениях.
Поэтому экономические нормативы действующего производства не охвачены
функциональным подходом, базируются на предметном экономическом анализе,
планировании от достигнутого уровня.
Функционально~--~стоимостной анализ управленческих систем позволяет выполнить
следующие виды работ:
\begin{itemize}
    \item Определение и проведение общего анализа себестоимости
    бизнес~--~процессов на предприятии (маркетинг, производство продукции и
    оказание услуг, сбыт, менеджмент качества, техническое и гарантийное
    обслуживание и др.)
    \item Проведение функционального анализа, связанного с установлением и
    обоснованием функций, выполняемых структурными подразделениями предприятий с
    целью обеспечения выпуска продукции высокого качества и оказания услуг
    \item Определение и анализ основных, дополнительных и ненужных
    функциональных затрат
    \item Сравнительный анализ альтернативных вариантов снижения затрат в
    производстве, сбыте и управлении за счёт упорядочения функций структурных
    подразделений предприятия
    \item Анализ интегрированного улучшения результатов деятельности
предприятия
\end{itemize}

В настоящее время метод ФСА стал всеобъемлющим инструментом
оценки систем, процессов и концепций.

\newpage
\subsection{Основные термины и определения}

\textbf{Функционально~--~стоимостной анализ (ФСА)} --- метод системного
исследования функций объектов, направленный на обеспечение общественно
необходимых потребительских свойств объектов и минимальных затрат на их
проявление на всех этапах жизненного цикла.

\textbf{Объект ФСА} --- изделие, технологический процесс, производственная
организационная структура, а также их отдельные элементы, подвергаемые
исследованию в целях выбора оптимального варианта реализации выполняемых ими
основных функций при минимальных затратах.

\textbf{Функция} --- внешнее проявление свойств объекта в определённых
условиях, сущность объекта.

\textbf{Конструкция} --- форма проявления функции.

\textbf{Носитель функции} --- материальный или иной объект либо отдельные
его элементы, а также их совокупность, реализующие функцию.

\textbf{Классификация функций} --- группировка функций по определённым
признакам.
Различают следующие группы:
\begin{itemize}
    \item По области проявления:
        \begin{itemize}
            \item общеобъектные (внешние)
            \item внутриобъектные (внутренние)
        \end{itemize}
    \item По роли в удовлетворении потребностей:
        \begin{itemize}
            \item среди внешних
                \begin{itemize}
                    \item главные (эксплуатационные)
                    \item второстепенные
                \end{itemize}
            \item среди внутренних
                \begin{itemize}
                    \item основные (рабочие)
                    \item вспомогательные
                \end{itemize}
        \end{itemize}
    \item По степени необходимости:
        \begin{itemize}
            \item необходимые (полезные)
            \item излишние (бесполезные, вредные)
        \end{itemize}
    \item По характеру проявления:
        \begin{itemize}
            \item номинальные (требуемые)
            \item действительные (реализуемые)
            \item потенциальные
        \end{itemize}
\end{itemize}

\textbf{Функционально~--~структурная модель (ФСМ)} --- условное изображение
исследуемого объекта в форме матрицы (графа),
получаемое путём совмещения структурно~--~элементной и функциональной моделей.

\textbf{Функционально~--~стоимостная диаграмма (ФСД)} --- графическое
представление соотношений значимости (роли) функций, качества их
исполнения и затрат на реализацию.

\textbf{Обобщённый показатель качества исполнения функции}
--- безразмерная величина, служащая для выбора способа осуществления функции,
определяемая на основе экспертных оценок значимости данной функции и единичной
оценки качества исполнения функции данным способом.
Характеризует технический уровень конструкторско~--~технологического решения.

\subsubsection{Выбор стратегии ФСА}
Различают прямую и обратную стратегию ФСА.

\textbf{Прямая стратегия}
направлена непосредственно на создание совершенного, экономичного и надёжного
изделия.

\textbf{Обратная стратегия}
предполагает совершенствование самого ФСА, т.е. его методики, организации
нормативного обеспечения.

Прямая стратегия предпочтительнее, т.к. больше соответствует целям и задачам
дипломного проектирования.
Чаще других используются такие формы прямой стратегии как
конструктивно~--~структурная и функционально~--~последовательная.
Для проекта лучше всего подходит последовательная стратегия.
Такая стратегия типична для ФСА информационно~--~измерительных систем, т.к.
принятие решений по способу исполнения функции одного из элементов изделия
оказывает влияние на выбор количества и типа других элементов, т.е. на структуру
изделия в целом.

\newpage
\subsection{Функционально~--~структурный анализ}
Изучили систему с целью определения основных направлений его совершенствования.
Для этого использовали функционально~--~последовательную стратегию ФСА
\cite[стр. 11]{econ_FSA}.
Резделение на подсистемы выполнили в соответствии с функциональной схемой.

По каждой функции выявили возможные способы её осуществления.
После отклонения вариантов, не удовлетворяющих требованию технического
задания, остались наиболее конкурентоспособные альтернативы:

\paragraph{Микроконтроллер}
Реализует высокоуровневую логику движения, исполняет команды бортовой ЭВМ,
осуществляет функции корректирующего звена в системе управления, выполняет
функции по контролю за общим состоянием системы.

\begin{tabular}{|p{5cm}|p{5cm}|p{5cm}|}
    \hline
    Вариант 1 & Вариант 2 & Вариант 3 \\
    \hline
    \textit{AM1802-Sitara ARM} &
    \textit{AT89C51ED2-SLSUM} &
    \textit{TMS320F28035} \\
    \hline
\end{tabular}

\paragraph{Двигатель}
Приводит в движение объект управления.

\begin{tabular}{|p{5cm}|p{5cm}|p{5cm}|}
    \hline
    Вариант 1 & Вариант 2 & Вариант 3 \\
    \hline
    Шаговый &
    Индукционный &
    Коллекторный \\
    \hline
\end{tabular}

\paragraph{Цифровой датчик угла}
Преобразует угловое положение ротора в цифровой сигнал.

\begin{tabular}{|p{5cm}|p{5cm}|p{5cm}|}
    \hline
    Вариант 1 & Вариант 2 & Вариант 3 \\
    \hline
    Потенциоиметр + АЦП &
    Индукционный, цифровой &
    Оптический, цифровой \\
    \hline
\end{tabular}

\paragraph{Драйвер ШИМ}
Преобразует цифровой сигнал ШИМ в аналоговый сигнал,
усиливает сигнал до значений необходимых двигателю.

\begin{tabular}{|p{5cm}|p{5cm}|p{5cm}|}
    \hline
    Вариант 1 & Вариант 2 & Вариант 3 \\
    \hline
    \textit{L293DNE} &
    \textit{FSBB15CH60 SPM27CA} &
    \textit{TI DRV8412} \\
    \hline
\end{tabular}

\paragraph{Редуктор}
Передает механическое воздействие от двигателя к
нагрузке, осуществляет механическую развязку, первичное усиление
стопорящего воздействия, редукцию механического воздействия.

\begin{tabular}{|p{5cm}|p{5cm}|p{5cm}|}
    \hline
    Вариант 1 & Вариант 2 & Вариант 3 \\
    \hline
    Волновой редуктор &
    Червячный редуктор &
    Редуктор с прямозубыми ЗК \\
    \hline
\end{tabular}

\paragraph{Операционная система}
Обеспечивает параллельное или псевдопараллельное
выполнение задач (многозадачность), распределяет ресурсы вычислительной
системы между процессами, разграничивает доступ различных процессов к
ресурсам. Обеспечение надежности вычислений (невозможности одного
вычислительного процесса намеренно или по ошибке повлиять на вычисления
в другом процессе), управляет ресурсами вычислительной системы.

\begin{tabular}{|p{5cm}|p{5cm}|p{5cm}|}
    \hline
    Вариант 1 & Вариант 2 & Вариант 3 \\
    \hline
    Linux &
    FreeRTOS &
    без ОС \\
    \hline
\end{tabular}


\subsubsection{Влияние функций на технический уровень изделия}
В таблице (\ref{tbl_technical_lvl_comparison}) каждой паре функций
соответствуют две ячейки $(i, j)$ и $(j, i)$.
Если значимость функции $i$ выше значимости функции
$j$, то в ячейке $(i, j)$ проставляют $1$, а в $(j, i)$ $0$, и наоборот.
В последней колонке таблице (\ref{tbl_technical_lvl_comparison}) весовые
коэффициенты $D_{\text{ТУ}}$ (значимость соотвествующих функций и, следовательно,
их влияние на технический уровень конечного изделия).
\begin{table}[ht!]
    \centering
    \begin{tabular}{|c|c|c|c|c|c|c|c|c|}
        \hline
        \multirow{2}{2.4cm}{
            \centering
            Номер варианта (альтернативы) ($i = \overline{1:n}$)
        } &
        \multicolumn{6}{c|}{
            \parbox[t]{2.4cm}{
                \centering
                Номер варианта (альтернативы) ($j = \overline{1:m}$)
            }
        } &
        \multirow{2}{1.7cm}[0pt]{
            \centering
            $$1 + \sum_{i=1}^n E_i$$
        } &
        \multirow{2}{3.2cm}{
            \centering
            $$ D_\text{ТУ} = \frac{1 + \sum_{i=1}^n E_i}{\sum_{i=1}^m E_i}$$
        } \\
        &
        \centering{1} &
        \centering{2} &
        \centering{3} &
        \centering{4} &
        \centering{5} &
        \centering{6} & & \\
        \hline \hline
        \centering{1} &---& 0 & 0 & 1 & 0 & 1 & 3 & 0.142857 \\ \hline
        \centering{2} & 1 &---& 1 & 1 & 1 & 1 & 6 & 0.285714 \\ \hline
        \centering{3} & 1 & 0 &---& 1 & 0 & 1 & 4 & 0.190476 \\ \hline
        \centering{4} & 0 & 0 & 0 &---& 0 & 1 & 2 & 0.095238 \\ \hline
        \centering{5} & 1 & 0 & 1 & 1 &---& 1 & 5 & 0.238095 \\ \hline
        \centering{6} & 0 & 0 & 0 & 0 & 0 &---& 1 & 0.047619 \\ \hline
        \hline
        \multicolumn{7}{|r|}{итого:} & 21 & 1 \\
        \hline
    \end{tabular}
    \caption{Cравнение альтернатив по техническому уровню}
    \label{tbl_technical_lvl_comparison}
\end{table}

\subsubsection{Сравнение функций по затратам}
Аналогично заполнили таблицу (\ref{tbl_costs_lvl_comparison}) весовым соответсвенно.
коэффициентами $D_\text{СТ}$ (затраты на осуществление соответствующего варианта
изделия).
\begin{table}[ht!]
    \centering
    \begin{tabular}{|c|c|c|c|c|c|c|c|c|}
        \hline
        \multirow{2}{2.4cm}{
            \centering
            Номер варианта (альтернативы) ($i = \overline{1:n}$)
        } &
        \multicolumn{6}{c|}{
            \parbox[t]{2.4cm}{
                \centering
                Номер варианта (альтернативы) ($j = \overline{1:m}$)
            }
        } &
        \multirow{2}{1.7cm}[0pt]{
            \centering
            $$1 + \sum_{i=1}^n E_i$$
        } &
        \multirow{2}{3.2cm}{
            \centering
            $$ D_\text{CТ} = \frac{1 + \sum_{i=1}^n E_i}{\sum_{i=1}^m E_i}$$
        } \\
        &
        \centering{1} &
        \centering{2} &
        \centering{3} &
        \centering{4} &
        \centering{5} &
        \centering{6} & & \\
        \hline \hline
        \centering{1} &---& 0 & 0 & 1 & 0 & 1 & 3 & 0.142857 \\ \hline
        \centering{2} & 1 &---& 1 & 1 & 0 & 1 & 5 & 0.238095 \\ \hline
        \centering{3} & 1 & 0 &---& 1 & 0 & 1 & 4 & 0.190476 \\ \hline
        \centering{4} & 0 & 0 & 0 &---& 0 & 1 & 2 & 0.095238 \\ \hline
        \centering{5} & 1 & 1 & 1 & 1 &---& 1 & 6 & 0.285714 \\ \hline
        \centering{6} & 0 & 0 & 0 & 0 & 0 &---& 1 & 0.047619 \\ \hline
        \hline
        \multicolumn{7}{|r|}{итого:} & 21 & 1 \\
        \hline
    \end{tabular}
    \caption{Cравнение альтернатив по техническому уровню}
    \label{tbl_costs_lvl_comparison}
\end{table}


\subsubsection{Функционально~--~стоимостная гистограмма}
В результате построения функционально~--~стоимостной гистограммы
(рис. \ref{pic_function_costs_histogram}) получаются два вектора~--~столбца
значений $D_\text{ТУ}(i)$ и $D_\text{СТ}(i)$, где $i$ - число функций в модели,
величины $D$ отражают долю влияния данной функции на технический уровень изделия
и долю затрат на исполнение функции в себестоимости изделия соответсвенно.
Функционально~--~стоимостная гистограмма строится на основании данных таблиц
(\ref{tbl_technical_lvl_comparison}) и (\ref{tbl_costs_lvl_comparison}).

\begin{figure}[ht]
    \centering
    \begin{tikzpicture}[scale=0.2]
        \coordinate (dtu1) at (0, 14.2857);
        \coordinate (dtu2) at (0, 28.5714);
        \coordinate (dtu3) at (0, 19.0476);
        \coordinate (dtu4) at (0, 09.5238);
        \coordinate (dtu5) at (0, 23.8095);
        \coordinate (dtu6) at (0, 04.7619);

        \coordinate (dst1) at (0, -14.2857);
        \coordinate (dst2) at (0, -23.8095);
        \coordinate (dst3) at (0, -19.0476);
        \coordinate (dst4) at (0, -09.5238);
        \coordinate (dst5) at (0, -28.5714);
        \coordinate (dst6) at (0, -04.7619);

        \coordinate (dx) at (11, 0);
        \coordinate (dy) at (0, 2);

        %Draw axis
        \coordinate (py) at (0,30);
        \coordinate (ny) at (0,-30);
        \coordinate (px) at ($7*(dx)$);
        \draw[->,  very thick]   (0,0) --  (px);
        \draw[<->, very thick]   (ny)  --  (py);

        \draw ($(dtu1) + 0*(dx)$) rectangle ($(dst1) + 1*(dx)$);
        \draw ($(dtu2) + 1*(dx)$) rectangle ($(dst2) + 2*(dx)$);
        \draw ($(dtu3) + 2*(dx)$) rectangle ($(dst3) + 3*(dx)$);
        \draw ($(dtu4) + 3*(dx)$) rectangle ($(dst4) + 4*(dx)$);
        \draw ($(dtu5) + 4*(dx)$) rectangle ($(dst5) + 5*(dx)$);
        \draw ($(dtu6) + 5*(dx)$) rectangle ($(dst6) + 6*(dx)$);

        \node at ($(dtu1) + 0*(dx) + 0.5*(dx) + (dy)$) {0.142857};
        \node at ($(dtu2) + 1*(dx) + 0.5*(dx) + (dy)$) {0.285714};
        \node at ($(dtu3) + 2*(dx) + 0.5*(dx) + (dy)$) {0.190476};
        \node at ($(dtu4) + 3*(dx) + 0.5*(dx) + (dy)$) {0.095238};
        \node at ($(dtu5) + 4*(dx) + 0.5*(dx) + (dy)$) {0.238095};
        \node at ($(dtu6) + 5*(dx) + 0.5*(dx) + (dy)$) {0.047619};

        \node at ($(dst1) + 0*(dx) + 0.5*(dx) - (dy)$) {0.142857};
        \node at ($(dst2) + 1*(dx) + 0.5*(dx) - (dy)$) {0.238095};
        \node at ($(dst3) + 2*(dx) + 0.5*(dx) - (dy)$) {0.190476};
        \node at ($(dst4) + 3*(dx) + 0.5*(dx) - (dy)$) {0.095238};
        \node at ($(dst5) + 4*(dx) + 0.5*(dx) - (dy)$) {0.285714};
        \node at ($(dst6) + 5*(dx) + 0.5*(dx) - (dy)$) {0.047619};

        \node at ($0.5*(dx) + (dy)$) {1};
        \node at ($1.5*(dx) + (dy)$) {2};
        \node at ($2.5*(dx) + (dy)$) {3};
        \node at ($3.5*(dx) + (dy)$) {4};
        \node at ($4.5*(dx) + (dy)$) {5};
        \node at ($5.5*(dx) + (dy)$) {6};

        \node at ($(py) + 0.5*(dx)$) {$D_\text{ТУ}$};
        \node at ($(ny) + 0.5*(dx)$) {$D_\text{СТ}$};
        \node at ($(px) + (dy)$) {Номер элемента};
    \end{tikzpicture}
    \caption{Функционально--стоимостная гистограмма}
    \label{pic_function_costs_histogram}
\end{figure}

\begin{equation}
    D_\text{ТУ} =
    \begin{bmatrix}
        0.142857 \\
        0.285714 \\
        0.190476 \\
        0.095238 \\
        0.238095 \\
        0.047619 \\
    \end{bmatrix}
    \label{eq_tu_matrix}
\end{equation}

\begin{equation}
    D_\text{CТ} =
    \begin{bmatrix}
        0.142857 \\
        0.238095 \\
        0.190476 \\
        0.095238 \\
        0.285714 \\
        0.047619 \\
    \end{bmatrix}
    \label{eq_st_matrix}
\end{equation}

\subsubsection{Оценка способов исполнения функций}
Составим функционально~--~стоимостные матрицы:
\begin{enumerate}
    \item $U(i,j)$ --- матрица оценки технического уровня
    \item $C(i,j)$ --- матрица оценки затрат
\end{enumerate}

Следующие далее таблицы составлены попарно аналогично таблицам
(\ref{tbl_technical_lvl_comparison}) и (\ref{tbl_costs_lvl_comparison})
соответственно. В них последний столбец будет соответствовать строке, отвечающей
за технический уровень или затраты по каждому элементу (исполняемой функции).

\begin{enumerate}
    \item Микроконтролер (Табл.
        \ref{tbl_mcu_tech_lvl_comparison},
        \ref{tbl_mcu_cost_lvl_comparison}).

    \item Двигатель (Табл.
        \ref{tbl_motor_tech_lvl_comparison},
        \ref{tbl_motor_cost_lvl_comparison}).

    \item Цифровой датчик угла (Табл.
        \ref{tbl_sensor_tech_lvl_comparison},
        \ref{tbl_sensor_cost_lvl_comparison}).

    \item Драйвер ШИМ (Табл.
        \ref{tbl_drv_tech_lvl_comparison},
        \ref{tbl_drv_cost_lvl_comparison}).

    \item Редуктор (Табл.
        \ref{tbl_red_tech_lvl_comparison},
        \ref{tbl_red_cost_lvl_comparison}).

    \item Операционная система (Табл.
        \ref{tbl_os_tech_lvl_comparison},
        \ref{tbl_os_cost_lvl_comparison}).

\end{enumerate}

\begin{table}[ht!]
    \centering
    \begin{tabular}{|c|c|c|c|c|}
        \hline
        \multirow{2}{2.4cm}[-0.5pc]{
            \centering
            Номер варианта (альтернативы) ($i = \overline{1:n}$)
        } &
        \multicolumn{3}{c|}{
            \parbox[t]{2.4cm}{
                \centering
                Номер варианта (альтернативы) ($j = \overline{1:m}$)
            }
        } &
        \multirow{2}{1.7cm}{
            \centering
            $$1 + \sum_{i=1}^n E_i$$
        } \\
        &
        \centering \rule{2pt}{0pt} 1 \rule{2pt}{0pt} &
        \centering \rule{2pt}{0pt} 2 \rule{2pt}{0pt} &
        \centering \rule{2pt}{0pt} 3 \rule{2pt}{0pt} & \\
        \hline \hline
        \centering{1} &---& 1 & 1 & 3 \\ \hline
        \centering{2} & 0 &---& 1 & 2 \\ \hline
        \centering{3} & 0 & 0 &---& 1 \\ \hline
    \end{tabular}
    \caption{Сравнение альтернатив микроконтролера по техническому уровню}
    \label{tbl_mcu_tech_lvl_comparison}
\end{table}

\begin{table}[ht!]
    \centering
    \begin{tabular}{|c|c|c|c|c|}
        \hline
        \multirow{2}{2.4cm}[-0.5pc]{
            \centering
            Номер варианта (альтернативы) ($i = \overline{1:n}$)
        } &
        \multicolumn{3}{c|}{
            \parbox[t]{2.4cm}{
                \centering
                Номер варианта (альтернативы) ($j = \overline{1:m}$)
            }
        } &
        \multirow{2}{1.7cm}{
            \centering
            $$1 + \sum_{i=1}^n E_i$$
        } \\
        &
        \centering \rule{2pt}{0pt} 1 \rule{2pt}{0pt} &
        \centering \rule{2pt}{0pt} 2 \rule{2pt}{0pt} &
        \centering \rule{2pt}{0pt} 3 \rule{2pt}{0pt} & \\
        \hline \hline
        \centering{1} &---& 1 & 1 & 3 \\ \hline
        \centering{2} & 0 &---& 1 & 2 \\ \hline
        \centering{3} & 0 & 0 &---& 1 \\ \hline
    \end{tabular}
    \caption{Сравнение альтернатив микроконтролера по затратам}
    \label{tbl_mcu_cost_lvl_comparison}
\end{table}

\begin{table}[ht!]
    \centering
    \begin{tabular}{|c|c|c|c|c|}
        \hline
        \multirow{2}{2.4cm}[-0.5pc]{
            \centering
            Номер варианта (альтернативы) ($i = \overline{1:n}$)
        } &
        \multicolumn{3}{c|}{
            \parbox[t]{2.4cm}{
                \centering
                Номер варианта (альтернативы) ($j = \overline{1:m}$)
            }
        } &
        \multirow{2}{1.7cm}{
            \centering
            $$1 + \sum_{i=1}^n E_i$$
        } \\
        &
        \centering \rule{2pt}{0pt} 1 \rule{2pt}{0pt} &
        \centering \rule{2pt}{0pt} 2 \rule{2pt}{0pt} &
        \centering \rule{2pt}{0pt} 3 \rule{2pt}{0pt} & \\
        \hline \hline
        \centering{1} &---& 1 & 1 & 3 \\ \hline
        \centering{2} & 0 &---& 1 & 2 \\ \hline
        \centering{3} & 0 & 0 &---& 1 \\ \hline
    \end{tabular}
    \caption{Сравнение альтернатив двигателя по техническому уровню}
    \label{tbl_motor_tech_lvl_comparison}
\end{table}

\begin{table}[ht!]
    \centering
    \begin{tabular}{|c|c|c|c|c|}
        \hline
        \multirow{2}{2.4cm}[-0.5pc]{
            \centering
            Номер варианта (альтернативы) ($i = \overline{1:n}$)
        } &
        \multicolumn{3}{c|}{
            \parbox[t]{2.4cm}{
                \centering
                Номер варианта (альтернативы) ($j = \overline{1:m}$)
            }
        } &
        \multirow{2}{1.7cm}{
            \centering
            $$1 + \sum_{i=1}^n E_i$$
        } \\
        &
        \centering \rule{2pt}{0pt} 1 \rule{2pt}{0pt} &
        \centering \rule{2pt}{0pt} 2 \rule{2pt}{0pt} &
        \centering \rule{2pt}{0pt} 3 \rule{2pt}{0pt} & \\
        \hline \hline
        \centering{1} &---& 0 & 1 & 2 \\ \hline
        \centering{2} & 1 &---& 1 & 3 \\ \hline
        \centering{3} & 0 & 0 &---& 1 \\ \hline
    \end{tabular}
    \caption{Сравнение альтернатив двигателя по затратам}
    \label{tbl_motor_cost_lvl_comparison}
\end{table}

\begin{table}[ht!]
    \centering
    \begin{tabular}{|c|c|c|c|c|}
        \hline
        \multirow{2}{2.4cm}[-0.5pc]{
            \centering
            Номер варианта (альтернативы) ($i = \overline{1:n}$)
        } &
        \multicolumn{3}{c|}{
            \parbox[t]{2.4cm}{
                \centering
                Номер варианта (альтернативы) ($j = \overline{1:m}$)
            }
        } &
        \multirow{2}{1.7cm}{
            \centering
            $$1 + \sum_{i=1}^n E_i$$
        } \\
        &
        \centering \rule{2pt}{0pt} 1 \rule{2pt}{0pt} &
        \centering \rule{2pt}{0pt} 2 \rule{2pt}{0pt} &
        \centering \rule{2pt}{0pt} 3 \rule{2pt}{0pt} & \\
        \hline \hline
        \centering{1} &---& 1 & 1 & 3 \\ \hline
        \centering{2} & 0 &---& 1 & 2 \\ \hline
        \centering{3} & 0 & 0 &---& 1 \\ \hline
    \end{tabular}
    \caption{Сравнение альтернатив цифрового датчика угла по техническому уровню}
    \label{tbl_sensor_tech_lvl_comparison}
\end{table}

\begin{table}[ht!]
    \centering
    \begin{tabular}{|c|c|c|c|c|}
        \hline
        \multirow{2}{2.4cm}[-0.5pc]{
            \centering
            Номер варианта (альтернативы) ($i = \overline{1:n}$)
        } &
        \multicolumn{3}{c|}{
            \parbox[t]{2.4cm}{
                \centering
                Номер варианта (альтернативы) ($j = \overline{1:m}$)
            }
        } &
        \multirow{2}{1.7cm}{
            \centering
            $$1 + \sum_{i=1}^n E_i$$
        } \\
        &
        \centering \rule{2pt}{0pt} 1 \rule{2pt}{0pt} &
        \centering \rule{2pt}{0pt} 2 \rule{2pt}{0pt} &
        \centering \rule{2pt}{0pt} 3 \rule{2pt}{0pt} & \\
        \hline \hline
        \centering{1} &---& 0 & 1 & 2 \\ \hline
        \centering{2} & 1 &---& 1 & 3 \\ \hline
        \centering{3} & 0 & 0 &---& 1 \\ \hline
    \end{tabular}
    \caption{Сравнение альтернатив цифрового датчика угла по затратам}
    \label{tbl_sensor_cost_lvl_comparison}
\end{table}

\begin{table}[ht!]
    \centering
    \begin{tabular}{|c|c|c|c|c|}
        \hline
        \multirow{2}{2.4cm}[-0.5pc]{
            \centering
            Номер варианта (альтернативы) ($i = \overline{1:n}$)
        } &
        \multicolumn{3}{c|}{
            \parbox[t]{2.4cm}{
                \centering
                Номер варианта (альтернативы) ($j = \overline{1:m}$)
            }
        } &
        \multirow{2}{1.7cm}{
            \centering
            $$1 + \sum_{i=1}^n E_i$$
        } \\
        &
        \centering \rule{2pt}{0pt} 1 \rule{2pt}{0pt} &
        \centering \rule{2pt}{0pt} 2 \rule{2pt}{0pt} &
        \centering \rule{2pt}{0pt} 3 \rule{2pt}{0pt} & \\
        \hline \hline
        \centering{1} &---& 1 & 1 & 3 \\ \hline
        \centering{2} & 0 &---& 1 & 2 \\ \hline
        \centering{3} & 0 & 0 &---& 1 \\ \hline
    \end{tabular}
    \caption{Сравнение альтернатив драйвера ШИМ по техническому уровню}
    \label{tbl_drv_tech_lvl_comparison}
\end{table}

\begin{table}[ht!]
    \centering
    \begin{tabular}{|c|c|c|c|c|}
        \hline
        \multirow{2}{2.4cm}[-0.5pc]{
            \centering
            Номер варианта (альтернативы) ($i = \overline{1:n}$)
        } &
        \multicolumn{3}{c|}{
            \parbox[t]{2.4cm}{
                \centering
                Номер варианта (альтернативы) ($j = \overline{1:m}$)
            }
        } &
        \multirow{2}{1.7cm}{
            \centering
            $$1 + \sum_{i=1}^n E_i$$
        } \\
        &
        \centering \rule{2pt}{0pt} 1 \rule{2pt}{0pt} &
        \centering \rule{2pt}{0pt} 2 \rule{2pt}{0pt} &
        \centering \rule{2pt}{0pt} 3 \rule{2pt}{0pt} & \\
        \hline \hline
        \centering{1} &---& 0 & 1 & 2 \\ \hline
        \centering{2} & 1 &---& 1 & 3 \\ \hline
        \centering{3} & 0 & 0 &---& 1 \\ \hline
    \end{tabular}
    \caption{Сравнение альтернатив драйвера ШИМ по затратам}
    \label{tbl_drv_cost_lvl_comparison}
\end{table}

\begin{table}[ht!]
    \centering
    \begin{tabular}{|c|c|c|c|c|}
        \hline
        \multirow{2}{2.4cm}[-0.5pc]{
            \centering
            Номер варианта (альтернативы) ($i = \overline{1:n}$)
        } &
        \multicolumn{3}{c|}{
            \parbox[t]{2.4cm}{
                \centering
                Номер варианта (альтернативы) ($j = \overline{1:m}$)
            }
        } &
        \multirow{2}{1.7cm}{
            \centering
            $$1 + \sum_{i=1}^n E_i$$
        } \\
        &
        \centering \rule{2pt}{0pt} 1 \rule{2pt}{0pt} &
        \centering \rule{2pt}{0pt} 2 \rule{2pt}{0pt} &
        \centering \rule{2pt}{0pt} 3 \rule{2pt}{0pt} & \\
        \hline \hline
        \centering{1} &---& 1 & 1 & 3 \\ \hline
        \centering{2} & 0 &---& 1 & 2 \\ \hline
        \centering{3} & 0 & 0 &---& 1 \\ \hline
    \end{tabular}
    \caption{Сравнение альтернатив редуктора по техническому уровню}
    \label{tbl_red_tech_lvl_comparison}
\end{table}

\begin{table}[ht!]
    \centering
    \begin{tabular}{|c|c|c|c|c|}
        \hline
        \multirow{2}{2.4cm}[-0.5pc]{
            \centering
            Номер варианта (альтернативы) ($i = \overline{1:n}$)
        } &
        \multicolumn{3}{c|}{
            \parbox[t]{2.4cm}{
                \centering
                Номер варианта (альтернативы) ($j = \overline{1:m}$)
            }
        } &
        \multirow{2}{1.7cm}{
            \centering
            $$1 + \sum_{i=1}^n E_i$$
        } \\
        &
        \centering \rule{2pt}{0pt} 1 \rule{2pt}{0pt} &
        \centering \rule{2pt}{0pt} 2 \rule{2pt}{0pt} &
        \centering \rule{2pt}{0pt} 3 \rule{2pt}{0pt} & \\
        \hline \hline
        \centering{1} &---& 1 & 1 & 3 \\ \hline
        \centering{2} & 0 &---& 1 & 2 \\ \hline
        \centering{3} & 0 & 0 &---& 1 \\ \hline
    \end{tabular}
    \caption{Сравнение альтернатив редуктора по затратам}
    \label{tbl_red_cost_lvl_comparison}
\end{table}

\begin{table}[ht!]
    \centering
    \begin{tabular}{|c|c|c|c|c|}
        \hline
        \multirow{2}{2.4cm}[-0.5pc]{
            \centering
            Номер варианта (альтернативы) ($i = \overline{1:n}$)
        } &
        \multicolumn{3}{c|}{
            \parbox[t]{2.4cm}{
                \centering
                Номер варианта (альтернативы) ($j = \overline{1:m}$)
            }
        } &
        \multirow{2}{1.7cm}{
            \centering
            $$1 + \sum_{i=1}^n E_i$$
        } \\
        &
        \centering \rule{2pt}{0pt} 1 \rule{2pt}{0pt} &
        \centering \rule{2pt}{0pt} 2 \rule{2pt}{0pt} &
        \centering \rule{2pt}{0pt} 3 \rule{2pt}{0pt} & \\
        \hline \hline
        \centering{1} &---& 0 & 0 & 1 \\ \hline
        \centering{2} & 1 &---& 0 & 2 \\ \hline
        \centering{3} & 1 & 1 &---& 3 \\ \hline
    \end{tabular}
    \caption{Сравнение альтернатив ОС по техническому уровню}
    \label{tbl_os_tech_lvl_comparison}
\end{table}

\begin{table}[ht!]
    \centering
    \begin{tabular}{|c|c|c|c|c|}
        \hline
        \multirow{2}{2.4cm}[-0.5pc]{
            \centering
            Номер варианта (альтернативы) ($i = \overline{1:n}$)
        } &
        \multicolumn{3}{c|}{
            \parbox[t]{2.4cm}{
                \centering
                Номер варианта (альтернативы) ($j = \overline{1:m}$)
            }
        } &
        \multirow{2}{1.7cm}{
            \centering
            $$1 + \sum_{i=1}^n E_i$$
        } \\
        &
        \centering \rule{2pt}{0pt} 1 \rule{2pt}{0pt} &
        \centering \rule{2pt}{0pt} 2 \rule{2pt}{0pt} &
        \centering \rule{2pt}{0pt} 3 \rule{2pt}{0pt} & \\
        \hline \hline
        \centering{1} &---& 1 & 1 & 3 \\ \hline
        \centering{2} & 0 &---& 1 & 2 \\ \hline
        \centering{3} & 0 & 0 &---& 1 \\ \hline
    \end{tabular}
    \caption{Сравнение альтернатив ОС по затратам}
    \label{tbl_os_cost_lvl_comparison}
\end{table}


\paragraph{Матрица оценки технического уровня}

На основании таблиц (
    \ref{tbl_mcu_tech_lvl_comparison}
  , \ref{tbl_motor_tech_lvl_comparison}
  , \ref{tbl_sensor_tech_lvl_comparison}
  , \ref{tbl_drv_tech_lvl_comparison}
  , \ref{tbl_red_tech_lvl_comparison}
  , \ref{tbl_os_tech_lvl_comparison}
)
формируем матрицу оценки технического уровня
(\ref{eq_matrix_tech_lvl_estimate}):

\begin{equation}
    U(6,3) =
    \begin{bmatrix}
        3 & 2 & 1 \\
        3 & 2 & 1 \\
        3 & 2 & 1 \\
        3 & 2 & 1 \\
        3 & 2 & 1 \\
        1 & 2 & 3 \\
    \end{bmatrix}
    \label{eq_matrix_tech_lvl_estimate}
\end{equation}

\paragraph{Матрица оценки затрат}

На основании таблиц (
    \ref{tbl_mcu_cost_lvl_comparison}
  , \ref{tbl_motor_cost_lvl_comparison}
  , \ref{tbl_sensor_cost_lvl_comparison}
  , \ref{tbl_drv_cost_lvl_comparison}
  , \ref{tbl_red_cost_lvl_comparison}
  , \ref{tbl_os_cost_lvl_comparison}
)
формируем матрицу оценки затрат
(\ref{eq_matrix_cost_lvl_estimate}):

\begin{equation}
    C(6,3) =
    \begin{bmatrix}
        3 & 2 & 1 \\
        2 & 3 & 1 \\
        2 & 3 & 1 \\
        2 & 3 & 1 \\
        3 & 2 & 1 \\
        3 & 2 & 1 \\
    \end{bmatrix}
    \label{eq_matrix_cost_lvl_estimate}
\end{equation}

\subsubsection{Формирование и оценка лучшего варианта изделия}
Взвешенную матрицу технического уровня $UR(i, j)$, по формуле
(\ref{eq_ur_computing_form}).

\begin{equation}
    UR(6,3) =
        \begin{bmatrix}
            D_\text{ТУ}[1] & 0 & 0 & 0 & 0 & 0 \\
            0 & D_\text{ТУ}[2] & 0 & 0 & 0 & 0 \\
            0 & 0 & D_\text{ТУ}[3] & 0 & 0 & 0 \\
            0 & 0 & 0 & D_\text{ТУ}[4] & 0 & 0 \\
            0 & 0 & 0 & 0 & D_\text{ТУ}[5] & 0 \\
            0 & 0 & 0 & 0 & 0 & D_\text{ТУ}[6] \\
        \end{bmatrix}
        \cdot
        U(6,3)
    \label{eq_ur_computing_form}
\end{equation}

$$
    UR(6,3) =
        \begin{bmatrix}
            0.142857 & 0.0      & 0.0      & 0.0      & 0.0      & 0.0      \\
            0.0      & 0.285714 & 0.0      & 0.0      & 0.0      & 0.0      \\
            0.0      & 0.0      & 0.190476 & 0.0      & 0.0      & 0.0      \\
            0.0      & 0.0      & 0.0      & 0.095238 & 0.0      & 0.0      \\
            0.0      & 0.0      & 0.0      & 0.0      & 0.238095 & 0.0      \\
            0.0      & 0.0      & 0.0      & 0.0      & 0.0      & 0.047619 \\
        \end{bmatrix}
        \cdot
        \begin{bmatrix}
            3 & 2 & 1 \\
            3 & 2 & 1 \\
            3 & 2 & 1 \\
            3 & 2 & 1 \\
            3 & 2 & 1 \\
            1 & 2 & 3 \\
        \end{bmatrix}
$$

\begin{equation}
    UR(6,3) =
        \begin{bmatrix}
            0.428571 & 0.285714 & 0.142857 \\
            0.857142 & 0.571428 & 0.285714 \\
            0.571428 & 0.380952 & 0.190476 \\
            0.285714 & 0.190476 & 0.095238 \\
            0.714285 & 0.47619  & 0.238095 \\
            0.047619 & 0.095238 & 0.142857 \\
        \end{bmatrix}
    \label{eq_ur_rez}
\end{equation}

Взвешенную матрицу технического уровня $CD(i, j)$, по формуле
(\ref{eq_cd_computing_form}).

\begin{equation}
    UR(6,3) =
        \begin{bmatrix}
            D_\text{СТ}[1] & 0 & 0 & 0 & 0 & 0 \\
            0 & D_\text{СТ}[2] & 0 & 0 & 0 & 0 \\
            0 & 0 & D_\text{СТ}[3] & 0 & 0 & 0 \\
            0 & 0 & 0 & D_\text{СТ}[4] & 0 & 0 \\
            0 & 0 & 0 & 0 & D_\text{СТ}[5] & 0 \\
            0 & 0 & 0 & 0 & 0 & D_\text{СТ}[6] \\
        \end{bmatrix}
        \cdot
        C(6,3)
    \label{eq_cd_computing_form}
\end{equation}

$$
    UR(6,3) =
        \begin{bmatrix}
            0.142857 & 0.0      & 0.0      & 0.0      & 0.0      & 0.0      \\
            0.0      & 0.238095 & 0.0      & 0.0      & 0.0      & 0.0      \\
            0.0      & 0.0      & 0.190476 & 0.0      & 0.0      & 0.0      \\
            0.0      & 0.0      & 0.0      & 0.095238 & 0.0      & 0.0      \\
            0.0      & 0.0      & 0.0      & 0.0      & 0.285714 & 0.0      \\
            0.0      & 0.0      & 0.0      & 0.0      & 0.0      & 0.047619 \\
        \end{bmatrix}
        \cdot
        \begin{bmatrix}
            3 & 2 & 1 \\
            2 & 3 & 1 \\
            2 & 3 & 1 \\
            2 & 3 & 1 \\
            3 & 2 & 1 \\
            3 & 2 & 1 \\
        \end{bmatrix}
$$
\begin{equation}
        UR(6,3) =
        \begin{bmatrix}
            0.428571 & 0.285714 & 0.142857 \\
            0.47619  & 0.714285 & 0.238095 \\
            0.380952 & 0.571428 & 0.190476 \\
            0.190476 & 0.285714 & 0.095238 \\
            0.857142 & 0.571428 & 0.285714 \\
            0.142857 & 0.095238 & 0.047619 \\
        \end{bmatrix}
    \label{eq_cd_rez}
\end{equation}

В качестве критерия для отбора лучшего варианта используем
критерий из \cite[стр. 21]{econ_FSA}:

\begin{equation}
    K = \frac{UR}{CD^2}
    \label{eq_econ_best_var_form}
\end{equation}

Согласно формуле (\ref{eq_econ_best_var_form}):
\begin{equation}
    K =
        \begin{bmatrix}
            2.333335 & 3.500003 & \textbf{7.0000070} \\
            3.780003 & 1.120001 & \textbf{5.0400050} \\
            3.937503 & 1.166667 & \textbf{5.2500052} \\
            7.875007 & 2.333335 & \textbf{10.500010} \\
            0.972223 & 1.458334 & \textbf{2.9166695} \\
            2.333335 & 10.50001 & \textbf{63.000063} \\
        \end{bmatrix}
    \label{eq_econ_best_var}
\end{equation}

Эта величина характеризует качество или технический уровень
элемента, приходящийся на единицу затрат. Знаменатель возведён в квадрат,
чтобы разделить варианты, которые при разных величинах UR и CD имеют
одинаковое значение K.

\newpage
\subsection{Выбор лучшего варианта}
В каждой строке полученной матрицы $K(i, j)$ находим элементы,
имеющие максимальное значение, соответствующее лучшему варианту
исполнения функции.
В результате проведённого функционально~--~стоимостного анализа
вариантов конструкции системы управления приводами двухстепенного манипулятора
спутника видеонаблюдения земной поверхности были определены оптимальные
реализации функцианальных, наиболее значимых частей системы
(Табл. \ref{tbl_econ_result}).

\begin{table}[ht!]
    \centering
    \begin{tabular}{|l|l|}
        \hline
        Функция & Выбранная альтернатива \\
        \hline
        \hline
        Микроконтролер & \textit{TMS320F28035 32bit} \\
        \hline
        Двигатель & Шаговый \\
        \hline
        Цифровой датчик угла & Оптический, цифровой \\
        \hline
        Драйвер ШИМ & \textit{TI DRV8412} \\
        \hline
        Редуктор & Редуктор с прямозубыми ЗК \\
        \hline
        Операционная система & без ОС \\
        \hline
    \end{tabular}
    \caption{Выбранные с помощью ФСА реализации основных функций}
    \label{tbl_econ_result}
\end{table}

