\newpage
\subsection{Технология сборки экспериментального стенда}
\subsubsection{Анализ особенностей конструкции}

Стенд выполнен ввиде 4--х пластин, на которых закреплены его ключевые элементы -
двигатель, датчик, подшипниковый узел. Пластины соединяются друг с другом с помощью
4--х соединительных стоек, для обеспечения лёгкости собираемости посадка стойки в
пластину - с зазором.

Пластины и соединительные стойки выполнены таким образом, чтобы их соединение
не требовало дополнительных элементов, таких как штифты и т.п.

Для удовлетворения этого условия, а так же для избежания необходимости
переустановов заготовки при изготовлении деталей, пластины и соединительные стойки
сделаны несимметричными относительно продольной оси, но это позволило изготовить
пластины за один установ на вертикально--фрезерном станке с ЧПУ, а соединительные
стойки - за один установ на токарно--револьном станке.

Соединительные стойки обладают лыской под ключ для избежания прокручивания стоек
в посадочном отверстии пластин при их соединении.

Вал--нагрузка, устанавливаемый в подишипниковый узел, является полной имитацией
нагрузки привода, приведённой к валу двигателя.

Для компенсации отклонений от параллельности осей валов датчика, двигателя и
вала--нагрузки используются соединительные муфты--сильфоны.

Конструкция вала--нагрузки предусматривает небольшие возможности по увеличению
момента инерции нагрузки.

Пластины образуют массивное основание, препятствующее опрокидыванию стенда
при реверсивных движениях и переброске при максимальных скоростях.

В качестве крепёжных элементов используются только стандартные элементы.

\subsubsection{Разработка маршрутной технологии сборки}

Технология сбоки экспериментального стенда проектировалась таким образом,
чтобы при сборке не требовались специальные рабочие и измерительные приспособления,
процесс сборки не был затруднён использованием штифтов и т.п.
