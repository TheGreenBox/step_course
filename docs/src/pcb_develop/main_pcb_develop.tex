\documentclass{article}

\usepackage[utf8]{inputenc}
% Выбор внутренней TEX−кодировки
\usepackage [T2A]{fontenc}
% Включение переносов для русского и английского языков
\usepackage[english,russian]{babel}

% Подключение гиперссылок
\usepackage{xcolor}
\usepackage[unicode]{hyperref}

% Настройка внешнего вида гиперссылок
\definecolor{LINKCOLOUR}{rgb}{0.1,0.0,0.9}
\hypersetup{colorlinks,breaklinks,urlcolor=LINKCOLOUR,linkcolor=LINKCOLOUR}

% Начинать первый параграф раздела следует с красной строки
\usepackage{indentfirst}

% Дополнительные математические пакеты
%\usepackage{weird,querr}
\usepackage{amssymb}
\usepackage{amsmath}

% Для корректного копирования из документа
\usepackage{cmap}

% Кто нибудь помнит зачем это тут? A: Неа...
% это многострочный текст, мы используем для многострочности в ячейках таблицы
\usepackage{multirow}

% стилевой пакет, открывающий доступ к большому числу типографских значков
\usepackage{textcomp}

% Меняем поля страницы
\usepackage{geometry}
\geometry{left=2cm}    % левое поле
\geometry{right=1.5cm} % правое поле
\geometry{top=1.5cm}     % верхнее поле
\geometry{bottom=1.5cm}  % нижнее поле

\usepackage{subfiles}

% работа с импортом изображений
\ifx\pdfoutput\undefined
\usepackage{graphicx}
\else
\usepackage[pdftex]{graphicx}
\fi

% секции и их структура
\usepackage[section]{placeins}
\usepackage{subcaption}

\begin{document}
\tableofcontents
\newpage
\section{Разработка платы управления}
\begin{flushright}
    \itshape
    There is only the Emperor, and he is our shield and protector.\\
    \ldots
\end{flushright}
\subsection{Перечень требований}
\subsection{Выбор компонентов}
\paragraph{Шины питания}
Все микросхемы можно разделить на три группы по уровню питающего напряжения:
\begin{itemize}
    \item \textbf{+12В}:
        \begin{itemize}
            \item Драйвер ШИМ.
        \end{itemize}
        Исполльзовали линейный регулятор напряжения, \textit{TPS54160DGQ Texas Instruments}
    \item \textbf{+5В}:
        \begin{itemize}
            \item Энкодер \textit{ЛИР-137А 6250-05-ПИ}
            \item Операционный усилитель OPA2350
            \item Преобразователь сигнала энекодера \textit{TXB0106PW}
            \item Интерфейс \textit{RS-422/485 SN75176B}
        \end{itemize}
        Исполльзовали линейный регулятор напряжения \textit{UA78M05CDCY Texas Instruments}
    \item \textbf{+3.3В}:
        \begin{itemize}
            \item Преобразователь сигнала энекодера \textit{TXB0106PW}
            \item Микроконтролер \textit{TMS320F28035}
        \end{itemize}
        Исполльзовали линейный регулятор напряжения, \textit{UA78M33CDCY Texas Instruments}
\end{itemize}


\paragraph{Cигнал энекодера}
Выбранный энкодер имеет TTL логику, а микроконтролер имеет логику 0..+3.3.
Для этого поставили 6-разрядный двунаправленный преобразователь нарпяжения,
OPA2350 Texas Instruments.

\paragraph{Cвязь с бортовой шиной данных}
Согласно тезническому заданию плата должна управляться со стороны бортовой
\textit{ЭВМ} посредством интерфейса \textit{RS-485}, в полудуплексном режиме.
В кабеле используется экранированная витая пара.
Baud rate 115200. Выбран интерфейс \textit{RS-422/485 SN75176B} от Texas Instruments.

%   TODO: ПЕРЕНЕСТИ это остюда в ТЗ !!!
%   В качетсве протокола передачи данных использовали модифицированны STX-ETX протокол.
%   Сообщение состояит из:
%
%   \begin{tabular}{|c|c|l|}
%   \hline
%   Тип & Длина, байт & Значение \\
%   \hline
%   STX    & 1      & 0x02 \\
%   \hline
%   Длина  & 1      & 0x04..0xff \\
%   \hline
%   Адрec  & 1      & 0x00..0x7f \\
%   \hline
%   Строка & 0..251 & 0x00..0x7f \\
%   \hline
%   ETX    & 1      & 0x03 \\
%   \hline
%   \end{tabular}

\paragraph{Индикация состояния}
В числе прочего для целей отладки, тестирования и последущих процедур контроля
необхрдимо иметь визуальное средство для котроля состояния платы.
Перечень ``жизненно важных'' систем платы включает:
шину бортового питания,
шину питаня \textit{12В},
шину питаня \textit{5В},
шину питаня \textit{3.3В},
бортовая шина передачи информации \textit{RS-485},
микроконтролер.

Для этого на плату установлены:
\begin{enumerate}
    \item Индикатор статуса основной программы микроконтролера, светодиод зеленый \textit{KP-1608MGC}.
    \item Индикатор статуса бртовой сети связи, светодиод голубой \textit{KPH-1608PBC-A}.
    \item Индикатор критических ошибок микроконтролера, светодиод красный \textit{KPH-1608SEC}.
    \item Индикатор напряжения на шине бортового питания, светодиод белый \textit{KPT-1608QWF-E}.
    \item Индикатор напряжения на шине питания \textit{+12В}, светодиод желтый \textit{KPT-1608YD}.
    \item Индикатор напряжения на шине питания \textit{+5В}, светодиод зеленый \textit{KP-1608VGC(A)}.
    \item Индикатор напряжения на шине питания \textit{+3.3В}, светодиод оранжевый \textit{KP-1608VS}.
\end{enumerate}

\subsection{Технологические параметры платы}
\paragraph{Выбор материала заготовки}
В качестве диэлектрика выбрали стеклотекстолит \textit{СТЕФ-У} ГОСТ 12652-74 \cite{GOST_12652_74}.
Длительная рабочая температура от \textit{-65\textcelsius} до \textit{+155\textcelsius}.
Предназначен для работы в условиях нормальной относительной влажности
окружающей среды при напряжении свыше \textit{1000В}.

\paragraph{Выбор класса точности платы}
Для платы выбран класc точности 3.
Печатные платы 3-гo класса - наиболее распространенные, поскольку, с одной
стороны, обеспечивают достаточно высокую плотность трассировки и монтажа, а с
другой — для их производства требуется рядовое, хотя и специализированное,
оборудование.

\paragraph{Расстояние между элементами проводящего рисунка}
Согласно \cite[Табл. 7]{GOST_23751_86} для 3 класса точности,
фольгированного стеклотекстолита:

\begin{tabular}{|c|l|}
    \hline
    Мин. расстояние, мм & Проводники \\
    \hline
    0.4 & Силовые линии - шины питания \textit{VCC} \\
    \hline
    0.2 & Остальные линии, включая внутренние шины питания \\
    \hline
\end{tabular}

\paragraph{Толщина печатного проводника}
Согласно \cite[Табл. 9]{GOST_23751_86} для 3 класса точности,
фольгированного стеклотекстолита и проводников с покрытием:

\begin{tabular}{|c|l|}
    \hline
    Мин. ширина, мм & Проводники \\
    \hline
    0.4 & Силовые линии - шины питания \textit{VCC} \\
    \hline
    0.2 & Остальные линии, включая внутренние шиниы питания \\
    \hline
\end{tabular}

\paragraph{Размеры переходных металлизированых отверстий}
Все переходные метализированые отверстия на плате, для не монтажных целей:

\begin{itemize}
    \item Диаметр площадки на фольге под отверстия --- \textit{1.2мм.}
    \item Диаметр металлизированного отверстия --- \textit{0.7мм.}
    \item Диаметр сверления --- \textit{0.8мм.}
\end{itemize}

\section{Сборка платы управления}
В качестве половины технологической части дипломного проекта рассмотрим
процесс сборки печатной платы.

\subsection{Анализ особенностей конструкции платы}
Плата имеет размеры 70x120 мм. с размещёнными на ней электрорадиоэлементами
преобладающее большинство из которых планарного монтажа.

Особенностями конструкции, существенными с точки зрения технологического
процесса сборки платы блока управления, являются:

\begin{itemize}
    \item Односторонняя печатная плата
    \item Высокая повторяемость типоразмеров ЭРЭ
    \item Коммутация с другими элементами конструкции осуществляется c помощью
    кабелей, подключаемых через разъемы
    \item ЭРЭ располагаются с одной стороны печатной платы
    \item В числе ЭРЭ есть не только \textit{SMD} компоненты,
    пайти оплавлением будет недостаточно
\end{itemize}

\subsection{Оборудование для сборки}
Преобладющее большинство компонентов планарного монтажа, поэтому основной способ
сбоки платы - это пайка оплавлением. Для этого была выбрана паяльная паста
на безсвинцовом припое с флюсом не требующим отмывки \textit{Indium NC-SMQ 92J}.
Так же имется небольшое количество не планарных компонентов, которые паяются
волной припоя. А устанавливаются перед этим вручную на плату.
Перед нанесение пасты плату необходимо промыть от технологических загрязнений.

\begin{itemize}
    \item Установка для отмывки печатных плат \textit{Радуга-60.2}
    \item Автоматический трафаретный принтер \textit{P500}
    \item Установщик компонентов \textit{SMT2000 IVAS tech}
    \item Конвекционная конвейерная печь \textit{BM745}
    \item Установка пайки двойной волной припоя \textit{WS-400F}
\end{itemize}

\subsection{Оценка технологичности конструкции}
На плате расположено много разнородных элементов, некоторые из которых
устанавливаются в единственном экземпляре, что значительно снижает технологичность
монтажа. Однако при разработке учитывались некоторые факторы технологичности, и
поэтому применяется всего двухслойная печатная плата с установкой элементов с
планарными выводами (за исключением электролитических конденсаторов и разъемов),
Плата разведена с шагом координатной сетки 0.5 мм.
Установленные на плату компоненты и количесвто выводов им соотвествующее,
сгруппированные по функциональному назначению перечислены в таблице
(\ref{components_and_mount_points}).

\begin{table}
    \centering
    \begin{tabular}{|l|c|c|}
        \hline
        Название & Количество, шт & Общее количество выводов, шт \\ \hline
        Микросхемы & 9 & 180 \\ \hline
        Чип резисторы & 64 & 128 \\ \hline
        Чип конденсаторы & 51 & 102 \\ \hline
        Электролитические конденсаторы & 6 & 12 \\ \hline
        Чип светодиоды & 7 & 14 \\ \hline
        Разъемы & 6 & 33 \\ \hline
        Чип индуктивность & 1 & 2 \\ \hline
        Чип стабилитрон & 1 & 2 \\ \hline
    \end{tabular}
    \caption{Установленные на плату компоненты по количеству выводов}
    \label{components_and_mount_points}
\end{table}

Исходные данные для определения комплексного показателя технологичности
приведены в таблице (\ref{technological_estaime_parametres}).

\begin{table}
    \centering
    \begin{tabular}{|p{9cm}|p{3cm}|p{3cm}|}
        \hline
        \begin{center} Исходные данные \end{center}
        & \begin{center} Обозначения \end{center}
        & \begin{center} Значение показателя\end{center} \\
        \hline
        Количество монтажных соединений, которые могут осуществляться
        механизированным или автоматизированным способом,
        т.е. имеются механизмы, оборудование или оснащение для
        выполнения монтажных соединений.
        & \begin{center} $H_{\text{ам}}$\end{center}
        % все кроме тех, что устанавливаем вручную, электролит. конд. и разъемы
        & \begin{center} 473 \end{center} \\
        \hline
        Общее количество монтажных соединений
        & \begin{center} $H_{\text{м}}$\end{center}
        % полное количество ножек
        & \begin{center} 473 \end{center} \\
        \hline
        Общее количество микросхем в блоке
        & \begin{center} $H_{\text{мс}}$\end{center}
        & \begin{center} 9 \end{center} \\
        \hline
        Общее количество ЭРЭ
        & \begin{center} $H_{\text{ЭРЭ}}$\end{center}
        & \begin{center} 136 \end{center} \\
        \hline
        Количество ЭРЭ, подготовка которых к монтажу осуществляется (может
        осуществляться) механизированным или автоматическим способом,
        т.е. имеются механизмы, оборудование или оснащение для выполнения
        монтажных соединений.
        & \begin{center} $H_{\text{мп ЭРЭ}}$ \end{center}
        % все кроме тех, что устанавливаем вручную, электр. конд. и разъемы
        & \begin{center} 123 \end{center} \\
        \hline
        Количество операций контроля и настройки, которые можно осуществить
        автоматизированным или механизированным способом.
        & \begin{center} $H_{\text{мкн}}$ \end{center}
        & \begin{center} 0 \end{center} \\
        \hline
        Общее количество операций контроля и настройки
        & \begin{center} $H_{\text{кн}}$ \end{center}
        & \begin{center} 2 \end{center} \\
        \hline
        Общее количество типоразмеров ЭРЭ в изделии
        & \begin{center} $H_{\text{т ЭРЭ}}$ \end{center}
        & \begin{center} 22 \end{center} \\
        \hline
        Количество типоразмеров оригинальных ЭРЭ в изделии
        & \begin{center} $H_{\text{тор ЭРЭ}}$ \end{center}
        & \begin{center} 0 \end{center} \\
        \hline
    \end{tabular}
    \caption{Показатели технологичности}
    \label{technological_estaime_parametres}
\end{table}

Коэффициент использования микросхем и транзисторных матриц в блоке.
Весовой коэффициент $\phi = 1$.
$$
K_{\text{испмс}}
    = \frac{H_\text{мс}}
           {H_\text{ам}}
    = \frac{9}{473}
    = 0.019027
$$

Коэффициент автоматизации и механизации монтажа.
Весовой коэффициент $\phi = 1$.
$$
K_{\text{гсм}}
    = \frac{H_\text{ам}}
           {H_\text{м}}
    = \frac{473}{473}
    = 1
$$

Коэффициент механизации подготовки ЭРЭ.
Весовой коэффициент $\phi = 0.75$.
$$
K_{\text{мс}}
    = \frac{H_\text{мп ЭРЭ}}
           {H_\text{ЭРЭ}}
    = \frac{123}{136}
    = 0.904412
$$

Коэффициент механизации контроля и настройки.
Весовой коэффициент $\phi = 0.5$.
$$
K_\text{кмн}
    = \frac{H_\text{мкн}}
           {H_\text{кн}}
    = \frac{0}{2}
    = 0
$$

Коэффициент повторяемости ЭРЭ.
Весовой коэффициент $\phi = 0.31$.
$$
K_\text{повЭРЭ}
    = 1 - \frac{H_\text{т ЭРЭ}}
               {H_\text{ЭРЭ}}
    = 1 - \frac{22}{136}
    = 0.838235
$$

Коэффициент применяемости ЭРЭ.
Весовой коэффициент $\phi = 0.187$.
$$
H_{\text{МКН}}
    = 1 - \frac{H_\text{тор ЭРЭ}}
               {H_\text{т ЭРЭ}}
    = 1 - \frac{0}{22}
    = 1
$$

Определение комплексного показателя технологичности.
$$
K   = \frac{\sum_{i=1}^6 K_i \cdot \phi_i }
         {\sum_{i=1}^6 \phi_i }
    = \frac{ 1 \cdot 0.019027
             + 1 \cdot 1
             + 0.75 \cdot 0.904412
             + 0.5 \cdot 0
             + 0.31 \cdot 0.838235
             + 0.187 \cdot 1 }
           { 1
             + 1
             + 0.75
             + 0.5
             + 0.31
             + 0.187 }
    = 0.572241
$$

Норматив комплексных показателей устанавливает ОСТ4 ГО.091.219 \cite{OST4_GO_010_011}.
В соответствии с данным ОСТ комплексный показатель технологичности электронных
блоков автоматизированных систем управления и электронно-вычислительной техники,
выпускаемой серийно, находится в пределах:

\begin{itemize}
    \item установочная серия --- К = 0.25...0.5
    \item установившееся серийное производство --- К = 0.3...0.6
\end{itemize}

Таким образом, изделия отвечают требованиям технологичности в условиях
серийного производства.

\newpage
\section[Список использованной литературы]{}
\begin{thebibliography}{30}
    \subfile{src/pcb_develop/bibliography}
\end{thebibliography}

\end{document}
