\documentclass{article}

\usepackage[utf8]{inputenc}
% Выбор внутренней TEX−кодировки
\usepackage [T2A]{fontenc}
% Включение переносов для русского и английского языков
\usepackage[english,russian]{babel}

% Подключение гиперссылок
\usepackage{xcolor}
\usepackage[unicode]{hyperref}

% Настройка внешнего вида гиперссылок
\definecolor{LINKCOLOUR}{rgb}{0.1,0.0,0.9}
\hypersetup{colorlinks,breaklinks,urlcolor=LINKCOLOUR,linkcolor=LINKCOLOUR}

% Начинать первый параграф раздела следует с красной строки
\usepackage{indentfirst}

% Дополнительные математические пакеты
%\usepackage{weird,querr}
\usepackage{amssymb}
\usepackage{amsmath}

% Для корректного копирования из документа
\usepackage{cmap}

% Кто нибудь помнит зачем это тут? A: Неа...
% это многострочный текст, мы используем для многострочности в ячейках таблицы
\usepackage{multirow}

% стилевой пакет, открывающий доступ к большому числу типографских значков
\usepackage{textcomp}

% Меняем поля страницы
\usepackage{geometry}
\geometry{left=2cm}    % левое поле
\geometry{right=1.5cm} % правое поле
\geometry{top=1.5cm}     % верхнее поле
\geometry{bottom=1.5cm}  % нижнее поле

\usepackage{subfiles}

% работа с импортом изображений
\ifx\pdfoutput\undefined
\usepackage{graphicx}
\else
\usepackage[pdftex]{graphicx}
\fi

% секции и их структура
\usepackage[section]{placeins}
\usepackage{subcaption}

\begin{document}
\section{Разработка платы управления}
\begin{flushright}
    \itshape
    There is only the Emperor, and he is our shield and protector.\\
    \ldots
\end{flushright}
\subsection{Перечень требований}
\subsection{Выбор компонентов}
\paragraph{Шины питания}
Все микросхемы можно разделить на три группы по уровню питающего напряжения:
\begin{itemize}
    \item \textbf{+12В}:
        \begin{itemize}
            \item Драйвер ШИМ.
        \end{itemize}
        Исполльзовали линейный регулятор напряжения, \textit{TPS54160DGQ Texas Instruments}
    \item \textbf{+5В}:
        \begin{itemize}
            \item Энкодер \textit{ЛИР-137А 6250-05-ПИ}
            \item Операционный усилитель OPA2350
            \item Преобразователь сигнала энекодера \textit{TXB0106PW}
            \item Интерфейс \textit{RS-422/485 SN75176B}
        \end{itemize}
        Исполльзовали линейный регулятор напряжения \textit{UA78M05CDCY Texas Instruments}
    \item \textbf{+3.3В}:
        \begin{itemize}
            \item Преобразователь сигнала энекодера \textit{TXB0106PW}
            \item Микроконтролер \textit{TMS320F28035}
        \end{itemize}
        Исполльзовали линейный регулятор напряжения, \textit{UA78M33CDCY Texas Instruments}
\end{itemize}

%    \item Операционный усилитель, DA6, DA5, OPA2350 Texas Instruments


\paragraph{Cигнал энекодера}
Выбранный энкодер имеет TTL логику, а микроконтролер имеет логику 0..+3.3.
Для этого поставили 6-разрядный двунаправленный преобразователь нарпяжения, OPA2350 Texas Instruments.

\paragraph{Cвязь с бортовой шиной данных}
Согласно тезническому заданию плата должна управляться со стороны бортовой \textit{ЭВМ} посредством
интерфейса \textit{RS-485}, в полудуплексном режиме. В кабеле используется экранированная витая пара.
Baud rate 115200. Выбран интерфейс \textit{RS-422/485 SN75176B} от Texas Instruments.

%   TODO: ПЕРЕНЕСТИ это остюда в ТЗ !!!
%   В качетсве протокола передачи данных использовали модифицированны STX-ETX протокол.
%   Сообщение состояит из:
%
%   \begin{tabular}{|c|c|l|}
%   \hline
%   Тип & Длина, байт & Значение \\
%   \hline
%   STX    & 1      & 0x02 \\
%   \hline
%   Длина  & 1      & 0x04..0xff \\
%   \hline
%   Адрec  & 1      & 0x00..0x7f \\
%   \hline
%   Строка & 0..251 & 0x00..0x7f \\
%   \hline
%   ETX    & 1      & 0x03 \\
%   \hline
%   \end{tabular}

\paragraph{Индикация состояния}
В числе прочего для целей отладки, тестирования и последущих процедур контроля
необхрдимо иметь визуальное средство для котроля состояния платы.
Перечень ``жизненно важных'' систем платы включает:
шину бортового питания,
шину питаня 12В,
шину питаня 5В,
шину питаня 3.3В,
бортовая шина передачи информации RS485,
микроконтролер.

Для этого на плату установлены:
\begin{enumerate}
    \item Индикатор статуса основной программы микроконтролера, светодиод зеленый \textit{KP-1608MGC}.
    \item Индикатор статуса бртовой сети связи, светодиод голубой \textit{KPH-1608PBC-A}.
    \item Индикатор критических ошибок микроконтролера, светодиод красный \textit{KPH-1608SEC}.
    \item Индикатор напряжения на шине бортового питания, светодиод белый \textit{KPT-1608QWF-E}.
    \item Индикатор напряжения на шине питания \textit{+12В}, светодиод желтый \textit{KPT-1608YD}.
    \item Индикатор напряжения на шине питания \textit{+5В}, светодиод зеленый \textit{KP-1608VGC(A)}.
    \item Индикатор напряжения на шине питания \textit{+3.3В}, светодиод оранжевый \textit{KP-1608VS}.
\end{enumerate}

\subsection{Технологические параметры платы}
\paragraph{Выбор материала заготовки}
В качестве диэлектрика выбрали стеклотекстолит СТЕФ-У.
Длительная рабочая температура от \textit{-65\textcelcius} до \textit{+155\textcelcius}.
Предназначен для работы в условиях нормальной относительной влажности
окружающей среды при напряжении свыше \textit{1000В}.

\paragraph{Выбор класса точности платы}


\newpage
\section[Список использованной литературы]{}
\begin{thebibliography}{30}
    \subfile{src/pcb_develop/bibliography}
\end{thebibliography}
\end{document}
