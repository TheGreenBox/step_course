\newpage
\subsection{Разработка платы управления}
\subsubsection{Перечень требований}
\subsubsection{Реализация подсистем}
\paragraph{Вычислительное устройство платы управления}
\paragraph{Усилитель мощности}

\paragraph{Cвязь с бортовой шиной данных}
Согласно тезническому заданию плата должна управляться со стороны бортовой
\textit{ЭВМ} посредством интерфейса \textit{RS-485}, в полудуплексном режиме.
В кабеле используется экранированная витая пара.
Baud rate 115200. Выбран интерфейс \textit{RS-422/485 SN75176B} от Texas Instruments.
% TODO: ПЕРЕНЕСТИ это остюда в ТЗ !!!
В качетсве протокола передачи данных использовали модифицированны STX-ETX протокол.
Сообщение состояит из:

\begin{tabular}{|c|c|l|}
   \hline
   Тип & Длина, байт & Значение \\
   \hline
   STX    & 1      & 0x02 \\
   \hline
   Длина  & 1      & 0x04..0xff \\
   \hline
   Адрec  & 1      & 0x00..0x7f \\
   \hline
   Строка & 0..251 & 0x00..0x7f \\
   \hline
   ETX    & 1      & 0x03 \\
   \hline
\end{tabular}

\paragraph{Датчик обратной связи}
Выбранный энкодер имеет TTL логику, а микроконтролер имеет логику 0..+3.3.
Для этого поставили 6-разрядный двунаправленный преобразователь нарпяжения,
OPA2350 Texas Instruments.
\paragraph{Датчик тока и датчик напряжения}
\paragraph{Шины питания}
Все микросхемы можно разделить на три группы по уровню питающего напряжения:
\begin{itemize}
    \item \textbf{+12В}:
        \begin{itemize}
            \item Драйвер ШИМ.
        \end{itemize}
        Исполльзовали линейный регулятор напряжения, \textit{TPS54160DGQ Texas Instruments}
    \item \textbf{+5В}:
        \begin{itemize}
            \item Энкодер \textit{ЛИР-137А 6250-05-ПИ}
            \item Операционный усилитель OPA2350
            \item Преобразователь сигнала энекодера \textit{TXB0106PW}
            \item Интерфейс \textit{RS-422/485 SN75176B}
        \end{itemize}
        Исполльзовали линейный регулятор напряжения \textit{UA78M05CDCY Texas Instruments}
    \item \textbf{+3.3В}:
        \begin{itemize}
            \item Преобразователь сигнала энекодера \textit{TXB0106PW}
            \item Микроконтролер \textit{TMS320F28035}
        \end{itemize}
        Исполльзовали линейный регулятор напряжения, \textit{UA78M33CDCY Texas Instruments}
\end{itemize}

\paragraph{Индикация состояния}
В числе прочего для целей отладки, тестирования и последущих процедур контроля
необхрдимо иметь визуальное средство для котроля состояния платы.
Перечень ``жизненно важных'' систем платы включает:
шину бортового питания,
шину питаня \textit{12В},
шину питаня \textit{5В},
шину питаня \textit{3.3В},
бортовая шина передачи информации \textit{RS-485},
микроконтролер.

Для этого на плату установлены:
\begin{enumerate}
    \item Индикатор статуса основной программы микроконтролера,
            светодиод зеленый \textit{KP-1608MGC}.
    \item Индикатор статуса бртовой сети связи,
            светодиод голубой \textit{KPH-1608PBC-A}.
    \item Индикатор критических ошибок микроконтролера,
            светодиод красный \textit{KPH-1608SEC}.
    \item Индикатор напряжения на шине бортового питания,
            светодиод белый \textit{KPT-1608QWF-E}.
    \item Индикатор напряжения на шине питания \textit{+12В},
            светодиод желтый \textit{KPT-1608YD}.
    \item Индикатор напряжения на шине питания \textit{+5В},
            светодиод зеленый \textit{KP-1608VGC(A)}.
    \item Индикатор напряжения на шине питания \textit{+3.3В},
            светодиод оранжевый \textit{KP-1608VS}.
\end{enumerate}

\subsubsection{Принципиальная схема платы упрывления}
\subsubsection{Конструкция платы управления}
\paragraph{Выбор материала заготовки}
В качестве диэлектрика выбрали стеклотекстолит
\textit{СТЕФ-У} ГОСТ 12652-74 \cite{GOST_12652_74}.
Длительная рабочая температура от \textit{-65\textcelsius}
до \textit{+155\textcelsius}.
Предназначен для работы в условиях нормальной относительной влажности
окружающей среды при напряжении свыше \textit{1000В}.

\paragraph{Выбор класса точности платы}
Для платы выбран класc точности 3.
Печатные платы 3-гo класса - наиболее распространенные, поскольку, с одной
стороны, обеспечивают достаточно высокую плотность трассировки и монтажа, а с
другой — для их производства требуется рядовое, хотя и специализированное,
оборудование.

\paragraph{Расстояние между элементами проводящего рисунка}
Согласно \cite[Табл. 7]{GOST_23751_86} для 3 класса точности,
фольгированного стеклотекстолита:

\begin{tabular}{|c|l|}
    \hline
    Мин. расстояние, мм & Проводники \\
    \hline
    0.4 & Силовые линии - шины питания \textit{VCC} \\
    \hline
    0.2 & Остальные линии, включая внутренние шины питания \\
    \hline
\end{tabular}

\paragraph{Толщина печатного проводника}
Согласно \cite[Табл. 9]{GOST_23751_86} для 3 класса точности,
фольгированного стеклотекстолита и проводников с покрытием:

\begin{tabular}{|c|l|}
    \hline
    Мин. ширина, мм & Проводники \\
    \hline
    0.4 & Силовые линии - шины питания \textit{VCC} \\
    \hline
    0.2 & Остальные линии, включая внутренние шиниы питания \\
    \hline
\end{tabular}

\paragraph{Размеры переходных металлизированых отверстий}
Все переходные метализированые отверстия на плате, для не монтажных целей:

\begin{itemize}
    \item Диаметр площадки на фольге под отверстия --- \textit{1.2мм.}
    \item Диаметр металлизированного отверстия --- \textit{0.7мм.}
    \item Диаметр сверления --- \textit{0.8мм.}
\end{itemize}
